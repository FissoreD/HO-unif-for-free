\documentclass{pres}
\newcounter{llength}
\newcommand{\listlength}[1]{
  \setcounter{llength}{0}%
  \foreach \i in #1 {\stepcounter{llength}}%
}

\usepackage{macros}
\usepackage[export]{adjustbox}% http://ctan.org/pkg/adjustbox
\addbibresource{./bib.bib}

\newtheorem{proposition}[theorem]{Proposition}

\setbeamerfont{title}{}
\setbeamerfont{subtitle}{parent={normal text},size=\small,shape=\itshape}

\author{Fissore Davide \& Enrico Tassi} 
\title{Higher-Order unification for free}
\subtitle{Reusing the meta-language unification for the object language}

\newcommand{\sepFrame}[1]{
  \section{#1}
  \begin{frame}
    \centering
    {\usebeamerfont*{frametitle}\usebeamercolor[fg]{frametitle} #1}
  \end{frame}
}

\newcommand{\subsepFrame}[1]{
  \subsection{#1}
  \begin{frame}
    \centering
    {\usebeamerfont*{frametitle}\usebeamercolor[fg]{frametitle} #1}
  \end{frame}
}

% \institute{Université Côte d'Azur}
\date{September 10, 2024}

\begin{document}


\makeatletter
\newcommand{\addXdef}[2]{\protected@xdef#1{#1 #2}}

\newcommand{\nth}[3]{
  \foreach \x [count=\k] in #2 {\ifnum\k=#3 \addXdef{#1}{\x} \fi}
}

% takes a unif symbol (\Ue or \Uo) and a list of pairs: (lhs, rhs) of unif pb
\newcommand{\printPb}[2]{
  \listlength{#2}
  \def\tableData{}% empty table
  \foreach \x [count=\i] in #2
    {
      \foreach \y [count=\j] in \x {
        \ifthenelse{\isodd{\j}}
          {\addXdef{\tableData}{\y & #1 &}}
          {\addXdef{\tableData}{\y}}
      } 
      \ifthenelse{\equal{\i}{2}}{}{
        \ifthenelse{\equal{\i}{\thellength}}{}{\addXdef{\tableData}{&\quad}}
      }
    }
  \tableData
}

% Takes a list of triple: coq-var, elpi-var, arity
\newcommand{\printMap}[1]{
  \def\tableData{}% empty table

  \newcommand{\printMapAux}[1]{
    \nth{\tableData}{##1}{1} \addXdef{\tableData}{\mapsto} 
    \nth{\tableData}{##1}{2} \addXdef{\tableData}{} }

  \listlength{#1}
  \foreach \x [count=\i] in #1
    {
      \printMapAux{\x}
      \ifthenelse{\equal{\thellength}{\i}}{}{
        \ifthenelse{\equal{\intcalcMod{\i}{3}}{0}}
        {\addXdef{\tableData}{\\}}
        {\addXdef{\tableData}{&\quad}}
        }
    }
  \ensuremath{
    \begin{array}{ccc}
      \tableData
    \end{array}
  }
}

% Take a list of triple for eta-links: (link-type, ctx, lhs, rhs), link-type is \eta or \beta
\newcommand{\printLink}[1]{
  \listlength{#1}
  \def\tableData{}% empty table

  \newcommand{\printLinkAux}[1]{
    \nth{\tableData}{##1}{2} \addXdef{\tableData}{&}
    \nth{\tableData}{##1}{1} \addXdef{\tableData}{&} 
    \nth{\tableData}{##1}{3} \addXdef{\tableData}{&} 
    \nth{\tableData}{##1}{4}
  }

  \foreach \x [count=\i] in #1
    {
      \printLinkAux{\x}
      \ifthenelse{\equal{\thellength}{\i}}{}{
        \ifthenelse{\isodd{\i}}
          {\addXdef{\tableData}{&\quad}}
          {\addXdef{\tableData}{\\}}
      }
    }
  \ensuremath{
    \begin{array}{rcclrccl}
      \tableData
    \end{array}
  }
}

\newcommand{\putBigPar}[2]{
  \listlength{#1}
  \ifthenelse{\thellength>#2}{\Big}{}
}

\newcommand{\hideEmpty}[2]{
  \ifthenelse{\equal{#1}{{}}}{}{#2}
}

% Input is: 
%   P = a list of pairs for FoUnifPb    (leftPb, rightPb)
%   T = a list of pairs for HoUnifPb    (leftPb, rightPb)
%   M = a list of triples for mappings  (FoVar, HoVar, Arity)
%   L = a list of 4-uplet for links     (link-type, ctx, lhs, rhs)
% T, M, L can be empty
% Note: this macro is used when the length of P > 1
\newcommand{\printAlll}[4]{
  \arraycolsep=2pt
  $$
  \begin{array}{rrclrcll}
    \foUnifPb = \{ & \printPb{\Uo}{#1} &\}
    \hideEmpty{#2}{\\
      \hoUnifPb = \{ & \printPb{\Ue}{#2} &\}
    }
    \hideEmpty{#3}{\\
      \mapStore = \putBigPar{#3}{3}\{ & \multicolumn{7}{l}{
        \printMap{#3} ~\putBigPar{#3}{3}\}
    }}
    \hideEmpty{#4}{\\
      \linkStore = \putBigPar{#4}{2}\{ & \multicolumn{7}{l}{
        \printLink{#4} ~\putBigPar{#4}{2}\}
    }}
  \end{array}
  $$
}

% Same as printAlll, but the length of P is 1, the if then else
% seems not to work on the fst parameter of multicolumn
\newcommand{\printAlllSingle}[4]{
  \arraycolsep=2pt
  $$
  \begin{array}{rrcll}
    \foUnifPb = \{ & \printPb{\Uo}{#1} &\}
    \hideEmpty{#2}{\\
      \hoUnifPb = \{ & \printPb{\Ue}{#2} &\}
    }
    \hideEmpty{#3}{\\
      \mapStore = \putBigPar{#3}{3}\{ & \multicolumn{4}{l}{
        \printMap{#3} ~\putBigPar{#3}{3}\}
    }}
    \hideEmpty{#4}{\\
      \linkStore = \putBigPar{#4}{2}\{ & \multicolumn{4}{l}{
        \printLink{#4} ~\putBigPar{#4}{2}\}
    }}
  \end{array}
  $$
}
\makeatother

\begin{frame}
  \titlepage
  \tiny Supported by ANR-17-EURE-0004 \hfill \includegraphics[height=1cm,valign=c]{UCA_DS4H_France2030.png}
\end{frame}

% \begin{frame}

%   \tableofcontents

% \end{frame}

% \sepFrame{Context}

\begin{frame}[fragile]
  \frametitle{Metaprogramming for type-class resolution}

  \begin{itemize}
    \item Our goal:
          \begin{itemize}
            \item Type-class solver for Coq in Elpi
          \end{itemize}
    \item Our problem:
          \begin{itemize}
            \item The Elpi's unification algorithm differs from Coq's one
          \end{itemize}
    \item Our contribution:
          \begin{itemize}
            \item Reusing the meta-language unification for the object language
          \end{itemize}
  \end{itemize}

\end{frame}

\begin{frame}[fragile]
  \frametitle{A type-class problem in Coq}

  \begin{onlyenv}<1>
    % Instance fin_fin: ~$\forall$~n, Finite (fin n).             (* r1 *)
    % Instance nfact_dec: ~$/\forall$~n nf, Decision (nfact n nf). (* r2 *)
    \begin{coqcode}
      Instance forall_dec: ~$\forall$~A P, Finite A ~$\to$~            (* r3 *)
        (~$\forall$~x:A, Decision (P x)) ~$\to$~ Decision (~$\forall$~x:A, P x).
    \end{coqcode}
  \end{onlyenv}

  \begin{onlyenv}<2>
    % Instance fin_fin: ~$\forall$~n, Finite (fin n).             (* r1 *)
    % Instance nfact_dec: ~$\forall$~n nf, Decision (nfact n nf). (* r2 *)
    \begin{coqcode}
      Instance forall_dec: ~$\forall$~A P, Finite A ~$\to$~            (* r3 *)
        (~$\forall$~x:A, Decision (P x)) ~$\to$~ ~\colorbox{yellow}{Decision ($\forall$x:A, P x)}~.
    \end{coqcode}
  \end{onlyenv}

  \begin{onlyenv}<3->
    % Instance fin_fin: ~$\forall$~n, Finite (fin n).             (* r1 *)
    % Instance nfact_dec: ~$\forall$~n nf, Decision (nfact n nf). (* r2 *)
    \begin{coqcode}
      Instance forall_dec: ~$\forall$~A P, ~\colorbox{yellow}{Finite A} $\to$~           (* r3 *)
        ~\colorbox{yellow}{($\forall$x:A, Decision (P x))}~ ~$\to$~ Decision (~$\forall$~x:A, P x).
    \end{coqcode}
  \end{onlyenv}

  \mysep{}

  \begin{coqcode}
    Goal Decision (~$\forall$~x: fin 7, nfact x 3).             (* g *)
  \end{coqcode}

  \begin{itemize}
    % \only<2>{\item Back-chain to \texttt{forall\_dec} with }
    \item<2-> $\{A \mapsto fin\ 7; P \mapsto \lambda x.(nfact\ x\ 3)\}$
    \item<3> subgoals:\\
      \coqIn{Finite (fin 7)} and \coqIn{(~$\forall$~x:A, Decision ((~$\lambda$~ x.(nfact x 3)) x))}
  \end{itemize}


\end{frame}

\begin{frame}[fragile]
  \frametitle{Coq terms in elpi : HOAS}

  \begin{center}
    \begin{tabular}{c || c}
      Coq                          & Elpi                                       \\
      \hline
      $f$                          & \elpiIn{c"f"}                               \\
      $f\appsep{}a$                & \elpiIn{app[c"f", c"a"]}                     \\
      $\lambda (x : T).F\appsep x$ & \elpiIn{fun T (x\ app[F, x])}        \\
      $\forall (x : T), F\appsep x$ & \elpiIn{all T (x\ app[F, x])}        \\
      \dots & \dots        \\
      % $\lambda x.\lambda y.F\appsep x\appsep y$ & \elpiIn{lam (x\ lam (y\ app[F, x, y]))}        \\
      % $\lambda x.F \appsep x\appsep a$          & \elpiIn{lam (x\ app[F, x, "a"])} \\
    \end{tabular}
  \end{center}

  % Note on unification:

  % \begin{itemize}
  %   \item In coq: $\lambda x.F\appsep x$ unifies with $\lambda x.f\ x\ 3$
  %   \item In elpi: \\
  %     ``\elpiIn{lam (x\app[F, x])}'' can't unify with ``\elpiIn{lam (x\app["f", x, 3])}''\\
  %      But, ``\elpiIn{lam (x\G x)}'' unifies with ``\elpiIn{lam (x\app["f", x, 3])}''
  % \end{itemize}

\end{frame}

\begin{frame}[fragile]
  \frametitle{The above type-class problem in elpi}

  % Instance fin_fin: ~$\forall$~n, Finite (fin n).             (* r1 *)
  % Instance nfact_dec: ~$\forall$~n nf, Decision (nfact n nf). (* r2 *)

  \begin{onlyenv}<1-2>
    \begin{coqcode}
      Instance forall_dec: ~$\forall$~A P, Finite A ~$\to$~            (* r3 *)
        (~$\forall$~x:A, Decision (P x)) ~$\to$~ Decision (~$\forall$~x:A, P x).
      
      Goal Decision (~$\forall$~x: fin 7, nfact x 3).             (* g *)
    \end{coqcode}
  \end{onlyenv}


  \only<1-2>{\centering $\downarrow$}

  % finite   (app ["fin", N]).                            % r1
  % decision (app ["nfact", N, NF]).                      % r2

  \begin{onlyenv}<2>
    \begin{elpicode}
      decision (all A (x\ app [P, x])) :- finite A,         % r3
        pi w\ decision (app [P, w]).
      
      ?- decision (all (app [c"fin", c"7"])                   % g
                          (x\ app [c"nfact", x, c"3"])).
    \end{elpicode}
  \end{onlyenv}

  % \phantom{\vspace{20pt}NOTE: Elpi can unify \elpiIn{(P x)} with \elpiIn{app["nfact", x, "3"]}}


\end{frame}

\def\appnfactPyg{app\PYG{+w}{ }\PYG{k+kd}{[}c\PYG{l+s+s2}{\PYGZdq{}}\PYG{l+s+s2}{nfact}\PYG{l+s+s2}{\PYGZdq{}}\PYG{k+kd}{,}\PYG{+w}{ }x\PYG{k+kd}{,}\PYG{+w}{ }c\PYG{l+s+s2}{\PYGZdq{}}\PYG{l+s+s2}{3}\PYG{l+s+s2}{\PYGZdq{}}\PYG{k+kd}{]}}

\begin{frame}[fragile]
  \frametitle{Solving the goal in elpi}
  
  \begin{elpicode}
    decision (all A (x\ ~\colorbox{yellow}{app [P, x]}~)) :- finite A,      % r3
      pi w\ decision (app [P, w]).

    ?- decision (all (app ["fin", "7"])                 % g
                        (x\ ~\colorbox{yellow}{\appnfactPyg}~)).
  \end{elpicode}

  % \onslide<2>{\vspace{20pt}NOTE: Elpi can unify \elpiIn{(P x)} with \elpiIn{app["nfact", x, "3"]}}


\end{frame}

\begin{frame}[fragile]
  \frametitle{The idea}

  \begin{elpicode}
    decision (all A (x\ ~\colorbox{yellow}{P' x}~)) :-                      % r3
      ~\colorbox{yellow}{link P' (fun (x$\backslash$ app[P, x]))}~, 
      finite A,
      pi w\ decision (P' x).

    ?- decision (all (app ["fin", "7"])                 % g
                        (x\ ~\appnfactPyg~)).
  \end{elpicode}
  

\end{frame}

% \sepFrame{Compilation and simulation}

\begin{frame}
  \frametitle{What we propose}

  \begin{enumerate}
    \item Compilation:
    \begin{itemize}
      \item Recognize \textit{problematic subterms} $p_1,\dots,p_n$
      \item Replace $p_i$ with fresh unification variables $X_i$
      \item \textit{Link} $p_i$ with $X_i$\\
        \quad \textit{A link is a suspended unification problem}
    \end{itemize}
    \item Runtime:
    \begin{itemize}
      \item Unify $p_i$ and $X_i$ only when some conditions hold
      \item Decompile remaining links
    \end{itemize}
  \end{enumerate}

  % \mysep{}

  % \begin{description}
  %   \item[NOTE:] This unification strategy is generalizable to any meta-language
  %     when manipulating terms of the object language
  % \end{description}

\end{frame}
\def\llam{\ensuremath{{\mathcal{L}_\lambda}}\xspace}
\begin{frame}
  \frametitle{Some notations}

  \begin{itemize}
    \item \foUnifPb: the unification problems in the object language (ol)
    \item \hoUnifPb: the unification problems in the meta-language (ml)
    \item \linkStore, \mapStore: the link store, the map store
    \item Three kinds of links: \maybebeta, \maybeeta, \maybelam\footnote{\llam is a notation for the \textit{pattern fragment}}
    % \item A link in \linkStore is like $X =_\odot t$
    % \item A mapping in \mapStore is like $\{X \mapsto t\}$ 
  \end{itemize}

  \mysep

  \begin{itemize}
    \item \runFx{\foUnifPb}{n}{\rho}: the run of $n$ unif pb in the ol
    \item \runHx{\foUnifPb}{n}{\rho'}: the run of $n$ unif pb in the ml
    \item \stepF{\foUnifPb}{i}{\rho_{i-1}}{\rho_i}: the execution of the $i^{th}$ unif pb in ol
    \item \stepH{\hoUnifPb}{i}{\sigma_{i-1}}{\linkStore_{i-1}}{\sigma_i}{\linkStore_i}: the exec of the $i^{th}$ unif pb in ml
  \end{itemize}

  % Spiegare \linkStore, \foUnifPb, \hoUnifPb, step, run, substitution application
  % pattern fragment, mapping, decompilation, link has lhs and rhs

\end{frame}

\begin{frame}
  \frametitle{Proven properties}

  % \begin{proposition}[Run equivalence]
  \begin{description}[]
    \item[Run Equivalence]
    $\forall \foUnifPb, \forall n,$ if $~\foUnifPb \subseteq \llambda$
    $$
      \runFx{\foUnifPb}{n}{\rho} \land
      \runHx{\foUnifPb}{n}{\rho'}
      \Rightarrow
      \forall s \in \foUnifPb, \rho s \Eo \rho' s
    $$
    \item [Simulation fidelity]
    $\forall \foUnifPb$, in the context of  \frun and \hrun,
    % if $~\foUnifPb \subseteq \llambda$ we have that
    $\forall i \in 1 \ldots n,$
    $$\stepF{\foUnifPb}{i}{\rho_{i-1}}{\rho_i}
    \Leftrightarrow
    \stepH{\hoUnifPb}{i}{\sigma_{i-1}}{\linkStore_{i-1}}{\sigma_i}{\linkStore_i}
    $$
    % \item[Fidelity ricovery]
    % In the context of \frun and \hrun, \\
    % if 
    % $\rho_{p-1} \foUnifPb_{p} \subseteq \llambda$ 
    % (even if $\;\foUnifPb_{p} \not\subseteq \llambda$)
    % then\\
    % $$
    % \stepF{\foUnifPb}{p}{\rho_{p-1}}{\rho_p} \Leftrightarrow
    % \stepH{\hoUnifPb}{p}{\sigma_{p-1}}{\linkStore_{p-1}}{\sigma_{p}}{\linkStore_p}
    % $$
    \item[Compilation round trip]
      % If $\C{s}{t}{l}$ and $l \in \linkStore$
      % then 
      % $$\D{\linkStore}{\rho} \land \rho t \Eo \rho s$$
      If the compilation of $s$ gives a term $t$ and the stores $\linkStore$ and $\mapStore$
      % and
      % $\sigma = \{ A \mapsto t\}$ and $\mapping{X}{A}{n} \in \mapStore$
      then $\forall \sigma$,
      $$\D{\sigma}{\mapStore}{\linkStore}{\rho} \land \rho t \Eo \rho s$$
  \end{description}
  % \end{proposition}

\end{frame}

\begin{frame}
  \frametitle{Problematic subterms recognition: \maybebeta}

  \begin{itemize}
    \item $X\appsep y$ becomes \elpiIn{A y} with mapping $\mapping{X}{A}{1}$
    \item For example, $\lambda y. X\appsep y = \lambda y. f\appsep y\appsep a$
    \item Is compiled into: \elpiIn{fun (w\ A w) = fun (w\ app[c"f", w, c"a"])}
    \item Unification gives: $\{A \mapsto \elpiIn{(w\ app[c"f", w, c"a"])}\}$
    \item Decompilation of $A$ gives $\{X \mapsto \lambda y.f\appsep y\appsep a\}$
  \end{itemize}

\end{frame}

\begin{frame}
  \frametitle{Problematic subterms recognition: \maybeeta}

  \begin{itemize}
    \item $\lambda x.s \in \maybeeta$, if $\exists \rho, \rho(\lambda x.s)$ is an $\eta$-redex
    \item Detection of \maybeeta\ terms is not trivial:
    \item \begin{center}
      \begin{tabular}{lll}
        %Term & Status & Evidence \\\hline
        $\lambda x. f \appsep (A \appsep x)$ & $\in\maybeeta$ & $\rho = \{~ A \mapsto \lambda x.x ~\}$ \\
        $\lambda x. f \appsep (A \appsep x) \appsep x$ & $\in\maybeeta$ & $\rho = \{~ A \mapsto \lambda x.a ~\}$\\
        $\lambda x. f \appsep x \appsep (A \appsep x)$ & $\not\in\maybeeta$ &\\
        $\lambda x. \lambda y. f \appsep (A \appsep x) \appsep (B \appsep y \appsep x)$ & $\in\maybeeta$ & $\rho = \{~ A \mapsto \lambda x.x~;~ B \mapsto \lambda y.\lambda x.y ~\}$
      \end{tabular}
    \end{center}
    % \item Need of some primitives like \texttt{may-contract-to} and \texttt{occurs-rigidly}
  \end{itemize}

\end{frame}

\begin{frame}[fragile]
  \frametitle{Problematic subterms recognition: \maybeeta\ link resumption}

  \begin{itemize}
    \item Several conditions: like lhs is assigned to a rigid term, two
          $\eta$-link with same lhs, the rhs becomes outside \maybeeta\dots
    \item These conditions guarantee the prefixed properties !
    \item An example: %
      \printAlllSingle
        {{{f,\lambda x.(f\appsep (X\appsep x))}}}
        {{{\elpiIn{"f"},\elpiIn{A}}}}
        {{{X,\elpiIn{B},1}}}
        {{{\eta,,\elpiIn{A},\elpiIn{fun (x\ app[c"f", B x])}}}}
    \item After unification of \elpiIn{A} with \elpiIn{c"f"},
          the lhs of the link becomes rigid and
          \elpiIn{fun (x\ app[c"f", B x])} is unified with \elpiIn{fun (x\ app[c"f", x])}
    \item That is $\{B \mapsto \elpiIn{x\x}\!\!\!\}$
    \item Decompilation will assign $\lambda x.x$ to $X$
    \end{itemize}

\end{frame}

\begin{frame}[fragile]
  \frametitle{Problematic subterms recognition: \maybelam}

  \begin{itemize}
    % \item We have a term not in \llambda
    \item Example: 
      \printAlll
      {{{X,\lambda x.a},
        {(X\appsep a),a}}}
      {{{\elpiIn{A},\elpiIn{fun (x\ c"a")} },
        {\elpiIn{B},\elpiIn{"a"}}}}
      {{{X,\elpiIn{A},0}}}
      {{{\llambda,,\elpiIn{B},\elpiIn{A (c"a")}}}}
    \item After unification of \elpiIn{A} with \elpiIn{fun (x\"a")}, \\
        the lhs of the \llambda-link becomes \elpiIn{c"a"},\\
        the link is triggered and $B$ is unified to \elpiIn{c"a"}
    \item Decompilation will assign $\lambda x.a$ to $A$
  \end{itemize}

\end{frame}

% \begin{frame}
%   \frametitle{Use of heuristics}

  

% \end{frame}

\begin{frame}[fragile]
  \frametitle{Going further: the Constraint Handling Rules}

  \begin{itemize}
    \item Elpi has a CHR for goal suspension and resumption
    \item This fits well our notion of link: a suspended unification problem
  \end{itemize}

  % \begin{elpicode}
  %   pred eta i:term, i:term.
  %   eta A (fun _ _ B as T) :- not (var A), not (var B), !, 
  %     unify-left-right A T. 
  %   eta A B :- progress-eta-right B B', !, A = B'. 
  %   eta A B :- progress-eta-left  A A', !, A' = B.
  %   eta A B :- scope-check A B, get-vars B Vars, 
  %     declare_constraint (eta A B) [A|Vars].
  % \end{elpicode}

  \mysep

  This can easily introduce new unification behaviors
  \begin{itemize}
    \item We can for example mimic the unification of the ol
    \item Add heuristic for HO unification outside the pattern fragment
  \end{itemize}

  \mysep

  \begin{elpicode}
    % By def, R is not in the pattern fragment
    link-llam L R :- not (var L), unif-heuristic L R.
  \end{elpicode}

  % \begin{elpicode}
  %   link-eta L R :- not (var L), !, eta-progress-lhs L R.
  %   link-eta L R :- not (maybe-eta R), !, eta-progress-rhs L R.
  %   link-eta L R :- declare_constraint (link-eta L R) [L,R].
  % \end{elpicode}

  % \begin{elpicode}
  %   rule (N1 ~$\triangleright$~ G1 ?- link-eta (uvar X LX1) T1)   % match
  %     /  (N2 ~$\triangleright$~ G2 ?- link-eta (uvar X LX2) T2)   % remove
  %     |  (relocate LX1 LX2 T2 T2')               % condition
  %     <=> (N1 ~$\triangleright$~ G1 ?- T1 = T2').                  % new goal
  % \end{elpicode}

\end{frame}

\begin{frame}
  \frametitle{Conclusion}

  \begin{itemize}
    \item Takes advantage of the unification capabilities of the meta language
          at the price of handling problematic sub-terms on the side.
    % \item As a result our encoding takes advantage of indexing data structures
    %       and mode analysis for clause filtering.
    \item It is worth mentioning that we replace terms with variables only when
          it is strictly needed, leaving the rest of the term structure intact
          and hence \emph{indexable}.
    \item Our approach is flexible enough to accommodate different strategies
          and \emph{heuristics} to handle terms outside the pattern fragment
  \end{itemize}

\end{frame}

\end{document}
