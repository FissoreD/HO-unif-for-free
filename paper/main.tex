\documentclass[sigconf,natbib=false]{acmart}
\usepackage[]{biblatex}

\AtBeginDocument{%
  \providecommand\BibTeX{{%
    Bib\TeX}}}
\addbibresource{bib.bib}

\usepackage{myTools}
\usepackage{macros}
\usepackage{enumitem}
\usepackage{intcalc}
\usepackage{nameref}

% TODO: set this fields
%\setcopyright{cc}
%\setcctype{by}
%\copyrightyear{2024}
%\acmYear{XXXX 2024}
%\acmBooktitle{YYY}
%\acmDOI{ZZZZZZZZZZZZ}

\setcopyright{acmlicensed}  %
\copyrightyear{2024}
\acmYear{2024}
\setcopyright{rightsretained}
\acmConference[PPDP 2024]{26th International Symposium on Principles and Practice of Declarative Programming}{September 10--11, 2024}{Milano, Italy}
\acmBooktitle{26th International Symposium on Principles and Practice of Declarative Programming (PPDP 2024), September 10--11, 2024, Milano, Italy}
\acmPrice{}
\acmDOI{10.1145/3678232.3678233}
\acmISBN{979-8-4007-0969-2/24/09}

\def\githubUrl{\url{https://github.com/FissoreD/ho-unif-for-free}}

% \xspaceaddexceptions{]\}}

\newtheorem{invariant}{Invariant}
\crefname{invariant}{invariant}{invariants}
\crefname{equation}{property}{properties}

\newcommand{\appsep}{\ensuremath{\textcolor{lightgray}{\cdot}}}
\newcommand{\EqualRel}{\ensuremath{=}}
\newcommand{\nEqualRel}{\ensuremath{\new}}
\newcommand{\UnifRel}{\ensuremath{\simeq}}
\newcommand{\nUnifRel}{\ensuremath{\not\simeq}}

\newcommand{\Uo}{\texorpdfstring{\ensuremath{\UnifRel_o}\xspace}{unif\_o}}
\newcommand{\nUo}{\ensuremath{\nUnifRel_o}\xspace}
\newcommand{\Eo}{\ensuremath{\EqualRel_o}\xspace}
\newcommand{\nEo}{\ensuremath{\nEqualRel_o}\xspace}

\newcommand{\Ue}{\ensuremath{\UnifRel_m}\xspace}
\newcommand{\nUe}{\ensuremath{\nUnifRel_m}\xspace}
\newcommand{\Ee}{\ensuremath{\EqualRel_m}\xspace}
\newcommand{\nEe}{\ensuremath{\nEqualRel_m}\xspace}
\newcommand{\llambda}{\ensuremath{\mathcal{L}}\xspace}

\newcommand{\linkMacro}[1]{\ensuremath{#1}\texttt{-link}\xspace}

\newcommand{\linkbeta}{\linkMacro{\llambda}}
\newcommand{\linketa} {\linkMacro{\eta}}

%\newcommand{\Fo}{\texorpdfstring{\ensuremath{\mathcal{F}_{\!o}\xspace}}{Fo}} % space non va
%\newcommand{\Ho}{\texorpdfstring{\ensuremath{\mathcal{H}_o}\xspace}{Ho}}
\newcommand{\Fo}{\texorpdfstring{\ensuremath{\mathcal{O}}\xspace}{O}}
\newcommand{\Ho}{\texorpdfstring{\ensuremath{\mathcal{M}}\xspace}{M}}

\newcommand*{\eqtau}{\ensuremath{\mathrel{\overset{\mathrm{\tau}}{=}}}}

\newcommand{\linketaM}[3]{\ensuremath{#1 \vdash #2 =_\eta #3}}
\newcommand{\linkbetaM}[3]{\ensuremath{#1 \vdash #2 =_{\llambda} #3}}
\newcommand{\mapping}[3]{\ensuremath{#1 \mapsto #2^#3}}

\newcommand{\lhs}{lhs\xspace}
\newcommand{\rhs}{rhs\xspace}

\newcommand{\linkStore}{\texorpdfstring{\ensuremath{\mathbb{L}}\xspace}{L}}
\newcommand{\mapStore}{\texorpdfstring{\ensuremath{\mathbb{M}}\xspace}{M}}
\newcommand{\foUnifPb}{\ensuremath{\mathbb{P}}\xspace}
\newcommand{\hoUnifPb}{\ensuremath{\mathbb{Q}}\xspace}


\makeatletter
\newcommand{\addXdef}[2]{\protected@xdef#1{#1 #2}}

\newcommand{\nth}[3]{
  \foreach \x [count=\k] in #2 {\ifnum\k=#3 \addXdef{#1}{\x} \fi}
}

% takes a unif symbol (\Ue or \Uo) and a list of pairs: (lhs, rhs) of unif pb
\newcommand{\printPb}[2]{
  \listlength{#2}
  \def\tableData{}% empty table
  \foreach \x [count=\i] in #2
    {
      \foreach \y [count=\j] in \x {
        \ifthenelse{\isodd{\j}}
          {\addXdef{\tableData}{\y & #1 &}}
          {\addXdef{\tableData}{\y}}
      } 
      \ifthenelse{\equal{\i}{2}}{}{
        \ifthenelse{\equal{\i}{\thellength}}{}{\addXdef{\tableData}{&\quad}}
      }
    }
  \tableData
}

% Takes a list of triple: coq-var, elpi-var, arity
\newcommand{\printMap}[1]{
  \def\tableData{}% empty table

  \newcommand{\printMapAux}[1]{
    \nth{\tableData}{##1}{1} \addXdef{\tableData}{\mapsto} 
    \nth{\tableData}{##1}{2} \addXdef{\tableData}{^} \nth{\tableData}{##1}{3}}

  \listlength{#1}
  \foreach \x [count=\i] in #1
    {
      \printMapAux{\x}
      \ifthenelse{\equal{\thellength}{\i}}{}{
        \ifthenelse{\equal{\intcalcMod{\i}{3}}{0}}
        {\addXdef{\tableData}{\\}}
        {\addXdef{\tableData}{&\quad}}
        }
    }
  \ensuremath{
    \begin{array}{ccc}
      \tableData
    \end{array}
  }
}

% Take a list of triple for eta-links: (link-type, ctx, lhs, rhs), link-type is \eta or \beta
\newcommand{\printLink}[1]{
  \listlength{#1}
  \def\tableData{}% empty table

  \newcommand{\printLinkAux}[1]{
    \nth{\tableData}{##1}{2} \addXdef{\tableData}{\vdash&} 
    \nth{\tableData}{##1}{3} \addXdef{\tableData}{&=_}
    \nth{\tableData}{##1}{1} \addXdef{\tableData}{&} 
    \nth{\tableData}{##1}{4}
  }

  \foreach \x [count=\i] in #1
    {
      \printLinkAux{\x}
      \ifthenelse{\equal{\thellength}{\i}}{}{
        \ifthenelse{\isodd{\i}}
          {\addXdef{\tableData}{&\quad}}
          {\addXdef{\tableData}{\\}}
      }
    }
  \ensuremath{
    \begin{array}{rcclrccl}
      \tableData
    \end{array}
  }
}

\newcommand{\putBigPar}[2]{
  \listlength{#1}
  \ifthenelse{\thellength>#2}{\Big}{}
}

\newcommand{\hideEmpty}[2]{
  \ifthenelse{\equal{#1}{{}}}{}{#2}
}

% Input is: 
%   P = a list of pairs for FoUnifPb    (leftPb, rightPb)
%   T = a list of pairs for HoUnifPb    (leftPb, rightPb)
%   M = a list of triples for mappings  (FoVar, HoVar, Arity)
%   L = a list of 4-uplet for links     (link-type, ctx, lhs, rhs)
% T, M, L can be empty
% Note: this macro is used when the length of P > 1
\newcommand{\printAlll}[4]{
  \arraycolsep=2pt
  $$
  \begin{array}{rrclrcll}
    \foUnifPb = \{ & \printPb{\Uo}{#1} &\}
    \hideEmpty{#2}{\\
      \hoUnifPb = \{ & \printPb{\Ue}{#2} &\}
    }
    \hideEmpty{#3}{\\
      \mapStore = \putBigPar{#3}{3}\{ & \multicolumn{7}{l}{
        \printMap{#3} ~\putBigPar{#3}{3}\}
    }}
    \hideEmpty{#4}{\\
      \linkStore = \putBigPar{#4}{2}\{ & \multicolumn{7}{l}{
        \printLink{#4} ~\putBigPar{#4}{2}\}
    }}
  \end{array}
  $$
}

% Same as printAlll, but the length of P is 1, the if then else
% seems not to work on the fst parameter of multicolumn
\newcommand{\printAlllSingle}[4]{
  \arraycolsep=2pt
  $$
  \begin{array}{rrcll}
    \foUnifPb = \{ & \printPb{\Uo}{#1} &\}
    \hideEmpty{#2}{\\
      \hoUnifPb = \{ & \printPb{\Ue}{#2} &\}
    }
    \hideEmpty{#3}{\\
      \mapStore = \putBigPar{#3}{3}\{ & \multicolumn{4}{l}{
        \printMap{#3} ~\putBigPar{#3}{3}\}
    }}
    \hideEmpty{#4}{\\
      \linkStore = \putBigPar{#4}{2}\{ & \multicolumn{4}{l}{
        \printLink{#4} ~\putBigPar{#4}{2}\}
    }}
  \end{array}
  $$
}
\makeatother

\begin{document}

% \title{HO unification from object language to meta language}
\title{Higher-Order unification for free!}
\subtitle{Reusing the meta-language unification for the object language}

\author{Davide Fissore}
\email{davide.fissore@inria.fr}
\affiliation{%
  \institution{Université Côte d'Azur, Inria}
  % \city{Nice}
  \country{France}}

\author{Enrico Tassi}
\email{enrico.tassi@inria.fr}
\affiliation{%
  \institution{Université Côte d'Azur, Inria}
  % \city{Nice}
  \country{France}}

\begin{abstract}
  Specifying and implementing a proof system from scratch requires significant effort.
  Logical Frameworks and Higher Order Logic Programming Languages provide
  dedicated, high-level meta languages to facilitate this task in two
  % key 
  ways: 1) variable binding and substitution are for free when meta language binders
  represent object logic ones; 2) proof construction, and %even 
  proof search, are
  greatly simplified by leveraging the unification procedure provided by the meta language.
  Notable examples of meta languages are Elf~\cite{elf}, Twelf~\cite{twelf},
  $\lambda$Prolog~\cite{miller_nadathur_2012}, Beluga~\cite{Beluga}, Abella~\cite{gacek2008abella} and
  Isabelle~\cite{10.1007/978-3-540-71067-7_7}
  which have been %utilized
  used to implement or specify %various
  many
  formal systems such as
  First Order Logic~\cite{felty88cade},
  Set Theory~\cite{10.1007/BF00881873},
  Higher Order Logic~\cite{books/sp/NipkowPW02}, and %even
  the Calculus of
  Constructions~\cite{felty93lics}.

  The object logic we are interested in is
  Coq's type theory~\cite{Coq-refman}. We aim to
  develop a higher-order unification-based 
  proof search procedure %for it 
  using the meta language
  Elpi~\cite{dunchev15lpar}, a dialect of $\lambda$Prolog.
  Elpi's equational theory includes
  $\beta\eta$-equivalence and features a
  higher-order unification procedure \Ue 
  %restricted to the
  for the
  pattern fragment~\cite{miller92jsc}.
  Elpi offers an encoding of Coq terms that is suitable
  for meta programming~\cite{tassi:hal-01637063,tassi:hal-01897468,gregoire:hal-03800154,newtc}
  but that restricts \Ue to first-order unification problems only.
  We refer to this basic encoding as \Fo.
  
  In this paper we translate unification problems in \Fo{} 
  to an alternative encoding called \Ho, from which we derive \Uo,
  the higher-order unification procedure of \Fo{}. \Uo
  honours $\beta\eta$-equivalence for terms
  within the pattern fragment, and allows for the use of
  heuristics when the terms fall outside the pattern fragment.
  Moreover, as \Uo{} delegates most of the work to \Ue,
  it can be used to efficiently simulate a logic program in \Fo{} by
  taking advantage of
  unification-related optimizations of the meta language, such as clause indexing.

\end{abstract}


\keywords{Logic Programming, Meta-Programming, Higher-Order Unification}

\maketitle

\section{Introduction}
\label{sec:intro}
We aim to implement a
form of proof search known as type-class resolution~\cite{wadler89,sozeau08}
for Coq's type system~\cite{Coq-refman}.
Type-class solvers are unification based proof search procedures
reminiscent of Prolog, which back-chain lemmas taken
from a database of ``type-class instances''. Given this
analogy with Logic Programming we want to leverage the
Elpi meta programming language~\cite{tassi:hal-01637063},
a dialect of $\lambda$Prolog, already used to extend
Coq in various ways~\cite{tassi:hal-01637063,tassi:hal-01897468,gregoire:hal-03800154,newtc}.

The use of a meta language facilitates the implementation of a
proof system in two key ways. The first and most well-know one is that variable binding and
substitution come for free. %, exploited in all works mentioned above.
The second one is that these meta languages come equipped with some form
of unification, a cornerstone of proof construction and proof search.
% exploited only
%in the notable case of of Higher Order Logic~\cite{books/sp/NipkowPW02}:
%the meta language Isabelle is such a good match for HOL that it could used to
%implement an interactive proof system for the object logic, in addition to
%it specification.\\
In this paper we focus on this last aspect of the implementation of
a Prolog-like proof system, precisely \emph{how to reuse the higher-order
unification procedure of the meta language in order to simulate a
higher-order logic program for the object language}.

We take as an example the \coqIn{Decision} and \coqIn{Finite} type classes
from the Stdpp
library~\cite{JUNG_KREBBERS_JOURDAN_BIZJAK_BIRKEDAL_DREYER_2018}.
The class  \coqIn{Decision}
identifies predicates equipped with a decision procedure, while
\coqIn{Finite} identifies types whose inhabitants can be enumerated in a (finite) list.
The following three type-class instances state that:
1) the type of natural numbers smaller than \coqIn{n}, called \coqIn{fin n},
is finite;
2) the predicate \coqIn{nfact n nf}, relating a natural number
\coqIn{n} to the number of its prime factors \coqIn{nf}, is decidable;
3) the universal closure of a predicate has a decision procedure if
its domain is finite and if the predicate is decidable.

\begin{coqcode}
Instance fin_fin: ~$\forall$~n, Finite (fin n).             (* r1 *)
Instance nfact_dec: ~$\forall$~n nf, Decision (nfact n nf). (* r2 *)
Instance forall_dec: ~$\forall$~A P, Finite A ~$\to$~            (* r3 *)
  ~($\forall$~x:A, Decision (P x)) ~$\to$~ Decision (~$\forall$~x:A, P x).
\end{coqcode}

\noindent Given this database, a type-class solver is expected to
prove the following statement automatically:

\begin{coqcode}
  Decision (~$\forall$~x: fin 7, nfact x 3)                   (* g *)
\end{coqcode}

\noindent
The proof found by the solver back-chains on rule 3 (the only rule
about the $\forall$ quantifier), and then solves the premises with
rules 1 and 2 respectively.
Note that rule 3 features a second-order parameter \coqIn{P} that represents
a function of type \coqIn{A ~$\to$~ Prop} (a 1-argument predicate over \coqIn{A}).
The solver infers a value for \coqIn{P} by unifying the conclusion
of rule 3 with the goal, and in particular, it has to solve the unification
problem \coqIn{P x = nfact x 3}. This higher-order problem falls in the so-called 
pattern fragment and admits a unique solution:
\begin{elpicode}
P = ~$\lambda$~x.nfact x 3                                       ~\customlabel{solution:intro}{(\rho)}~
\end{elpicode}
In the rest of this paper, we refer to the pattern fragment as
\llambda~\cite{miller1991}.

In order to implement such a search in Elpi, we shall describe the encoding
of Coq terms and then the encoding of instances as rules.
Elpi comes equipped with
a Higher-Order Abstract Syntax (HOAS~\cite{10.1145/53990.54010}) datatype of
Coq terms, called \elpiIn{tm}, that includes (among others) the following
constructors:

\begin{elpicode}
type lam  tm -> (tm -> tm) -> tm.     % lambda abstraction
type app  list tm -> tm.              % n-ary application
type all  tm -> (tm -> tm) -> tm.     % forall quantifier
type con  string -> tm.               % constants
\end{elpicode}

\noindent
Following the standard syntax of $\lambda$Prolog~\cite{miller_nadathur_2012},
the meta-level binding of a variable \elpiIn{x} in an expression
\elpiIn{e} is written as <<\elpiIn{x\ e}>>, while square brackets delimit a
list of terms separated by comma. For example, the term
<<\coqIn{~$\forall$~y:t, nfact y 3}>> is encoded as follows:

\begin{elpicode}
all (con"t") y\ app [con"nfact", y, con"3"]
\end{elpicode}

\noindent
We now illustrate a naive encoding of the three instances above as higher-order
logic programming rules: capital letters denote rule
parameters; \elpiIn{:-} separates the rule's head from the premises, and
<<\elpiIn{pi w\ p}>> introduces a fresh nominal constant \elpiIn{w}
for the premise \elpiIn{p}.

\begin{elpicode}
finite   (app [con"fin", N]).                         ~\customlabel{clause:r1}{(r1)}~
decision (app [con"nfact", N, NF]).                   ~\customlabel{clause:r2}{(r2)}~
decision (all A x\ app [P, x]) :- finite A,           ~\customlabel{clause:r3}{(r3)}~
  pi w\ decision (app [P, w]).
\end{elpicode}

\noindent
Unfortunately this encoding of rule \ref{clause:r3} does not work
since it uses the predicate \coqIn{P} as a first order term: for the meta
language its type is \elpiIn{tm}. If we try to back-chain the rule
\ref{clause:r3} on the encoding of the goal \ref{goal:g} given here

\begin{elpicode}
decision (all (app [con"fin", con"7"]) x\              ~\customlabel{goal:g}{(g)}~
  app [con"nfact", x, con"3"]).
\end{elpicode}

\noindent
we obtain an unsolvable unification problem \ref{problem:p}:
the two lists of terms have different lengths!
%The root cause is that
%\ref{problem:p} is an higher order in CIC, but becomes
%first order in the meta language due to the ``naive'' encoding.

\begin{elpicode}
app [con"nfact", x, con"3"] = app [P, x]               ~\customlabel{problem:p}{(p)}~
\end{elpicode}

\noindent
The problem above has no solution in a first-order language
where \elpiIn{P} can only stand for a predicate symbol, while in a higher-order
language \elpiIn{P} can stand for any 1-argument predicate.

In this paper we study a more sophisticated encoding of Coq terms and rules
that, on a first approximation, would reshape \ref{clause:r3} as follows:

\begin{elpicode}
decision (all A x\ Pm x) :- link Pm P A, finite A,    ~\customlabel{clause:r3a}{(r3')}~
  pi x\ decision (app [P, x]).
\end{elpicode}

\noindent
Since \elpiIn{Pm} is a higher-order unification variable
of type \elpiIn{tm -> tm},
with \elpiIn{x}
in its scope, the unification problem \ref{problem:pa}
admits one solution:

\begin{elpicode}
app [con"nfact", x, con"3"] = Pm x                    ~\customlabel{problem:pa}{(p')}~
Pm = x\ app [con"nfact", x, con"3"]                   ~\customlabel{solution:pm}{(\sigma)}~
\end{elpicode}

\noindent
Once the head of rule \ref{clause:r3a} unifies with the goal \ref{goal:g},
the premise <<\elpiIn{link Pm A P}>> brings the assignment \ref{solution:pm}
back to the domain \elpiIn{tm} of Coq terms, obtaining the expected solution
\ref{solution:intro}:

\begin{elpicode}
P = lam A x\ app [con"nfact", x, con"3"]
\end{elpicode}

\noindent
This simple example is sufficient to show that the encoding we seek
is not trivial and does not only concern the head of rules, but the entire sequence
of unification problems that constitute the execution of a logic program.

In fact
the solution for \elpiIn{P} above generates a
(Coq) $\beta$-redex in the second premise (the predicate
under the \elpiIn{pi w\ }\hspace{-0.4em}).
% We show below the premise before and
% after the instantiation of \elpiIn{P}:
% \begin{elpicode}
% decision (app [                   P                  , w])
% decision (app [lam A (a\ app [con"nfact", a, con"3"]), w])
% \end{elpicode}
% \noindent
In turn, this redex prevents rule \ref{clause:r2} from backchaining properly since
the following unification problem has no solution:

\begin{elpicode}
app [ lam A (a\ app [con"nfact", a, con"3"]) , x] =
app [ con"nfact"                             , N, NF]
\end{elpicode}

\noindent
The root cause of the problems we outlined in this example
is a subtle mismatch between the equational theories of the meta language
and the object language, which in turn makes the 
unification procedures of the meta language weak.
The equational theory of the meta-language Elpi encompasses
$\beta\eta$-equivalence and its unification procedure can solve higher-order
problems in the pattern fragment. Although 
the equational theory of Coq is much richer, for efficiency and predictability
reasons automatic proof search procedures typically employ a unification
procedure that only captures a $\beta\eta$-equivalence and only operates
in \llambda. The similarity is striking, but one needs to exercise some caution in order to
simulate a higher-order logic program in Coq using the unification of Elpi.

\paragraph{Contributions}
In this paper we identify a minimal language \Fo{} in which the problems
sketched in the introduction can be formally described.
We detail an encoding of a logic program in \Fo{} to a strongly related
logic program in \Ho, the language of the meta language, and we show that
the higher-order unification procedure of the meta language \Ue{} can be
used to simulate a higher-order unification procedure \Uo for
the object language that features $\beta\eta$-conversion. We show how \Uo
can be extended with heuristics to deal with problems outside the pattern
fragment.\\
Section~\ref{sec:problem-statement} states the problem formally and gives the
intuition behind our solution; \cref{sec:simulation} sets up a basic
simulation of first-order logic programs, \cref{sec:llam} and \cref{sec:eta}
extend it to higher-order logic programs in the pattern fragment
while \cref{sec:beta} goes beyond the pattern fragment.
Section \ref{sec:implementation} discusses the implementation in Elpi.\\
The $\lambda$Prolog code discussed in the paper can be accessed at the
address \githubUrl.

\section{Problem statement and solution} %%%%%%%%%%%%%%%%%%%%%%
\label{sec:problem-statement}

\newcommand{\specunif}[3]{
  \{#3_1, #3_2\} \subseteq \llambda \Rightarrow %
    \exists \rho, %
      \rho #3_1 #1 \rho #3_2  %
        \Leftrightarrow #3_1 #2 #3_2 \mapsto \rho' \subseteq \rho
}


\newcommand{\unifcorrectX}[4]{\ensuremath{
  % \forall \rho #3_1 #3_2, %
    % \{#3_1, #3_2\} \subseteq \llambda \Rightarrow %
   #4 \Rightarrow
      #3_1 #2 #3_2 \mapsto \rho
        \Rightarrow
          \rho #3_1 #1 \rho #3_2  %
}}
\newcommand{\unifcorrect}[3]{\unifcorrectX{#1}{#2}{#3}{ \{#3_1, #3_2\} \subseteq \llambda}}

\newcommand{\unifcompleteX}[4]{\ensuremath{
  % \forall #3_1 #3_2, %
    % \{#3_1, #3_2\} \subseteq \llambda \Rightarrow %
    #4 \Rightarrow
      % \forall \rho, %
        \rho #3_1 #1 \rho #3_2  %
          \Rightarrow \exists \rho', #3_1 #2 #3_2 \mapsto \rho' \land \rho' \subseteq \rho
}}
\newcommand{\unifcomplete}[3]{\unifcompleteX{#1}{#2}{#3}{\{#3_1, #3_2\} \subseteq \llambda}}

\newcommand{\maybeeta}{\texorpdfstring{\ensuremath{\Diamond\eta}\xspace}{maybeeta}}
\newcommand{\maybebeta}{\texorpdfstring{\ensuremath{\Diamond\beta}\xspace}{maybebeta}}
\newcommand{\notllambda}{\texorpdfstring{\ensuremath{\Diamond\llambda}\xspace}{maybellam}}
%\newcommand{\notllambda}{\ensuremath{\Diamond\beta}\xspace}

Even if we encountered the problem working on Coq's Calculus of Inductive
Constructions, we devise
a minimal setting to ease its study. In this setting, we have
a (first order) language \Fo{} with a rich equational
theory and a \Ho{} language with a simpler one.


\subsection{Preliminaries: \Fo{} and \Ho{}}

In order to reason about unification we provide a description of the
\Fo{} and \Ho languages where unification variables
are first-class terms, i.e. they have a concrete syntax as
shown in \cref{code:common-terms}.
Unification variables
in \Fo{} (\elpiIn{o-uva} term constructor) have no explicit scope:
the ``arguments'' of a higher-order variable are given via the \elpiIn{o-app}
constructor. For example the term <<\coqIn{P x}>> is represented as
<<\elpiIn{o-app [o-uva N, x]}>>, where \elpiIn{N} is the memory address
of \elpiIn{P} and \elpiIn{x} is a bound variable.
In \Ho{} the representation of <<\coqIn{P x}>> is instead <<\elpiIn{uva N [x]}>>,
since unification variables are higher-order and come equipped with an
explicit scope.
%
% We keep these languages
%minimal, for example, we omit the \elpiIn{all} quantifier of CIC we used
%in the example in Section~\ref{sec:intro} together with the type notation of
%terms carried by the \elpiIn{lam} constructor.
%
{
\setlength{\abovecaptionskip}{0pt}
\setlength{\belowcaptionskip}{-13pt}
\begin{figure}[b] %[H]
  \begin{tabular}{ll}
  \begin{minipage}{0.21\textwidth}
   \inputrawelpicode{code/fo_tm}
  \end{minipage}
  &
  \begin{minipage}{0.24\textwidth}
   \inputrawelpicode{code/ho_tm}
  \end{minipage}
  \end{tabular}\vspace{4pt}
  \caption{The \Fo{} and \Ho languages}\vspace{0.3em}
  \label{code:common-terms}
  \Description[code:common-terms]{code:common-terms}
\end{figure}
}
%\todo{use faddr for Fo}
%
%faddr and addr are memory addresses, the details are given in bla,
%here they are just unique identifiers for unif variables and their types
%do now allow to mistake one for the other.
%\todo{cleanup}

\paragraph{Notational conventions}

When we write \Ho terms outside code blocks we follow the
usual $\lambda$-calculus notation, reserving $f, g, a, b$ for constants,
$x, y, z$ for bound variables and $X, Y, Z, F, G, H$ for unification variables.
However, we need to
distinguish between the ``application'' of a unification variable
to its scope and the application of a term to a list of arguments.
We write the scope of unification variables in subscript
while we use juxtaposition for regular application.
Here are few examples:\\
\vspace{4pt}
{
\setlength{\tabcolsep}{1em}
\begin{tabular}{ll}
  $f\appsep a$                  & \elpiIn{app [con "f", con "a"]}\\
  $\lambda x.\lambda y.F_{x y}$ & \elpiIn{lam x\ lam y\ uva F [x, y]} \\
  $\lambda x.F_{x} \appsep a$   & \elpiIn{lam x\ app [uva F [x], con "a"]} \\
  $\lambda x.F_{x} \appsep x$   & \elpiIn{lam x\ app [uva F [x], x]} \\
\end{tabular}
}
\vspace{4pt}

\noindent
When it is clear from the context, we shall use the same syntax for \Fo{} terms
(although we never use subscripts for unification variables).\\
We use $s$, $s_1$, \ldots for terms in \Fo{} and $t$, $t_1$, \ldots for
terms in \Ho{}.

\subsection{Equational theories and unification}

% In order to specify unification, we need to
% define the equational theory and
% substitution (unification-variable assignment).

\subsubsection{Term equality: \Eo and \Ee}
For both languages, we extend the equational theory
over ground terms to the full language by adding reflexivity for
unification variables, i.e. a unification variable is equal to itself
but different to any other term.

The first four rules are common to both equalities
and define the usual congruence over terms. Since
we use an HOAS encoding, they also capture $\alpha$-equivalence.
In addition to that, \Eo has rules for $\eta$ and $\beta$-equivalence.

\begin{elpicode}
  ~\PYG{k+kd}{type} \PYG{n+nf}{(\Eo)} \PYG{k+kt}{to -> to -> o}~.                              ~($\customlabel{pred:fequal}{\Eo}$)~
  o-con X ~$\Eo$~ o-con X.
  o-app A ~$\Eo$~ o-app B :- forall2 (~$\Eo$~) A B.
  o-lam F ~$\Eo$~ o-lam G :- pi x\ x ~$\Eo$~ x => F x ~$\Eo$~ G x.    ~($\customlabel{clause:lam-lam}{\lambda\lambda}$)~
  o-uva N ~$\Eo$~ o-uva N.
  o-lam F ~$\Eo$~ T :-                                       ~($\customlabel{clause:eta1}{\eta_l}$)~
    pi x\ beta T [x] (T' x), x ~$\Eo$~ x => F x ~$\Eo$~ T' x.
  T ~$\Eo$~ o-lam F :-                                       ~($\customlabel{clause:eta2}{\eta_r}$)~
    pi x\ beta T [x] (T' x), x ~$\Eo$~ x => T' x ~$\Eo$~ F x.
  o-app [o-lam X|L] ~$\Eo$~ T :- beta (o-lam X) L R, R ~$\Eo$~ T. ~($\customlabel{clause:beta1}{\beta_l}$)~
  T ~$\Eo$~ o-app [o-lam X|L] :- beta (o-lam X) L R, T ~$\Eo$~ R. ~($\customlabel{clause:beta2}{\beta_r}$)~
\end{elpicode}


\begin{elpicode}
  ~\PYG{k+kd}{type} \PYG{n+nf}{(\Ee)} \PYG{k+kt}{tm -> tm -> o}~.
  con C ~\Ee~con C.
  app A ~\Ee~app B :- forall2 ~(\Ee)~ A B.
  lam F ~\Ee~lam G :- pi x\ x ~\Ee~x => F x ~\Ee~G x.
  uva N A ~\Ee~uva N B :- forall2 ~(\Ee)~ A B.
\end{elpicode}

\noindent
Note that the symbol \elpiIn{|} separates the head of a list from the tail;
that \elpiIn{forall2} applies a binary predicate to two lists (of equal
length); and that \elpiIn{=>} is $\lambda$Prolog syntax for
implication and is used to augment the
current program with an extra rule. Both procedures use implication to
assume that the fresh nominal constants introduces by \elpiIn{pi} are
equal to themselves.

The main point in showing these equality tests is to remark how
weaker \Ee is, and to identify the four rules that need special
treatment in the implementation of \Uo.
For brevity, we omit the code of \elpiIn{beta}:
it is sufficient to know that <<\elpiIn{beta F L R}>> computes in \elpiIn{R} the
weak-head normal form of <<\elpiIn{o-app [F|L]}>>.

\paragraph{Substitution: $\rho s$ and $\sigma t$}

We write $\sigma = \{~ X \mapsto t ~\}$ for the substitution that assigns
the term $t$ to the variable $X$.
We write $\sigma t$ for the application of
the substitution to a term $t$, and $\sigma X = \{~ \sigma t ~|~ t \in X ~\}$ when
$X$ is a set of terms.
We write $\sigma \subseteq \sigma'$ when $\sigma$ is more
general than $\sigma'$.
% The domain of a substitution is the set of unification variables for which
% it provides an assignment.
% We write $\sigma \cup \sigma'$ to denote the concatenation of
% two substitutions whose domains are disjoint.
We shall use $\rho$ for \Fo{} substitutions,
and $\sigma$ for the \Ho ones.
For brevity, in this section, we consider
the substitution for \Fo{} and \Ho{} identical.
We defer to \cref{sec:grounwork} a more precise description
pointing out their differences.

\paragraph{Term unification: \Uo vs. \Ue}

% Although we provide an implementation of the meta-language unification \Ue
% in the supplementary material (that we used for testing purposes) we only
% describe its signature here.
\Ho{}'s unification signature is:

\begin{elpicode}
  ~\PYG{k+kd}{type} \PYG{n+nf}{(\Ue)} \PYG{k+kt}{tm -> tm -> subst -> subst -> o}~.
\end{elpicode}


\noindent
We write 
$\sigma t_1 \Ue \sigma t_2 \mapsto \sigma'$ when
$\sigma t_1$ and $\sigma t_2$ unify with substitution $\sigma'$.
Note that $\sigma'$ is a refined (i.e. extended) version of $\sigma$; this is
reflected by the signature above that relates two substitutions.
We write $t_1 \Ue t_2 \mapsto \sigma'$ when
the initial substitution $\sigma$ is empty.
% Note that if $\sigma t_1 \Ue \sigma t_2 \mapsto \sigma'$ then
% the domains of $\sigma$ and $\sigma'$ are disjoint.
We write \llambda as the set of terms that are in the pattern-fragment, i.e.
every unification variable is applied to a list of distinct names.

The specification of a ``good'' unification procedure restricted
to a domain $\mathcal{D}$ is the following:

\begin{proposition}[Good unification]\label{prop:good-unif}
A good unification $\UnifRel\;$ for equality $\;\EqualRel\;$ in the domain
$\mathcal{D}$ satisfies the following properties:%\noteD{use $\sigma$, not $\rho$}
  \begin{gather}
    \unifcorrectX{\EqualRel}{\UnifRel}{t}{\{~t_1,~ t_2~\} \subseteq \mathcal{D}}\\
    \unifcompleteX{\EqualRel}{\UnifRel}{t}{\{~ t_1,~ t_2 ~\} \subseteq \mathcal{D}}
  \end{gather}
\end{proposition}  

\noindent
The meta language of choice is expected to provide 
an implementation of \Ue that is a good unification for \Ee
in \llambda.

Even if we provide an implementation of the object-language unification
\Uo{} in \cref{sec:founif}, our final goal is to simulate the execution of
an entire logic-program.

\subsection{The problem: logic-program simulation}
We represent a logic program \emph{run} in \Fo{} as
a sequence of $n$ \emph{steps}. Each step $p$ consists in
unifying two terms, $\foUnifPb_{p_l}$ and
$\foUnifPb_{p_r}$, taken from the list of all unification
problems \foUnifPb.
We write $\foUnifPb_p = \{~ \foUnifPb_{p_l},\, \foUnifPb_{p_r} ~\}$
and $s \in \foUnifPb \Leftrightarrow \exists p, s \in \foUnifPb_p$.
% \footnote{If the same rule is used multiple times in a run we
% consider as many copies as needed of the terms composing the
% rules, with fresh unification variables each time. Also, we do not
% model backtracking directly, but rather the fact that a step in a run may
% fail, and in that case the entire run is said to fail. A regular logic program
% that succeeds is modeled as a many runs, with possibly overlapping starts,
% where at least one run does not end in a failure. }
The composition of these steps starting from the
empty substitution $\rho_0$ produces the final
substitution $\rho_n$, which is the result of the
logic program execution.
%
\newcommand{\C}[4]{\ensuremath{\langle #1 \rangle}\mapsto(#2,#3,#4)}
\newcommand{\D}[4]{\ensuremath{\langle #1,#2,#3 \rangle^{-1}\mapsto #4}}
\newcommand{\progress}{\ensuremath{\mathrm{progress}}\xspace}
\newcommand{\fstep}{\ensuremath{{\mathrm{step}}_o}\xspace}
\newcommand{\hstep}{\ensuremath{{\mathrm{step}}_m}\xspace}
\newcommand{\frun}{\ensuremath{{\mathrm{run}}_o}\xspace}
\newcommand{\hrun}{\ensuremath{{\mathrm{run}}_m}\xspace}
\newcommand{\stepF}[4]{\ensuremath{\fstep(#1,#2,#3) \mapsto #4}}
\newcommand{\stepFD}[4]{%
\ensuremath{#3 #1_{#2_l} \Uo #3 #1_{#2_r} \mapsto #4}}
\newcommand{\stepH}[6]{\ensuremath{\hstep(#1,#2,#3,#4) \mapsto (#5, #6)}}
\newcommand{\stepHD}[6]{\ensuremath{%
#3 #1_{#2_l} \Ue #3 #1_{#2_r} \mapsto #4 \land \progress(#6,#4) \mapsto (#6',#5)}}
\newcommand{\runF}[3]{\ensuremath{\frun(#1,#2) \mapsto #3_{#2}}}
\newcommand{\runFx}[3]{\ensuremath{\frun(#1,#2) \mapsto #3}}
\newcommand{\runFD}[2]{\ensuremath{%
\bigwedge_{p = 1}^{#2} \stepF{#1}{p}{\rho_{p-1}}{\rho_{p}}}}
\newcommand{\runH}[3]{\ensuremath{\hrun(#1,#2) \mapsto #3_{#2}}}
\newcommand{\runHx}[3]{\ensuremath{\hrun(#1,#2) \mapsto #3}}
\newcommand{\runHD}[3]{\ensuremath{%
\bigwedge_{p = 1}^{#2} \stepH{#1}{p}{\sigma_{p-1}}{#3_{p-1}}{\sigma_{p}}{#3_p}}}
\newcommand{\deff}{\ensuremath{\stackrel{{\scriptscriptstyle def}}{=\!=\!\!\!=}}}
%
$$
\begin{array}{l}
\stepF{\foUnifPb}{p}{\rho}{\rho'}
\deff
\stepFD{\foUnifPb}{p}{\rho}{\rho'}\vspace{2pt}\\
\runF{\foUnifPb}{n}{\rho}
\deff
\runFD{\foUnifPb}{n}
\end{array}
$$

\noindent
In order to simulate a logic program we start by compiling
\foUnifPb to \hoUnifPb, i.e.
each \Fo{}-term $s \in \foUnifPb$ is translated to a \Ho{}-term $t \in \hoUnifPb$.
We write this translation as $\C{s}{t}{m}{l}$. The implementation of the compiler
is detailed in \cref{sec:simulation,sec:eta,sec:beta}, here we just point
out that it additionally produces a variable map $m$ and a list of links $l$.
The variable map connects unification variables in \Ho to variables
in \Fo{} and is used to ``decompile'' the substitution, that is written
$\D{\sigma}{m}{l}{\rho}$. Links are an accessory piece of information whose
description is deferred to \cref{sec:nutshell}.

Then we simulate each run in \Fo{} with a run in \Ho as follows:
%
%Note that $\sigma_0$ is the empty substitution.
$$
\begin{array}{l}
\stepH{\hoUnifPb}{p}{\sigma}{\linkStore}{\sigma''}{\linkStore'} \deff\vspace{2pt}\\
  \qquad\stepHD{\hoUnifPb}{p}{\sigma}{\sigma'}{\sigma''}{\linkStore}\vspace{2pt}\\
  \runH{\foUnifPb}{n}{\rho} \deff \vspace{2pt}\\
  \qquad \hoUnifPb \times \mapStore \times \linkStore_0 = \{ (t,m,l) | s \in \foUnifPb, \C{s}{t}{m}{l} \}\vspace{2pt}\\
  \qquad \runHD{\hoUnifPb}{n}{\linkStore}\vspace{2pt}\\
  \qquad \D{\sigma_{n}}{\mapStore}{\linkStore_{n}}{\rho_{n}}
\end{array}
$$

\noindent
By analogy with \foUnifPb, we write $\hoUnifPb_{p_l}$ and $\hoUnifPb_{p_r}$
for the two \Ho{} terms being unified at step $p$, and we write $\hoUnifPb_p$
for the set $\{~ \hoUnifPb_{p_l},\, \hoUnifPb_{p_r} ~\}$.\\
The \hstep{} procedure is made of two sub-steps: a call to the meta-language
unification and a check for \progress{} on the set of links, that intuitively
will compensate for the weaker equational theory honored by \Ue.
Procedure \hrun{} compiles all terms in \foUnifPb{}, then executes each step, and
finally decompiles the solution.
We claim:

\begin{proposition}[Run equivalence]\label{prop:simulation}
$\forall \foUnifPb, \forall n,$ if $~\foUnifPb \subseteq \llambda$
$$
  \runFx{\foUnifPb}{n}{\rho} \land
  \runHx{\foUnifPb}{n}{\rho'}
  \Rightarrow
  \forall s \in \foUnifPb, \rho s \Eo \rho' s
$$
\end{proposition}

\noindent
That is, the two executions give equivalent results if all terms in
\foUnifPb are in the pattern fragment. Moreover:

\begin{proposition}[Simulation fidelity]\label{prop:fidelity}
In the context of  \frun and \hrun, if $~\foUnifPb \subseteq \llambda$ we have that
$\forall p \in 1 \ldots n,$
$$
\stepF{\foUnifPb}{p}{\rho_{p-1}}{\rho_p}
\Leftrightarrow
\stepH{\hoUnifPb}{p}{\sigma_{p-1}}{\linkStore_{p-1}}{\sigma_p}{\linkStore_p}
$$
\end{proposition}
\noindent
In particular, this property guarantees that a \emph{failure} in the \Fo{} run
is matched by a failure in \Ho{} \emph{at the same step}. We consider this
property very important from a practical point of view since it guarantees
that the execution traces are strongly related, and in turn, this enables a user
to debug a logic program in \Fo{} by looking at its execution trace in
\Ho{}.

We also claim that \hrun handles terms outside \llambda in the following sense:

\begin{proposition}[Fidelity recovery]\label{prop:fidelity-recovery} 
In the context of \frun and \hrun, if 
$\rho_{p-1} \foUnifPb_{p} \subseteq \llambda$ 
(even if $\;\foUnifPb_{p} \not\subseteq \llambda$)
then
$$
\stepF{\foUnifPb}{p}{\rho_{p-1}}{\rho_p} \Leftrightarrow
\stepH{\hoUnifPb}{p}{\sigma_{p-1}}{\linkStore_{p-1}}{\sigma_{p}}{\linkStore_p}
$$
\end{proposition}

\noindent
In other words, if the two terms involved in a step
re-enter \llambda, then \hstep and \fstep are again related, even if
$\foUnifPb \not\subseteq \llambda$ and hence \cref{prop:fidelity} does not apply.
Indeed, the main difference between \cref{prop:fidelity} and \cref{prop:fidelity-recovery}
is that the assumption of the former is purely static, it can be checked upfront. 
When this assumption is not satisfied, one can still simulate a logic program and
have guarantees of fidelity if, at run time, decidability of higher-order
unification is restored.

This property has practical relevance since in many logic programming
implementations, including Elpi, the order in which unification problems
are tackled does matter.
The simplest example is the sequence $F \UnifRel \lambda x.a$ and
$F \appsep a \UnifRel a$: the second problem is not in \llambda and has two
unifiers, namely $\sigma_1 = \{~ F \mapsto \lambda x.x ~\}$ and
$\sigma_2 = \{~ F \mapsto \lambda x.a ~\}$. The first problem picks $\sigma_2$,
making the second problem re-enter \llambda.

\paragraph{Backtracking} We omit it from our model of logic-program execution
since it plays a very minor role, orthogonal to higher-order unification.
We point out that each \emph{run} corresponds to a (proof search) branch in the
logic program that either fails at some point, or succeeds. A computation that
succeeds by backtracking, exploring multiple branches, could be
modeled as a set of runs with a (possibly non-empty) common prefix.

\subsection{The solution (in a nutshell)}
\label{sec:nutshell}
A term $s$ is compiled to a term $t$ where every
``problematic'' subterm $e$ is replaced by a fresh unification variable $h$
with an accessory \emph{link} that represents a suspended unification problem
$h \Ue e$. As a result, \Ue is ``well behaved'' on $t$, in the sense that
it does not contradict \Eo as it would otherwise do on the
``problematic'' sub-terms.\\
We now define ``problematic'' and ``well behaved'' more formally.
We use the $\Diamond$ symbol since it stands for ``possibly'' in modal logic
and all problematic terms are characterized by some uncertainty.

\begin{definition}[\maybebeta]\label{def:maybebeta}
  \maybebeta is the set of terms of the form $F \appsep x_1 \ldots x_n$
  such that $x_1 \ldots x_n$ are distinct names (of bound variables).
\end{definition}

\noindent
An example of a \maybebeta{} term is the application $F \appsep x$.
This term is problematic since the application node of 
its syntax tree cannot be used to justify a
unification failure, i.e. by properly instantiating $F$, the term
head constructor may become a $\lambda$, or a constant, or a name, or remain an application.

\begin{definition}[\maybeeta]\label{def:maybeeta}
  \maybeeta is the set of terms $\lambda x.s$ such that $\exists \rho, \rho (\lambda x.s)$
  is an $\eta$-redex. % without the lambda, 5.11 is not true, that
  % is we assume variables are in W
\end{definition}

\noindent
An example of a term $s$ in \maybeeta{} is
$\lambda x.\lambda y.F \appsep y \appsep x$
since the substitution
$\rho = \{ F \mapsto \lambda x.\lambda y.f \appsep y \appsep x\}$
makes $\rho s \Eo{} \lambda x.\lambda y.f \appsep x \appsep y$,
which is the eta-long form of $f$. This term is problematic since
its leading $\lambda$ abstraction cannot justify a
unification failure against a constant $f$.

\begin{definition}[\notllambda]\label{def:notllambda}
  \notllambda is the set of terms of the form $F \appsep t_1 \ldots t_n$
  such that $t_1 \ldots t_n$ are not distinct names.
\end{definition}

\noindent
These terms are problematic for the very same reason terms in \maybebeta are,
but they cannot be handled directly by the unification of the meta language, which
is only required to handle terms in \llambda. Still, given $s \in \notllambda$
there may exist a substitution $\rho$ such that $\rho s \in \llambda$.

% An example of $t$ in \notllambda{} is $F \appsep a$ for a constant $a$.
% Note however tha
% an oracle could provide an assignment $\rho = \{ F \mapsto \lambda x.x\}$
% that makes the resulting term fall back in \llambda.
% \todo{said before}

\newcommand{\subterm}[1]{\ensuremath{\mathcal{P}(#1)}}
We write $\subterm{t}$ for the set of sub-terms of $t$, and
%\begin{definition}[Subterms \subterm{t}] The set of sub terms of $t$ is the
%  largest set $\subterm{t}$ that can be obtained by the following rules.
%$$
%\begin{array}{l}
%t \in \subterm{t}\\
%t = f t_1\ldots t_n \Rightarrow \subterm{t_i} \subseteq \subterm{t} \land f \in \subterm{t}\\
%t = \lambda x.t' \Rightarrow \subterm{t'} \subseteq \subterm{t}\\
%\end{array}
%$$
%\end{definition}
%\noindent
we write $\subterm{X} = \bigcup_{t\in X} \subterm{t}$ when $X$ is a set of terms.

\newcommand{\wellb}{\ensuremath{\mathcal{W}}\xspace}
\begin{definition}[Well behaved set]
Given a set of terms $X$, % \subseteq \Ho{}$,
$$
\wellb(X) \Leftrightarrow \forall t \in \subterm{X},
t \not\in (\maybebeta{} ~\cup~ \maybeeta{} ~\cup~ \notllambda{})
$$
\end{definition}

\noindent
We write $\wellb(t)$ as a short for $\wellb(\{t\})$.
We claim our compiler validates the following properties: 

\begin{proposition}[\wellb-enforcing]\label{prop:w-enforcing}
  Given a term $s$ in \Fo{}, 
  $$
    \C{s}{t}{m}{l} \Rightarrow \wellb(t)
  $$
\end{proposition}

\begin{proposition}[compiler-useful]\label{prop:w-noncontra}
  Given two terms $s_1$ and $s_2$, if $\;\exists \rho, \rho s_1 \Eo \rho s_2$,
  then 
  $$
    \C{s_i}{t_i}{m_i}{l_i} ~\mathrm{for}~ i \in \{ 1,2 \} \Rightarrow
    t_1 \Ue t_2 \mapsto \sigma \label{prop:compilation-w}
  $$
\end{proposition}

\noindent
In other words the compiler outputs terms in \wellb even if its
input is not. And if two terms can be unified in the source language
then their images can also be unified in the target language. 
Note that the property holds for any substitution, not necessarily a most
general one: when applied to terms in \wellb
the meta-language unification 
\Ue{} agrees with the object-language equality \Eo. Note that a compiler that
translate all terms to unification variables would satisfy this
property, but wound certainly fail to verify \cref{prop:fidelity}.

\begin{proposition}[\wellb{}-preservation]\label{prop:w-preservation}
$\forall \hoUnifPb, \forall \linkStore, \forall p, \forall \sigma, \forall \sigma'$
$$
\begin{array}{l}
\wellb(\sigma\hoUnifPb) \land
  \sigma\hoUnifPb_{p_l} \Ue \sigma\hoUnifPb_{p_r} \mapsto {\sigma'}
  \Rightarrow \wellb(\sigma' \hoUnifPb)\\
\wellb(\sigma\hoUnifPb) \land
  \progress(\linkStore,\sigma) \mapsto (\_,\sigma')
  \Rightarrow \wellb(\sigma' \hoUnifPb)
\end{array}
$$
\end{proposition}

\noindent
Proposition \ref{prop:w-preservation} is key to proving \cref{prop:simulation,prop:fidelity}.
Informally, it says that the problematic terms moved on the side by the compiler
are not reintroduced by \hstep, hence \Ue{} can continue to operate properly.
In \cref{sec:llam,sec:eta,sec:beta}
we describe how to %the compiler
recognize terms in \maybebeta, \maybeeta and
\notllambda and how \progress takes care of links and how it preserves \wellb
and ensures \cref{prop:simulation,prop:fidelity,prop:fidelity-recovery}.

We prove \fullref{prop:fidelity} in
\cref{sec:eta,sec:beta,sec:invariant1}
and \fullref{prop:simulation} only in \cref{sec:llam},
where links are not present yet.
In order to prove \cref{prop:simulation} in the presence of links
one needs to refine \cref{prop:fidelity} by relating the two
substitutions $\rho_p$ and $\sigma_p$: they are equivalent only by using decompilation to 
take into account terms in \linkStore.
Then, by induction
on the number of steps, \cref{prop:fidelity} implies \cref{prop:simulation}.
We discuss \fullref{prop:fidelity-recovery} in \cref{sec:beta}.

% Note that proposition \ref{prop:w-preservation} does not hold for \hrun as a whole
% since decompilation can introduce (actually restore) problematic terms.

\section{Basic compilation and simulation}
\label{sec:simulation}


\subsection{Memory map (\mapStore) and substitution (\texorpdfstring{$\rho$ and $\sigma$}{rho and sigma})}
\label{sec:grounwork}

% The predicate to test this condition is called \elpiIn{pattern-fragment}:

% \begin{elpicode}
type pattern-fragment list A -> o.
\end{elpicode}


Unification variables are identified by a (unique) memory address.
The memory and its associated operations are described below:

\begin{elpicode}
  typeabbrev (mem A) (list (option A)).
  type set? addr -> mem A -> A -> o.
  type unset? addr -> mem A -> o.
  type assign addr -> mem A -> A -> mem A -> o.
  type new mem A -> addr -> mem A -> o.
\end{elpicode}


\noindent
If a memory cell is \elpiIn{none}, then the corresponding unification variable
is not set. \elpiIn{assign} sets an unset cell to the given value, while
\elpiIn{new} finds the first unused address and sets it to \elpiIn{none}.

Since each \Ho unification variable occurs together with a scope,
its assignment needs to be abstracted over it to enable the
instantiation of the same assignment to different scopes.
This is expressed by the \elpiIn{inctx} container, and in particular
its \elpiIn{abs} binding constructor.

\begin{elpicode}
  kind inctx type -> type.                              ~($\customlabel{data:inctx}{\cdot\vdash\cdot}$)~
  type abs (tm -> inctx A) -> inctx A.
  type val A -> inctx A.
  typeabbrev assignment (inctx tm).
  typeabbrev subst (mem assignment).
\end{elpicode}


\noindent
A solution to a \Fo{} variable is, instead, a plain term, that is, \elpiIn{fsubst}
is an abbreviation for \elpiIn{mem to}.
%We call \elpiIn{fsubst} the memory of \Fo{}, while we call \elpiIn{subst}
%the one of \Ho.
%
The compiler establishes a mapping between variables of the two languages.

\begin{elpicode}
  kind ovariable type.
  type ov addr -> ovariable.
  kind mvariable type.
  type mv addr -> arity -> mvariable.
  kind mapping type.
  ~\PYG{k+kd}{type} \PYG{n+nf}{(<->)} \PYG{k+kt}{ovariable -> mvariable -> mapping}~.
  typeabbrev mmap (list mapping).
\end{elpicode}


\noindent
Each \elpiIn{mvariable} is stored in the mapping together with
its arity (a number) so that the code of \ref{clause:malloc} below can preserve
the following property:

\begin{invariant}[Unification-variable arity]\label{inv:uvaarity}
  Each variable \elpiIn{A}
  in \Ho has a (unique) arity \elpiIn{N} and each occurrence
  \elpiIn{(uva A L)} is such that \elpiIn{L} has length \elpiIn{N}.
\end{invariant}

\begin{elpicode}
  type m-alloc ovariable -> mvariable -> mmap -> mmap -> 
    subst -> subst -> o.                            ~\customlabel{clause:malloc}{(malloc)}~
  m-alloc Ov Mv M M S S :- mem M (Ov <-> Mv), !.
  m-alloc Ov Mv M [Ov <-> Mv|M] S S1 :- Mv = mv N _, new S N S1.
\end{elpicode}


\noindent
When a single \elpiIn{ovariable} occurs multiple times with different numbers
of arguments, the compiler generates multiple mappings for it, on a first
approximation, and then ensures the mapping is bijective by introducing
\linketa; this detail is discussed in section \ref{sec:invariant1}.

%Applying the substitution corresponds to dereferencing a term with respect to
%the memory.
%  To ease the comparison we split \Fo{} dereferencing into a
% \elpiIn{fder} step and a \elpiIn{napp} one. The former step replaces references
% to memory cells that are set with their values, and has a corresponding
% operation in \Ho, namely \elpiIn{deref}. On the contrary \elpiIn{napp}
% has no corresponding operation in \Ho, and only ensures that
% terms of the form <<\elpiIn{o-app[o-app L1|L2]}>> are replaced by
% <<\elpiIn{o-app L3}>> where \elpiIn{L3} is the concatenation of \elpiIn{L1}
% and \elpiIn{L2}. The reasons for this asymmetry is
% that an \elpiIn{o-app} node with a flexible head is always mapped
% to a \elpiIn{uva} (as per \cref{sec:simulation,sec:beta}),
% preventing nested applications to materialize.
% \begin{elpicode}
  type fder fsubst -> to -> to -> o.
  fder _ (o-con C) (o-con C).
  fder S (o-app A) (o-app B) :- map (fder S) A B.
  fder S (o-lam F) (o-lam G) :-
    pi x\ fder S x x => fder S (F x) (G x).
  fder S (o-uva N) R :- set? N S T, fder S T R.
  fder S (o-uva N) (o-uva N) :- unset? N S.

  type fderef fsubst -> to -> to -> o.                 ~($\customlabel{pred:fder}{\rho s}$)~
  fderef S T T2 :- fder S T T1, napp T1 T2.
\end{elpicode}

It is worth examining the code of \elpiIn{deref}, that
applies the substitution to a \Ho{} term. Notice how assignments are moved
to the current scope, i.e. the \elpiIn{abs}-bound variables are renamed
with the names in the scope of the unification variable occurrence.

\begin{elpicode}
  type deref subst -> tm -> tm -> o.                   ~($\customlabel{pred:deref}{\sigma t}$)~
  deref _ (con C) (con C).
  deref S (app A) (app B) :- map (deref S) A B.
  deref S (lam F) (lam G) :-
    pi x\ deref S x x => deref S (F x) (G x).
  deref S (uva N L) R :- set? N S A,
    move A L T, deref S T R.
  deref S (uva N A) (uva N B) :- unset? N S,
    map (deref S) A B.
\end{elpicode}


\noindent
Note that \elpiIn{move} strongly relies on invariant \ref{inv:uvaarity}: the length
of the arguments of all occurrences of a unification variable is the same. Hence,
they have the same simple type for the meta-level, and therefore the number of
\elpiIn{abs} nodes in the assignment matches that length.
This guarantees that \elpiIn{move} never fails.

\begin{elpicode}
  type move assignment -> list tm -> tm -> o.
  move (abs Bo) [H|L] R :- move (Bo H) L R.
  move (val A)  [] A :- !.

\end{elpicode}


We write $\sigma = \{~ A_{xy} \mapsto y ~\}$ for the assignment
<<\elpiIn{abs x\abs y\y}>> and $\sigma = \{~ A \mapsto \lambda x.\lambda y.y ~\}$
for <<\elpiIn{lam x\lam y\y}>>.

\subsection{Links (\linkStore)}

\noindent
As mentioned in section~\ref{sec:nutshell}, the compiler
replaces terms in \maybeeta, \maybebeta, and \notllambda with fresh
variables linked to the problematic terms. Terms in \maybebeta do not
need a link since \Ho{} variables faithfully represent
the problematic term thanks to their scope.

\begin{elpicode}
  kind baselink type.
  type link-eta  tm -> tm -> baselink.
  type link-llam tm -> tm -> baselink.
  typeabbrev links (list (inctx baselink)).
\end{elpicode}


\noindent
The right-hand side of a link, the problematic term, can occur under binders.
To accommodate this situation, the compiler wraps \elpiIn{baselink} using
the \elpiIn{inctx} container (see~\ref{data:inctx} also used for \elpiIn{subst}).

\begin{invariant}[Link left hand side]\label{inv:linklhs}
  The left-hand side of a suspended link
  is a variable.
\end{invariant}

\noindent
New links are suspended by construction.
If the left-hand side is assigned during a step, then 
the link is considered for progress and possibly eliminated.
This is discussed in \cref{sec:eta} and \cref{sec:beta}.

In the examples we represent links as equations
%between two terms
under a context.
The equality sign is subscripted with the
kind of \elpiIn{baselink}. For example $\linkbetaM{x}{A_x}{F_x~a}$ corresponds to:
\begin{elpicode}
abs x\ val (link-llam (uva A [x]) (app[uva F [x],con "a"]))
\end{elpicode}

% The first three rules unify terms with same rigid heads, and
% call the unification relation on the sub-terms. If $t_1$ (resp. $t_2$) is an
% assigned variables, $t_1$ is dereferenced to $t_1'$ (resp. $t_2'$) and the
% unification is called between $t_1'$ and $t_2$ (resp. $t_1$ and $t_2'$). If both
% terms are unification variables, we test that their arguments are in the pattern
% fragment, we allocate a new variable $w$ in $\rho_1$ such that $w$ is the
% pruning of the arguments of $t_1$ and $t_2$, we assign both $t_1$ and $t_2$ to
% $w$ and return the new mapping $\rho_2$ containing all the new variable
% assignment. Finally, if only one of the two terms is an unification variable
% $v$, after having verified that $v$ does not occur in the other term $t$, we
% bind $v$ to $t$ and return the new substitution mapping.

% \old

% A key property needed in unification is being able to verify if two terms are
% equal wrt a given equational theory. This relation allow to compare terms under
% a certain substitution mapping, so that any time a variable $v$ is assigned in a
% subterm, a dereferencing of $v$ is performed. After variable dereferencing, the
% test for equality is continued on the new-created subterm.

% The base equality function over terms can be defined as follows:

% The solution we are proposing aim to overcome these unification issues by 1)
% compiling the terms $t$ and $u$ of the OL into an internal version $t'$ and $u'$
% in the ML; 2) unifying $t'$ and $u'$ at the meta level instantiating meta
% variables; 3) decompiling the meta variable into terms of the OL; 4) assigning
% the variables of the OL with the decompiled version of their corresponding meta
% variables. We claim that $t$ and $u$ unify if and only if $t'$ and $u'$ unify
% and that the substitution in the object language is the same as the one returned
% by the ML. \noteE[inline]{same or $\supseteq$ or $\subseteq$}

% In the following section we explain how we deal with term (de)compilation and
% links between unification variables.

% \section[Compilation: fo\_tm to tm]{Basic simulation of \Fo{} in \Ho{}}

% In this section we describe a basic compilation scheme that we refine
% later, in the following sections. This scheme is sufficient to implement
% a \hstep that respects $\beta$-conversion for terms in \llambda.
% The extension to $\beta\eta$-conversion is described in \cref{sec:eta} and
% the support for terms outside \llambda in \cref{sec:beta}.

\subsection{Compilation}
\label{sec:compilation}

The simple compiler described in this section serves as a base for the
extensions in \cref{sec:llam,sec:eta,sec:beta}.
Its main task is to beta normalize the term and map one syntax tree to
the other.
In order to bring back the substitution from \Ho{} to \Fo{} the compiler
builds a ``memory map'' connecting the \Fo{} variables to the \Ho ones.
% using the routine \ref{clause:malloc}.

The signature of the \elpiIn{comp} predicate below allows for the generation of
links (suspended unification problems), which plays no role in this section
but plays a major role in \cref{sec:llam,sec:eta,sec:beta}.

\begin{elpicode}
  type comp to -> tm -> mmap -> mmap -> links -> links ->
    subst -> subst -> o.
  comp (o-con C) (con C) M M L L S S.
  comp (o-lam F) (lam F1) M1 M2 L1 L2 S1 S2 :-            ~\customlabel{rule:complam}{(\ensuremath{c_\lambda})}~
    comp-lam F F1 M1 M2 L1 L2 S1 S2.
  comp (o-uva A) (uva B []) M1 M2 L L S1 S2 :- 
    m-alloc (ov A) (mv B (arity z)) M1 M2 S1 S2.
  comp (o-app A) (app A1) M1 M2 L1 L2 S1 S2 :-            ~\customlabel{rule:compapp}{(\ensuremath{c_@})}~
    fold6 comp A A1 M1 M2 L1 L2 S1 S2.
\end{elpicode}

\begin{elpicode}
type compile to -> tm -> mmap -> mmap ->  links -> links -> 
  subst -> subst -> o.
compile F G M1 M2 L1 L2 S1 S2 :-
  beta-normal F F', comp F' G M1 M2 L1 L2 S1 S2.
\end{elpicode}


\noindent
With respect to \cref{sec:problem-statement}, the signature also allows
for updates to the substitution.
The code above uses that possibility
in order to allocate space for the variables, i.e. it sets their memory
address to \elpiIn{none} (a detail not worth mentioning in the
previous sections).

\begin{elpicode}
  type comp-lam (to -> to) -> (tm -> tm) -> 
    mmap -> mmap -> links -> links -> subst -> subst -> o.
  comp-lam F G M1 M2 L1 L3 S1 S2 :-
    pi x y\ (pi M L S\ comp x y M M L L S S) =>         ~\customlabel{rule:complamh}{(\ensuremath{H_\lambda})}~
      comp (F x) (G y) M1 M2 L1 (L2 y) S1 S2,
    close-links L2 L3.
\end{elpicode}


\noindent
In the code above, the syntax \elpiIn{pi x y\..} is syntactic sugar for
iterated \elpiIn{pi} abstraction, as in \elpiIn{pi x\ pi y\..}.
The auxiliary predicate \elpiIn{fold6} folds a predicate
with three accumulators (the memory map, the links and the substitution)
over a lists to obtain a new list and the final values of the
accumulators.

The auxiliary function \elpiIn{close-links} tests if the bound variable
\elpiIn{v} really occurs in the link. If it does, the link is wrapped into
an additional \elpiIn{abs} node binding \elpiIn{v}. In this way links generated
deep inside the compiled terms can be moved outside their original context
of binders.

\begin{elpicode}
  type close-links (tm -> links) -> links -> o.
  close-links (v\[X v|L v]) [abs X|R] :- close-links L R.
  close-links (_\[]) [].
\end{elpicode}


% \noindent
% \todo{remove pruning optimization from code above}

\subsection{Execution}
\label{sec:execution}

A step in \Ho consists in unifying two terms and reconsidering all
links for progress. If either of these tasks fails, we consider the entire step to
fail. It is at this granularity that we can relate steps in the
two languages.


\begin{elpicode}
type hstep tm -> tm -> links -> links -> subst -> subst -> o.
hstep T1 T2 L1 L2 S1 S3 :-
  (T1 ~$\Ue$~ T2) S1 S2,
  progress L1 L2 S2 S3.
\end{elpicode}


\noindent
Note that the infix notation \elpiIn{((A ~\Ue~B) C D)} is syntactic sugar for
\elpiIn{((~\Ue\!\!\!~) A B C D)}.

Reconsidering links is a fixpoint process because the progress of a link
can update the substitution, which may then enable another link to progress.

\begin{elpicode}
  type progress links -> links -> subst -> subst -> o.
  progress L L3 S S3 :-
    progress1 L L1 S S1,
    occur-check-links L1,
    scope-check L1 L2 S1 S2
    if (L = L2, S = S2) (L3 = L2, S3 = S2)
       (progress L2 L3 S2 S3).
\end{elpicode}


\subsubsection{Progress}
In the base compilation scheme, \elpiIn{progress1} is the identity function
on both the links and the substitution, so the fixpoint trivially terminates.
Sections \ref{sec:eta} and \ref{sec:beta} add rules to \elpiIn{progress1}
and explain why the do not hinder termination.
% For brevity we omit the code
% that applies the substitution \elpiIn{S1} to all terms in \linkStore.

\subsubsection{Scope check}\label{sec:sc}
The predicate \elpiIn{scope-check} 
replaces each
link of the form \linketaM{\Gamma}{X_{x_1\ldots x_n}}{t} with
a link
\linketaM{{x_1\ldots x_n}}{X_{x_1\ldots x_n}}{t'} whenever $\{x_1\ldots x_n\} \subset \Gamma$.
The term $t' = \lambda x.Y_{{x_1\ldots x_n}~x}$ is obtained after
executing $\lambda x.Y_{{x_1\ldots x_n}~x} \Ue t$ using a fresh $Y$.
This unification
ensures that $t'$ cannot contain variables in $\Gamma / \{x_1\ldots x_n\}$,
and if it fails then \progress hence \hstep fails.
In turn this grants fidelity in the case where the \lhs of an \linketa
is pruned of a name that occurs (rigidly) in $t$:
links represent suspended unification problems
\emph{that may succeed}.

\subsubsection{Occur check}\label{sec:oc}
Since compilation moves problematic terms out of the sight of \Ue{},
that procedure can only perform a partial occur check. For example, the
unification problem $X \Ue f~Y$ cannot generate a cyclic substitution alone,
but should be disallowed if $\linkStore$ contains a link like
$\linketaM{}{Y}{\lambda z.X_z}$: we don't know yet if $Y$ will feature
a lambda in head position, but we surely know it contains $X$.
The procedure \elpiIn{occur-check-links} is in charge of
performing this check that is needed in order to
guarantee \fullref{prop:fidelity}.


\subsection{Substitution decompilation}\label{sec:decompilation}

Decompiling the substitution involves three steps.

First and foremost, problematic terms stored in
\linkStore have to be moved back into the game:
a suspended link must be turned into a valid assignment.
This operation is possible thanks to \fullref{inv:linklhs},
and that no link causes an occur-check~(\ref{sec:oc})
and the fact that \linkStore is duplicate-free (see \cref{sec:invariant1}).

The second step allocates
in the memory of \Fo{}
new variables used to implement the pruning of higher-order variables. 
For example,
$F\appsep x\appsep y = F\appsep x\appsep z$ requires allocating a 
variable $G$ in order to express the assignment $F_{ab} \mapsto G_a$.

The final step is to decompile each assignment. The \elpiIn{val}
node carries a term that is easy to decompile since \mapStore is a bijection.
Since the \elpiIn{o-lam} node
carries no additional information (other than the function body),
each \elpiIn{abs} node can be decompiled to a \elpiIn{o-lam} one.

However, if the \elpiIn{o-lam} node was carrying more information,
as is the case for Coq where it holds the type of the bound variable,
one would need to store this piece of information in the memory map,
were we store the arity (that is a very simple function type).

\begin{proposition}[Compilation round trip]\label{prop:comprt}
  % \elpiIn{compile F T [] M [] L [] S} then 
  % if \elpiIn{T} $\in$ S
  If $\C{s}{t}{m}{l}$ and $l \in \linkStore$ and $m \in \mapStore$
  and
  $\sigma = \{ A \mapsto t\}$ and $\mapping{X}{A}{n} \in \mapStore$
  then $\D{\sigma}{\mapStore}{\linkStore}{\rho}$ and
  $\rho X \Eo \rho s$.
\end{proposition}

% \noindent
% The statement above roughly states that the translation from
% \Fo{} to \Ho is not lossy, it can be reversed. The statement
% is made convoluted by the fact that we hid the procedure to
% decompile a term in favor of the one to decompile substitutions.
We omit to sketch the proof of this property for brevity.

% \noteD{Should also decompile links\dots}
% \begin{proof}[Proof sketch]
% Trivial since \mapStore is a bijection and 
% the terms are beta normal.
% some discussion about commit maybellam to be done later.
% \end{proof}

\subsection{Definition of \Uo{} and its properties}\label{sec:founif}

We already have all the ingredients to show the code of \Uo{}.

\begin{elpicode}
~\PYG{k+kd}{type} \PYG{n+nf}{(\Uo)} \PYG{k+kt}{to -> to -> fsubst -> o}~.
(A ~$\Uo$~ B) F :-
  compile A A' [] M1 [] L1 [] S1,
  compile B B' M1 M2 L1 L2 S1 S2,
  hstep A' B' L2 L3 S2 S3,
  decompile M2 L3 S3 [] F.
\end{elpicode}


\noindent
So far the compiler is very basic. It does not really
enforce that the terms passed to \hstep are in \wellb,
and indeed makes no use of the higher-order capabilities
of the meta language (all generated
variables have an empty scope).
Still, we can prove that \Uo is a good
``first-order'' unification algorithm if the input
already happens to be in \wellb.
Later, when the compiler will ensure \fullref{prop:w-enforcing},
the proof will be refined to cover for the new cases.

\begin{theorem}[Properties of \Uo in \wellb]\label{prop:uniffow}
  The implementation of \Uo above is a good unification
  for \Eo
  (as per \cref{prop:good-unif})
  in the domain \wellb.
\end{theorem}
\begin{proof}[Proof sketch]
In this setting, \Ee is as strong as \Eo on ground terms.
What we have to show is that whenever two different \Fo{}
terms can be made equal by a substitution $\rho$,
we can produce this $\rho$ by finding
a $\sigma$ via \Ue{} on the corresponding \Ho terms and by decompiling it.
If we look at the syntax of \Fo{}, the only interesting case is
<<\elpiIn{o-uva X ~\Uo~s}>>. In this case, after compilation, we have
<<\elpiIn{uva Y [] ~\Ue~t}>>
that succeeds with $\sigma = \{ Y \mapsto t\}$ and
$\sigma$ is decompiled to $\rho = \{ X \mapsto s\}$ by \cref{prop:comprt}.
\end{proof}

\begin{theorem}[Fidelity in \wellb] \Fullref{prop:simulation} and
\fullref{prop:fidelity} hold if $\;\wellb{(\foUnifPb)}$. \label{thm:fidfow}
\end{theorem}
\begin{proof}[Proof sketch]
Since \elpiIn{progress1} is a no-op then \fstep and \hstep are the same.
By \cref{prop:uniffow}, \Ue is equivalent to \Uo in \wellb.
\end{proof}

\subsection{Notational conventions}

In the following sections we use the following notation for input and output
of the compiler. \foUnifPb represents the input unification problems in \Fo{}
while \hoUnifPb their corresponding compiled version with memory mapping
\mapStore and links \linkStore. For example:
%
\printAlll
  {{{p_1,p_2},{p_3,p_4}}}
  {{{t_1,t_2},{t_3,t_4}}}
  {{{X_1,A_1,{{n}}},{X_2,A_2,{{m}}}}}
  {{{\eta,\Gamma,a,b}}}

  \noindent
We index each sub-problem, sub-mapping, and sub-link with its position
starting from $1$ and counting from left to right, top to bottom.
For example, $\hoUnifPb_2$ corresponds to the \Ho problem $t_3 \Ue t_4$.

As we deal with each category of problematic terms we
weaken the assumptions of~\cref{thm:fidfow} using the following
definitions:

\begin{definition}[$\wellb_\beta$]
  $
  \wellb_\beta(X) \Leftrightarrow \forall t \in \subterm{X}, t \not\in (\maybeeta{} ~\cup~ \notllambda{})
  $
  \end{definition}
  
\begin{definition}[$\wellb_{\beta \eta}$]
  $
  \wellb_{\beta \eta}(X) \Leftrightarrow \forall t \in \subterm{X}, t \not\in \notllambda{}
  $
\end{definition}
  
  


%The compiled version of each $\foUnifPb_i$ is represented by $\hoUnifPb_i$.

% \subsection{Limitations of by this basic scheme}
% \label{sec:basic-comp-limitations}
% The basic compilation scheme is not about to
% deal wit the following problem:
% \printAlll
%   {{{\lambda x y.X \appsep y \appsep x, \lambda x y.x},{\lambda x. f \appsep (X \appsep x) \appsep x,Y}}}
%   {{}}
%   {{}}
%   {{}}
% % \begin{gather}
% % \lambda x y.F \appsep y \appsep x = \lambda x y.x \label{eq:unif-eta1}\\
% % \lambda x. f \appsep (F \appsep x) \appsep x = G \label{eq:unif-eta2}
% % % \lambda x. f \appsep (F \appsep x) \appsep x = f \appsep (\lambda y.y) \label{eq:unif-eta2}
% % \end{gather}

% \noindent Note that here $X$ is used with different arities, moreover
% in the second problem the left hand side happens to be an
% eta expansion (of $f (\lambda y.y)$) only after we discover (at run time)
% that $X = \lambda x\lambda y.y$ (i.e. that $X$ discards the $x$ argument).
% Both problems are addressed in the next two sections.
  
\section{Handling of \maybebeta}\label{sec:llam}

% In order to make \Uo higher-order, we need to
% take care of terms in \maybebeta.
% In the example below, we can see that
The basic compiler given in 
the previous section is unable to make the following
higher-order unification problem succeed.
%
% make test ONLY=7006 TEX=tex
\printAlllSingle
  {{{\lambda x.(f\appsep (X\appsep x)\appsep a),\lambda x.(f\appsep x\appsep a)}}}
  {{{\lambda x.(f\appsep (A\appsep x)\appsep a),\lambda x.(f\appsep x\appsep a)}}}
  {{{X,A,0}}}
  {{}}
The unification problem $\hoUnifPb_1$ fails while trying to unify
$A \appsep x$ and $x$, which is equivalent to <<\elpiIn{app [uva A [], x]}>>
versus \elpiIn{x}.
In order to exploit the higher-order unification algorithm of the meta language,
we %need to 
compile the \Fo{} term $X\appsep x$ into the \Ho term $A_x$.
%,which is \elpiIn{uva A [x]}.

\subsection{Compilation and decompilation}

We add the following rule before rule~\ref{rule:compapp}, where
\elpiIn{pattern-fragment} is a predicate checking if a list of
terms is a list of distinct names.

\begin{elpicode}
  comp (o-app [o-uva A|Ag]) (uva B Ag1) M1 M2 L L S1 S2 :-
    pattern-fragment Ag, !,
      fold6 comp Ag Ag1 M1 M1 L L S1 S1,
      len Ag Arity, 
      m-alloc (ov A) (mv B (arity Arity)) M1 M2 S1 S2.
\end{elpicode}


\noindent
Note that compiling \elpiIn{Ag} cannot create new mappings nor links, since \elpiIn{Ag}
is made of bound variables, and the hypothetical rule 
\ref{rule:complamh}
loaded by \elpiIn{comp-lam}
grants this property.

% The only detail worth discussing is the fact that the procedure updates a
% substitution, rather than just crafting one as presented in
% section~\ref{sec:problem-statement}. The reason is that the algorithm folds
% over a term, updating a substitution while it traverses it.\noteE[inline]{explain better}

\paragraph{Decompilation}
No change
to \elpiIn{commit-link}
since no link is added.
% by the compilation of 
%\maybebeta terms, no modification to the \elpiIn{commit-link}
%is needed.

\paragraph{Progress}
No change to \elpiIn{progress} since no link is added.
% Similarly to decompilation, since no link is produced,
% no modification to the \elpiIn{progress} predicate is needed.

\begin{lemma}[Properties of \Uo in $\wellb_\beta$]\label{prop:uniffowb}
  Given the updated compilation scheme, the \Uo of~\cref{sec:founif}
  is a good unification (as per \cref{prop:good-unif})
  for \Eo in the domain $\wellb_\beta$.
\end{lemma}

  \begin{proof}[Proof sketch]
  If we look at the \Fo{} terms, there is one more interesting case,
  namely \elpiIn{o-app[o-uva X|W] ~\Uo~s} when \elpiIn{W}
  are distinct names compiled to $\vec{w}$. In this case
  the \Ho problem is $Y_{\vec{w}} \Ue t$ that succeeds with
   $\sigma = \{ Y_{\vec{y}} \mapsto t[\vec{w}/\vec{y}]\}$, which in turn
   is decompiled to $\rho = \{ Y \mapsto \lambda \vec{y}.s[\vec{w}/\vec{y}]\}$.
   Thanks to \ref{clause:beta1} we have
   $(\lambda \vec{y}.s[\vec{w}/\vec{y}])~\vec{w} \Eo s$.
  \end{proof}
  
\begin{lemma}[$\wellb$-partial-enforcement]\label{lem:w-enforcement-maybebeta}
  $\forall s, \C{s}{t}{m}{l}$, if $\:\wellb_\beta(s)$  then $\:\wellb(t)$.
\end{lemma}
% \begin{proof}[Proof sketch]
%   problematic terms are mapped to uva by comp,
%   the problematic o-app node is gone.
%   if F x = t then  F = lam x.t[]
%   \todo{finish/maybe W enforcement is not needed here}
% \end{proof}
\noindent
In other words the compiler eliminates all subterms that are in \maybebeta.

\begin{theorem}[Fidelity in $\wellb_\beta$] \Fullref{prop:simulation} and
\fullref{prop:fidelity} hold if $\;\wellb_\beta(\foUnifPb)$.
\label{th:fidelity-beta}
\end{theorem}
\begin{proof}[Proof sketch] Thanks to \cref{lem:w-enforcement-maybebeta},
  \Ue is as powerful as \Uo in $\wellb_\beta$, as well as in \wellb by
  \cref{thm:fidfow}.
\end{proof}

% \subsection{Example}

% OK

% \begin{elpicode}
% Terms [ o-lam x\ o-app[o-con"g",o-app[o-uva z, x]]
%       , o-lam x\ o-app[o-con"g", o-con"a"] ]
% Problems = [ pr z (s z) ] % $\lambda x.g (F x) = \lambda x.g a$
% lam x\ app[con"g",uva z [x]] ~\Uo~lam x\ app[con"g", con"a"]
% link z z (s z)
% HS = [some (abs x\con"a")]
% S = [some (o-lam x\o-con a)]
% \end{elpicode}

% KO

% \begin{elpicode}
%   Terms [ o-lam x\ o-app[o-con"g",o-app[o-uva z, x]]
%   , o-lam x\ o-app[o-con"g", o-con"a"] ]
% Problems = [ pr 0 1   % $A = \lambda x.x$
%            , pr 2 3 ] % $A a = a$
% lam x\ app[con"g",uva z [x]] ~\Uo~lam x\ app[con"g", con"a"]
% link z z (s z)
% HS = [some (abs x\con"a")]
% S = [some (o-lam x\o-con a)]
% lam x\ app[f, app[X, x]] = Y,
%   lam x\ x) = X.
% \end{elpicode}

% \otext{Goal: $s_1 \Uo s_2$ is compiled into $t_1 \Ue t_2$}
% \otext{What is done: uvars \elpiIn{fo_uv} of OL are replaced into uvars \elpiIn{ho_uv} of the ML}
% \otext{Each \elpiIn{fo_uv} is linked to an \elpiIn{ho_uv} of the OL}
% \otext{Example needing the compiler v0 (tra l'altro lo scope è ignorato):\\ \elpiIn{lam x\ app[con"g",app[uv 0, x]] ~\Uo~lam x\ app[con"g", c"a"]}}
% \otext{Links used to instantiate vars of elpi}
% \otext{After all links, the solution in links are compacted and given to coq}
% \otext{It is not so simple, see next sections (multi-vars, eta, beta)}


% The compilation step is meant to recover the higher-order variables of the OL,
% expressed in a first order way, by replacing them with higher-order variables in
% the ML. In particular, every time a variable of the OL is encountered in the
% original term, it is replaced with a meta variable, and if the OL variable is
% applied to a list of distinct names $L$, then this list becomes the scope of the variable.
% For all the other constructors of
% \elpiIn{tm}, the same term constructor is returned and its arguments are
% recursively compiled. The predicate in charge for term compilation is:

% \elpiIn{type comp tm -> tm -> links -> links -> subst -> subst -> o}.

% \noindent
% where, we take the term of the OL, produce the term of the ML, take a list
% of link and produce a list of new links, take a substitution and return a
% new substitution.

% In particular, due to programming constraints, we need to drag the old subst and
% return a new one extended, if needed, with the new declared meta-variables.

% The following code
% %
% \begin{elpicode}
%   kind link type.
%   type link nat -> nat -> nat -> subst.
% \end{elpicode}
% %
% \noindent
% defines a link, which is a relation between to variables indexes, the first
% being the index of a OL variable and the second being the index of a ML
% variable. The third integer\noteE[inline]{integer or nat?} is the number of term in the
% scope of the two variables, or equivalently, in a typed language, their arity.

% As an example, let's study the following unification problem (a slightly
% modified version from \cref{sec:intro}):

% \begin{elpicode}
%   lam x\ app[c"decision", app[c"nfact", x, c"3"]] ~\Uo~
%     lam x\ app [c"decision", app[uv 0, x]]
% \end{elpicode}

% \noindent
% we have the main unification problem where the nested \elpiIn{app} nodes have
% lists of different lengths making the unification to fail. The compilation of
% these terms produces a new unification problem with the following shape:

% \begin{elpicode}
%   lam x\ app[c"decision", app[c"nfact", x, c"3"]] ~\Ue~
%     lam x\ app [c"decision", uv 1 [x]]
% \end{elpicode}

% \noindent
% The main difference is the replacement of the subterm \elpiIn{app[uv 0, x]} of
% the OL with the subterm \elpiIn{uv 0 [x]}. Variable indexes are chosen by the
% ML, that is, the index \elpiIn{0} for that unification variable of the OL term
% has not the sam meaning of the index \elpiIn{0} in the ML. There exists two
% different substitution mapping, one for the OL and one for the ML and the indexes
% of variable point to the respective substitution.

% decomp che mappa abs verso lam
% \noindent
% \otext{An other example: \\
%   \elpiIn{lam x\ app[f, app[X, x]] = Y, (lam x\ x) = X.}}

% \section{Use of multivars}

% Se il termine initziale è della forma

% \begin{elpicode}
%   app[con"xxx", (lam x\ lam y\ Y y x), (lam x\ f)]
%   =
%   app[con"xxx",X,X]
% \end{elpicode}

% allora se non uso due X diverse non ho modo di recuperare il quoziente che mi manca.

% a sto punto consideriamo liste di problemi e così da eliminare sta xxx senza
% perdità di generalità (e facciamo problemi più corti, e modellizziamo anche la
% sequenza)

\section{Handling of \maybeeta}\label{sec:eta}
A term 
$\lambda x. t \appsep x$ is said to be the $\eta$-redex of $t$ if
$x$ does not occur free in $t$, and conversely we say that $t$ is the $\eta$-contraction of
$\lambda x. t \appsep x$. The equational theory of \Fo{} identifies these terms,
but the current compilation scheme does not,
as shown by the following example:
while $\lambda x.X \appsep x \Uo{} f$ does admit the solution
$\rho = \{~ X \mapsto f ~\}$, the corresponding problem in
\hoUnifPb does not.
%
\printAlll
  {{{\lambda x. X \appsep x, f}}}
  {{{\lambda x. A_x, f}}}
  {{{X,A,1}}}
  {{}}

\noindent
The reason is that <<\elpiIn{lam x\ uva A [x]}>> and
<<\elpiIn{con"f"}>> start with different term constructors.

In order to guarantee \cref{prop:simulation}, we detect
lambda abstractions that can disappear by $\eta$-contraction
(\cref{sec:etadetection}), and we modify the compiler so that it
generates fresh unification variables
in their place and moves the problematic term 
from \hoUnifPb to \linkStore (\cref{sec:etacomp}). The compilation
of the problem \foUnifPb above is refined to: 
%
\printAlll
  {{{\lambda x.X \appsep x,f}}}
  {{{A,f}}}
  {{{X,B,1}}}
  {{{\eta,,A,\lambda x.B_x}}}

\noindent
As per \cref{inv:linklhs} the \linketa left-hand side is a variable
while the right-hand side is a term in \maybeeta that has the following property:

\begin{invariant}[\linketa \rhs]
  The \rhs of any \linketa %in \linkStore 
  has the shape $\lambda x.t$
  and $t$ is in \wellb. 
  %where $t_x$ is a \maybeeta term and $x$ is free in $t$.
  \label{inv:link-eta-right}
\end{invariant}

Each \linketa is kept in the link store \linkStore during execution
and is activated under some conditions.
Activation is implemented
in \cref{sec:etaprogress}
by extending the \elpiIn{progress1}
predicate defined in \cref{sec:execution}.

\subsection{Detection of \maybeeta}\label{sec:etadetection}

When compiling a term $t$, we need to determine if any
subterm $s \in \subterm{t}$ that is of the form $\lambda x. r$
can be an $\eta$-redex, i.e. if
there exists a substitution $\rho$ such that $\rho (\lambda x.r) \Eo s$.
The detection of lambda abstractions that can ``disappear''
is not as trivial as it may seems. Here a few examples:
%
\begin{center}
  \begin{tabular}{lll}
    %Term & Status & Evidence \\\hline
    $\lambda x. f \appsep (A \appsep x)$ & $\in\maybeeta$ & $\rho = \{~ A \mapsto \lambda x.x ~\}$ \\
    $\lambda x. f \appsep (A \appsep x) \appsep x$ & $\in\maybeeta$ & $\rho = \{~ A \mapsto \lambda x.a ~\}$\\
    $\lambda x. f \appsep x \appsep (A \appsep x)$ & $\not\in\maybeeta$ &\\
    $\lambda x. \lambda y. f \appsep (A \appsep x) \appsep (B \appsep y \appsep x)$ & $\in\maybeeta$ & $\rho = \{~ A \mapsto \lambda x.x~;~ B \mapsto \lambda y.\lambda x.y ~\}$
  \end{tabular}
\end{center}
\vspace{4pt}

% In the examples above, the first term is a \maybeeta since $A_x$ can
% reduce to $x$ by setting $A_x = \lambda x.x$, 
% the second one is not a \maybeeta since it exists no substitution
% for $A_x$ such that $A_x$ reduces to $x$ and $x$ is not free in the subterm $f \appsep x$.
\noindent
The first two examples are easy, and show how a unification variable can expose
or erase a variable in its scope, turning the resulting term into an $\eta$-redex or not.\\
The third example shows that when a variable occurs outside the scope of a unification
variable, it cannot be erased and can hence prevent a term from being an $\eta$-redex.\\
The last example shows the recursive nature of the check we need to implement.
The term starts with a spine of two lambdas, hence the whole term
is in \maybeeta iff the inner term $\lambda y.f\appsep (A \appsep x) \appsep (B \appsep y \appsep x)$
is in \maybeeta itself. If it is, it could $\eta$-contract to
$f \appsep (A \appsep x)$ making $\lambda x.f \appsep (A \appsep x)$ a potential
$\eta$-redex.\\
We can now detail how \maybeeta terms are detected.

\newcommand{\reduceto}{\emph{may-contract-to}}
\begin{definition}[\reduceto]
  A $\beta$-normal term $s$ \reduceto{} a name $x$ if there exists a
  substitution $\rho$ such that $\rho s \Eo{} x$.
\end{definition}

\begin{lemma}\label{lem:reduceto}
A $\beta$-normal term $s = \lambda x_1 \ldots x_n.t$
%, where $x, x_1\ldots x_n$ can occur in $t$, 
\reduceto{} $x$ only if one of the following three conditions holds:
\begin{enumerate}
  \item $n = 0$ and $t = x$;
  \item $t$ is the application of $x$ to a list
     of terms $l$ and each $l_i$ \reduceto{} $x_i$
     (e.g. $\lambda x_1 \ldots x_n.x \appsep x_1 \ldots x_n \Eo{} x$) ;
  \item $t$ is a unification variable with scope
    $W$, and for any $v \in \{ x, x_1 \ldots x_n \}$,
    there exists a $w \in W$, such that $w$ \reduceto{} $v$
    (if $n = 0$ this is equivalent to $x \in W$).
\end{enumerate}
\end{lemma}
\begin{proof}[Proof sketch]
Since our terms are in $\beta$-normal form the only
rule that can play a role is \ref{clause:eta1}, and note that
that rule uses \elpiIn{beta} to add an argument to an
application or to the scope of a variable.
If the term $s$ is not exactly $x$ (case 1)
it can only be an $\eta$-redex of $x$, or a unification
variable that can be assigned to $x$, or a combination of both.
If $s$ begins with a lambda, then the lambda can only disappear by $\eta$
contraction. In that case, the term $t$ under the spine of binders
$x_1\ldots x_n$ can either be $x$ applied to terms that
\reduceto{} these variables (case 2), or a unification variable
that can be assigned to the application $x \appsep x_1 \ldots x_n$ (case 3).
\end{proof}

% \noindent
% Note that this condition does not require the term to be in \llambda.
% \noteE[inline]{Is this relevant}

\newcommand{\occursrigid}{\emph{occurs-rigidly}\xspace}
\newcommand{\occurrigid}{\emph{occur-rigidly}\xspace}
\begin{definition}[\occursrigid]\label{def:occrigid}
  A name $x$ \occursrigid{} in a $\beta$-normal term $t$, if $\forall \rho, x \in
  \subterm{\rho t}$
\end{definition}

In other words, $x$ \occursrigid in $t$ if it occurs in $t$
outside of the scope of a unification variable $X$; otherwise, an instantiation
of $X$ can make $x$ disappears from $t$.
Note that $\eta$-contracting $t$ cannot make $x$ disappear, since
$x$ is not a locally bound variable inside $t$.

We can now describe the implementation of \maybeeta detection:

\newcommand{\testmaybeeta}{\emph{maybe-eta}\xspace}
\begin{definition}[\testmaybeeta]\label{def:testmaybeeta}
  Given a $\beta$-normal term
  $s = \lambda x_1 \ldots x_n.t$, \testmaybeeta{} $s$ holds if any
  of the following holds:
  \begin{enumerate}
    \item $t$ is a constant $c$ or a name $y$ applied to the arguments
      $l_1 \ldots l_m$ such that 
      $m \geq n$ and for every $i$ such that $m - n < i \leq m$
      the term  $l_i$
      \reduceto{} $x_i$, and
      no $x_i$ \occursrigid{} in $l_1 \ldots l_{m-n} ~ y$;
    \item $t$ is a
      unification variable with scope $W$ and
      for each $x_i$ there exists a $w_j \in W$ such that $w_j$
      \reduceto{} $x_i$.
  \end{enumerate}
\end{definition}
\begin{lemma}[\maybeeta detection]\label{lem:maybeeta}
  If $t$ is a $\beta$-normal term
  and
  $t \in \maybeeta$ then \testmaybeeta{} $t$ holds.
\end{lemma}
\begin{proof}[Proof sketch]
Follows from \cref{def:occrigid} and \cref{lem:reduceto}
\end{proof}

\noindent
Remark that the converse of \cref{lem:maybeeta} does not hold: 
there exists a term $t$ satisfying the criteria (1) of
\cref{def:testmaybeeta} that is not in $\maybeeta$, i.e.
there exists no substitution $\rho$ such that $\rho t$ is an
$\eta$-redex. A simple counter example is
$\lambda x. f \appsep (A \appsep x) \appsep (A \appsep x)$
since $x$ does not \occurrigid{} in the first argument
of $f$,
and the second argument of $f$ \reduceto{} $x$.
In other words $A \appsep x$
may either use or discard $x$, but our analysis does not
take into account that \emph{the same
term} cannot have two contrasting behaviors.

As we will see in the rest of this section, this is not a problem
since it does not break
\cref{prop:simulation} nor \cref{prop:fidelity}.
% A term in \maybeeta{} is compiled to a
% unification variable and a link (see \cref{sec:etacomp}):
% the link makes progress (see \cref{sec:etaprogress})
% in the same step in which the
% variable is instantiated, and that compensates
% for this coarse analysis.

% The implementation we propose for the \maybeeta relation is given below.

% \begin{elpicode}
  type occurs-rigidly to -> to -> o.
  occurs-rigidly N N.
  occurs-rigidly _ (o-app [o-uva _|_]) :- !, fail.
  occurs-rigidly N (o-app L) :- exists (occurs-rigidly N) L.
  occurs-rigidly N (o-lam B) :- pi x\ occurs-rigidly N (B x).

  type reducible-to list to -> to -> to -> o.
  reducible-to _ N N :- !.
  reducible-to L N (o-app[o-uva _|Args]) :- !, 
    forall1 (x\ exists (reducible-to [] x) Args) [N|L]. 
  reducible-to L N (o-lam B) :- !, 
    pi x\ reducible-to [x | L] N (B x).
  reducible-to L N (o-app [N|Args]) :-
    last-n {len L} Args R,
    forall2 (reducible-to []) R {rev L}.

  type maybe-eta to -> list to -> o.                  ~\customlabel{rule:maybeeta}{(\maybeeta)}~
  maybe-eta (o-app[o-uva _|Args]) L :- !,
    forall1 (x\ exists (reducible-to [] x) Args) L, !. 
  maybe-eta (o-lam B) L :- !, pi x\ maybe-eta (B x) [x | L].
  maybe-eta (o-app [T|Args]) L :- (name T; T = o-con _),
    split-last-n {len L} Args First Last,
    none (x\ exists (y\ occurs-rigidly x y) First) L,
    forall2 (reducible-to []) {rev L} Last.
\end{elpicode}


% Here a complex maybeeta example
% \begin{gather}
%   T = \lambda y. f \appsep A_{xy} \appsep (B \appsep a \appsep (\lambda z.y\appsep C_z)) \appsep D_x
% \end{gather}

\subsection{Compilation and decompilation}\label{sec:etacomp}

% Thanks to the \elpiIn{maybe-eta} predicate, we can detect ``$\eta$-problematic''
% terms and, consequently replace them with fresh \Ho unification variables at
% compilation time. The code below illustrate how this relation is used to for
% term compilation.

\paragraph{Compilation}
The following rule is inserted just before rule~\ref{rule:complam} from the code in
\cref{sec:compilation}.

\begin{elpicode}
  comp (o-lam F) (uva A Scope) M1 M2 L1 L3 S1 S3 :-
    maybe-eta (o-lam F) [], !,
      alloc S1 A S2,
      comp-lam F F1 M1 M2 L1 L2 S2 S3,
      get-scope (lam F1) Scope, 
      L3 = [val (link-eta (uva A Scope) (lam F1)) | L2].
\end{elpicode}


\noindent
Whenever \elpiIn{o-lam F} is detected to be in
\maybeeta it is compiled to \elpiIn{lam F1} and replaced by the fresh
variable \elpiIn{A}. This variable sees all the names free in
\elpiIn{lam F1} and is connected to \elpiIn{lam F1} via a \linketa.
\Fullref{inv:linklhs} holds for this link. Moreover:

\begin{corollary}
  The \rhs of any \linketa has exactly one lambda abstraction, hence
  the rule above respects \fullref{inv:link-eta-right}.
  \label{cor:rhs-eta-onelamb}
\end{corollary}

\begin{proof}[Proof sketch]
  By contradiction, suppose that the rule above is applied and that
  the \rhs of the link is $\lambda x.\lambda y.t$, where $x$ and $y$ occur in $t$.
  If $\testmaybeeta{}~\lambda y.t$ holds then the recursive call to
  \elpiIn{comp} (made by \elpiIn{comp-lam}) must have put a fresh variable
  in its place, so this case is impossible.
  Otherwise, if $\testmaybeeta{}~\lambda y.t$ does not hold, then also
  $\testmaybeeta{}~\lambda x.\lambda y.t$ does not hold either, contradicting
  the assumption that the rule was applied.
\end{proof}

\begin{lemma}[$\wellb$-partial-enforcement]\label{lem:w-enforcement-maybeeta}
  $\forall s, \C{s}{t}{m}{l}$, if $\:\wellb_{\beta \eta}(s)$  then $\:\wellb(t)$.
\end{lemma}

\begin{proof}[Proof sketch]
  By \cref{lem:maybeeta}
\end{proof}


\paragraph{Decompilation}
The decompilation of a \linketa is performed by unifying
the \lhs with the \rhs. Note that this unification never fails, since the \lhs is a
flexible term not appearing in any other \linketa
(by \cref{def:progressetadedup}).

\subsection{Progress}\label{sec:etaprogress}

\linketa{}s are meant to delay the unification of ``problematic'' terms until
we know for sure if their head lambdas can be $\eta$-contracted.

\newcommand{\progressetaleft}{\emph{$\eta$-progress-\lhs}\xspace}
\begin{definition}[\progressetaleft]\label{def:progressetaleft}
A link \linketaM{\Gamma}{X}{t} is removed from \linkStore when
$X$ becomes rigid. Let $y\in\Gamma$. There are two cases:
\begin{enumerate}
  \item if $X = a$ or $X = y$ or $X = f \appsep a_1\ldots a_n$
    we unify the $\eta$-redex of $X$ with $t$, that is we run
    $\lambda x.X \appsep x\Ue{} t$
    \item if $X = \lambda x.s$ we run $X \Ue{} t$.
\end{enumerate}
\end{definition}

% There is a third case in which a link is removed from \linkStore, namely
% when the \lhs is assigned to a variable that is the \lhs of another
% \linketa.

\newcommand{\progressetadedup}{\emph{$\eta$-progress-deduplicate}\xspace}
\begin{definition}[\progressetadedup]\label{def:progressetadedup}
  A link \linketaM{\Gamma}{X_{\vec{s}}}{T} is removed from \linkStore when
  another link \linketaM{\Delta}{X_{\vec{r}}}{T'} is in  \linkStore.
  By \cref{inv:uvaarity} the length of $\vec{s}$ and $\vec{r}$ is the same;
  hence we can move the term $T'$ from $\Delta$ to $\Gamma$ by renaming its
  bound variables, i.e. $T'' = T'[\vec{r}/\vec{s}]$.
  We then run $T \Ue{} T''$ (under the context $\Gamma$).
\end{definition}

\begin{lemma}
  Let $\lambda x.t$ be the \rhs of a \linketa, then $\wellb(t)$, enforcing \cref{inv:link-eta-right}.
  \label{lemma:unif-eta-aux}
\end{lemma}

\begin{proof}[Proof sketch]
  By \cref{lem:w-enforcement-maybeeta}.
\end{proof}

% \noteD[inline]{Below the proof of \cref{prop:w-preservation}, ho usato 3 lemmi ausiliari,
% forse si può compattare in una prova più piccola?}
\begin{lemma}
  Given a \linketa \linketaM{\Gamma}{X}{\lambda x.t}, the unification done by \progressetaleft is between
  terms in \wellb.
  \label{lemma:unif-eta-1}
\end{lemma}

\begin{proof}[Proof sketch]
  Let $\sigma$ be a substitution such that $\wellb(\sigma X)$ (since \Ue preserves \wellb).
  By \cref{inv:link-eta-right}, we have $\wellb(t)$.
  If $\sigma X = \lambda x.r$, then $r$ is unified with $t$ and both terms are in \wellb.
  Otherwise, $\lambda x. X\appsep x$ is unified with $\lambda x. t$, and $X\appsep x$ and $t$
  are both in \wellb.
\end{proof}

% \begin{lemma}
%   Given a \linketa \linketaM{\Gamma}{X}{\lambda x.t}, the unification done by \progressetaright is between
%   terms in \wellb.
%   \label{lemma:unif-eta-2}
% \end{lemma}

% \begin{proof}[Proof sketch]
%   $X$ is variable, and, by \cref{def:progressetaright}, $\lambda x.t$ is either no more
%   a \maybeeta, i.e. it is not a $\eta$-redex and, so, $\wellb(\lambda x.t)$, 
%   the unification is performed between $X$ and $\lambda x.t$.
%   Otherwise, $\lambda x.t$ has the shape $\lambda x.t' \appsep x$ with $x$ not free in $t'$ and
%    $\wellb(t')$. In this second case the unification is done between $X$ and $t'$.
%   In both cases, unifications are between \wellb terms.
% \end{proof}

\begin{lemma}
  The unification done by \progressetadedup is between
  terms in \wellb.
  \label{lemma:unif-eta-3}
\end{lemma}

\begin{proof}
  Given two \linketa{}s \linketaM{\Gamma_1}{X}{\lambda x.t} and
  \linketaM{\Gamma_2}{Y}{\lambda x.t'} 
  the unification is performed between $t$ and $t'$.
  By \cref{lem:w-enforcement-maybeeta} both terms are in
  \wellb.
\end{proof}

\begin{lemma}
  The progress of \linketa guarantees \fullref{prop:w-preservation}
  \label{lemma:unif-wellb}
\end{lemma}

\begin{proof}[Proof sketch]
  By \cref{lemma:unif-eta-1,lemma:unif-eta-3}, every
  unification performed by the activation of a \linketa is between
  terms in \wellb.
\end{proof}

\begin{lemma}
  \elpiIn{progress} terminates.
  \label{lemma:prog-eta-terminates}
\end{lemma}

\begin{proof}[Proof sketch]
  Rules \ref{def:progressetaleft} and
  \ref{def:progressetadedup} remove one link from \linkStore, hence they
  cannot be applied indefinitely.
  Moreover each rule only relies on terminating operations such as \Ue,
  $\eta$-contraction, $\eta$-redex, relocation (a recursive copy of a
  finite term).
  % The addition of rules for \elpiIn{progres1} complicates the function
  % \elpiIn{progress}. We can note, however, that they do not prevent the
  % termination of \elpiIn{progress}. 1) If a link is activated it is removed from
  % \linkStore and the recursive call to \elpiIn{progress} will have a smaller
  % list of links to recurse on. Moreover, link activation only runs terminating
  % instructions (such as unification). 2) If a link is deduplicated, the
  % termination of \elpiIn{progress} is still guaranteed since again we reduce
  % \linkStore and the instructions run by link deduplications are all
  % terminating. 3) If a link is neither activated nor deduplicated, i.e. it
  % remains suspended, then \linkStore remains unchanged like the substitution;
  % therefore, \elpiIn{if (L = L1, S1 = S2)} succeeds and \elpiIn{progress}
  % terminates.
\end{proof}  

\begin{theorem}[Fidelity in $\wellb_{\beta \eta}$]
  % F x = z, F x y = z y
  Given a list of unification problems \foUnifPb such that
  $\wellb_{\beta \eta}(\foUnifPb)$, if the 
  memory map is bijective then
  the introduction 
  of \linketa guarantees \fullref{prop:fidelity}.
  %\footnote{The bijective property of \mapStore is proved in \cref{sec:invariant1}}
  \label{lemma:fidelity-maybeeta}
\end{theorem}

\begin{proof}[Proof sketch] %\todo{to reformulate}
  % For any step $i$, we want to prove that $\fstep_i$ succeeds iff $\hstep_i$
  % succeeds. A $\hstep_i$ is made by a unification
  % $u = \sigma_{i-1}\hoUnifPb_{i_l} \Ue \sigma_{i-1}\hoUnifPb_{i_r} \mapsto \sigma_i'$ and a link
  % progression $p = \progress(\linkStore_{i-1},\sigma_i') \mapsto (\linkStore_i,\sigma_i)$.
  We want to prove that \fstep succeeds iff \hstep
  succeeds, where \hstep is made by a unification step $u$ and a link
  progression $p$.
  Without loss of generality we consider a
  \Fo{} unification problem of the form $s_1 \Uo s_2$
  where $s_1 \in \maybeeta$ and is compiled to a variable $X$ with the accessory link
  $\linketaM{\Gamma}{X}{\lambda x.t_1}$. 
  The unification $u$ always succeeds, since $X$ is a fresh variable, we 
  only have to consider the link progression $p$.
  Two cases should be analyzed:
  \begin{itemize}
    \item $s_2 \in \maybeeta$, in this case, the unification problem becomes $X \Ue Y$ 
          with the extra link $\linketaM{}{Y}{\lambda x.t_2}$. After the unification of 
          $X$ and $Y$, \progressetadedup is fired and, once the lambdas are crossed,
          $t_1$ and $t_2$ are unified. 
          This unification succeeds iff $s_1 \Uo s_2$ succeeds by \cref{th:fidelity-beta}.


    \item $s_2 \notin \maybeeta$,
          in this case $s_2$ is compiled to $t_2$ and the unification step $u$ assigns $t_2$ to $X$
          and triggers \progressetaleft.
          If $s_1$ and $s_2$ are equal thanks to \ref{clause:eta1} then
          progression $p$ unifies $\lambda x.t_1$ with the $\eta$-redex of $t_2$ namly
          $\lambda x.t_2\appsep x$, otherwise they are equal thanks to rule \ref{clause:lam-lam}
          and the link progress $p$ unifies $\lambda x.t_1$ with $t_2$.
          Alternatively, if $s_1$ are different $s_2$
          it cannot be because of the $\lambda$ constructor in the head of $t_1$,
          since rule \ref{clause:lam-lam} or rule \ref{clause:eta1} move under it, eventually finding two
          different subterms $s_1'$ and $s_2'$. By \cref{th:fidelity-beta},
          \Ue fails on terms $t_1'$ and $t_2'$ even if \progressetaleft
          $\eta$-expands $t_2$.

  \end{itemize}
  % wlog we assume s1 is in \maybeeta, s2 is not (if both are thanks to dedup we are back in this situation).
  % we assume s1 is compiled to X and $\linketaM{\Gamma}{X}{t1}$.
  % In all cases $X \Ue t2$ succeeds since X is a variable and
  % hence assigns t2 to X.
  % Rule \progressetaleft unify with \Ue X (hence t2) with t1, with the only
  % additional care of eta expanding t2 if needed (5.9.1).
% 
  % Two cases should be analyzed: 1) $\fstep_i$ succeeds to unify 
  % $s1$ with $s2$. The interesting case is when fostep succeds by
  % finding a $\rho$ such that $\rho s1 \Eo \rho s2$ requires
  % rules~\ref{clause:eta1} and \ref{clause:eta2} at the top level,
  % and that is in exactly the cases in which T (from page 3) is
  % not a lam (see 5.9.1) compensating for the missing rules in \Ue.
  % 
  % 2) If $\fstep_i$ fails to unify $s1 \Uo s2$
  % it means that, after applying rules~\ref{clause:eta1} and \ref{clause:eta2}
  % and the one o-lam-o-lam we find different terms.
  % (5.9.1) only adds lambdas that are forcibly consumed by
  % lam-lam rule.
\end{proof}

% Note that \progressetadedup always unifies terms starting with exaclty one
% lambda, hence lam-lma, so we can't fail due to that lambda or succed thanks
% to that lambda. Hence it does not impact fidelity.

\begin{corollary}[Properties of \Uo in $\wellb_{\beta\eta}$]\label{prop:unifetawb}
  Given the updated compilation scheme, procedure \Uo of~\cref{sec:founif}
  is a good unification (as per \cref{prop:good-unif})
  for \Eo in the domain $\wellb_{\beta\eta}$.
\end{corollary}

\paragraph{Example of \progressetaleft}

The example at the beginning of \cref{sec:eta}, once
$\sigma = \{~ A \mapsto f ~\}$, triggers \progressetaleft since the link
becomes \linketaM{}{f}{\lambda x.B_{x}} and the \lhs is a constant.
This rule runs $\lambda x.f \appsep x \Ue{} \lambda x.B_{x}$,
resulting in $\sigma = \{~ A \mapsto f ~;~ B_{x} \mapsto f ~\}$.
Since $X$ is mapped to $B$ and $B_x$ is decompiled into $\lambda x.f \appsep x$,
by the definition of \linketa decompilation
$\sigma$ is decompiled to $\rho = \{~ X \mapsto \lambda x. f \appsep x ~\}$.

% \todo{explain why decompile does so and which eta?}

\begin{comment}
\paragraph{Example of \progressetadedup}

A very basic example of \linketa deduplication, is given below:
% make test ONLY=7002 TEX=tex
\printAlll
  {{{\lambda x.(X\appsep x),\lambda x.(Y\appsep x)}}}
  {{{A,C}}}
  {{{X,B,1},
    {Y,D,1}}}
  {{{\eta,,A,\lambda x.B_{x}},
    {\eta,,C,\lambda x.D_{x}}}}

\noindent
The result of $A \Ue{} C$ is that the two \linketa share the same \lhs.
By unifying the two \rhs we get
$\sigma = \{ A~ \mapsto C, B \mapsto D ~\}$.
In turn, given the map \mapStore, this second assignment is decompiled to
$\rho = \{~ X \mapsto Y ~\}$ as expected.
\end{comment}



%\section{Enforcing inv.: \NoCaseChange{\nameref{inv:uvaarity}}}
\section{Making \mapStore a bijection}
\label{sec:invariant1}

We want to allow for partial application of functions.
As a consequence
the same unification variable $X$ can be used multiple times
with different arities.
When it is the case the
memory map \mapStore generated by the compiler is not
a bijection. For example:
\begin{comment}
%
% make test ONLY=7001 TEX=tex
\printAlll
  {{{\lambda x.\lambda y.(X\appsep y\appsep x),\lambda x.\lambda y.x},
    {\lambda x.(f\appsep (X\appsep x)\appsep x),Y}}}
  {{{A,\lambda x.\lambda y.x},
    {D,F}}}
  {{{X,E,1},
    {Y,F,0},
    {X,C,2}}}
  {{{\eta,,D,\lambda x.(f\appsep E_{x}\appsep x)},
    {\eta,,A,\lambda x.B_{x}},
    {\eta,x,B_{x},\lambda y.C_{y x}}}}
\end{comment}
% make test ONLY=7007 TEX=tex
\printAlll
  {{{\lambda x.(X\appsep x),f},
    {X,\lambda x.a}}}
  {{{A,f},
    {C,\lambda x.a}}}
  {{{X,C,0},
    {X,B,1}}}
  {{{\eta,,A,\lambda x.B_{x}}}}
% \todo{simplify example since no prog right}
in the unification problems \foUnifPb above, $X$ is used with arity $1$ in
$\foUnifPb_1$ and with arity $0$ in $\foUnifPb_2$. 
In order to preserve \fullref{inv:uvaarity} the compiler
did generate two entries in \mapStore for $X$, namely $B$ and $C$. 
However, there is no connection between these two variables
and any incompatible assignements to them would not be detected,
in turn breaking \cref{prop:fidelity}. 
% This incompatibility will only be detected during the
% decompilation phase.
To address this issue, we post-process \mapStore
to ensure the following property:

\begin{proposition}[\mapStore is a bijection]
  After compilation, for each \Fo{}-variable $X$ in \foUnifPb{} and for each
  \Ho{}-variable $A$ in \hoUnifPb{} there is exactly one entry $X \mapsto A^n$
  in \mapStore.
\end{proposition}

\noindent
Note that running \Ue{} may require allocating new variables in \Ho{} as
explained in~\cref{sec:decompilation}. The property above ignores these variables since they are fresh and have no corresponding
variable in \Fo.

The core procedure of this post-processing step is \emph{align-arity} that
is iterated by \emph{map-deduplication}:

\newcommand{\alignarity}{\emph{align-arity}}
\begin{definition}[\alignarity] Given two mappings
  $m_1 : X \mapsto A^m$ and
  $m_2 : X \mapsto C^n$ where $m < n$ and $d = n - m$,
  $\alignarity{}~ m_1 ~ m_2$ generates the following $d$ links, one
  for each $i$ such that $0 \leq i < d$,
\[%
  \linketaM{x_0 \ldots x_{m+i}}
           {B^i_{x_0 \ldots x_{m+i}}}
           {\lambda x_{m+i+1}.B^{i+1}_{x_0 \ldots x_{m+i+1}}}
\]%
where $B^i$ is a fresh variable of arity $m+i$, and $B^0 = A$ as well as $B^d = C$.
\end{definition}

The intuition is that we $\eta$-expand the occurrence of the variable
with lower arity to match the higher arity. Since each \linketa can
add exactly one lambda, we need as many links as the difference between the
two arities.

\newcommand{\mapdeduplication}{\emph{map-deduplication}\xspace}

\begin{definition}[\mapdeduplication]
  Forall mappings $m_1, m_2\in \mapStore$ such that 
  $m_1 : X \mapsto A^m$ and
  $m_2 : X \mapsto C^n$ and $m < n$
  we remove $m_1$ from \mapStore and
  add to \linkStore the result of $\alignarity{}~m_1~m_2$.
\end{definition}

\begin{comment}
In order to grant fidelity, we add a new link progression for \linketa{}s.

\newcommand{\progressetaright}{\emph{$\eta$-progress-\rhs}\xspace}
\begin{definition}[\progressetaright]\label{def:progressetaright} A link
\linketaM{\Gamma}{X}{t} is removed from \linkStore when either 1)
$\testmaybeeta~t$ does not hold (anymore) or 2) by $\eta$-contracting $t$ to
$t'$ where $t'$ does not start with the \elpiIn{lam} constructor. In the
first case, $X$ is unified with $t$, and in the second one, $X$ is unified with
$t'$ (under the context $\Gamma$).
\end{definition}
\end{comment}

\begin{theorem}[Fidelity with \mapdeduplication]
  Given a list of unification problems \foUnifPb, such that 
  $\wellb_{\beta \eta}(\foUnifPb)$,
  if $\ \foUnifPb$ contains the same \Fo{}-variable used at different arities,
  then \mapdeduplication guarantees \fullref{prop:fidelity}
\end{theorem}

\def\chain{\ensuremath{\vec{c}}\xspace}

\begin{proof}[Proof sketch]
  Let $X$ be a \Fo{}-variable appearing in two different subterms $s_1$ and $s_2$.
  In $s_1$ variable $X$ is applied to the list of terms $x_1\dots x_m$ while, in $s_2$,
  it is applied to $y_1\dots y_n$. Without loss of generality we take $m < n$.
  After \mapdeduplication we have two distinct \Ho{}-variables for $X$, say $A$ and $C$,
  related by a chain \chain of \linketa{}s of length $n - m$. We prove that
  if $A$ and $C$ are assigned respectively to the \wellb terms $\lambda x_1\dots
  x_p.t_1$ and $\lambda x_1\dots x_q.t_2$, then a failure occurs iff
  these terms do not unify. By \progressetaleft, the assignment of $A$
  activates $p$ links of \chain. In particular, when $p \geq n$, then the entire
  chain collapses and unification is performed between $\lambda x_n\dots
  x_p.t_1$ and $\lambda x_1\dots x_q.t_2$. By \cref{th:fidelity-beta} and since
  we work with \wellb terms, this unification succeeds iff it does in \Fo.
  Otherwise if $p < n$ then we have to consider two cases. 1) If $t_1$ is a rigid
  term then $t_2$ must be the $\eta$-redex of $t_1$, implying $q = 0$. If it
  is not the case, then \progressetaleft
  would eventually cause a failure. 2) If $t_1$ is a
  unification variable then \elpiIn{scope-check} guarantees that $t_2$ does not use
  $x_1\dots x_p$ as free variables, causing otherwise a unification failure.
  After this check, $m - n - p$ links of \chain remain suspended, as we expect:
  decompilation is charged to wake all links in \linkStore and eventually relate
  $t_1$ with $t_2$.
  %
  % Attenzione a questo caso:
  %    A =𝜂 λx.(B x)   e    x |- B x =𝜂 λx.(C x)           (from a map dedup, where A e C, sono stessa variabile con diff arities)
  %  se (C x) è instanziato a (f (D x)) allora non c'è nessun progress:
  %  il rhs del secondo link diventa (λx.(f (D x))) che è un maybe-eta
  % 
  % Nella nuova versione sottintendo questo caso: p = 0 e t_1 è flessibile
  %
  % Without loss of generality, we pose $m < n$. If $A^m$ is assigned to a term
  % $\lambda \vec{x}.t$ with $q$ being the length of $\vec{x}$. If $t$ is a rigid
  % term, then the chain of links between $A^m$ and $C^n$ is completely triggered
  % so that $C^n$ is assigned to $\lambda \vec{x}.t$ $\eta$-expanded $n-m$ times.
  % Note that in this case fidelity is guaranteed, since the term unified with $C^n$
  % sohuld also unify with $A^m$. Otherwise, if $t$ is a variable, the chain of
  % links is shrinked by $q$. Note that if this length is bigger then $n-m$
  % the full chain collapse. % and $A^m$ and $C^n$ are agian tiny coupled. 
  % Otherwise,
  % if $q < n-m$, it means that $C^n$ can always be assigned to any term starting
  % with at least $q$ lambdas. In this second case fidelity is also guaranteed,
  % due to the constraint of the number of lambda by which $C^n$ should start.
  %
  % not starting with $\lambda$, then the entrire chain of link is
  % immediately waken up, assigning $C^n$ to $x_1\dots x_{n-m} \vdash t \appsep
  % x_1\dots x_{n-m}$, which is the $t$ $\eta$-expanded $n-m$ times. Hence, $A^m$
  % and $C^n$ are tiny coupled. If $A^m$ is assigned to a term of the form
  % $\lambda \vec{x}.t$, then if $t$ is rigid, the chain of links between $A^m$
  % and $C^n$ completely triggered, is shrinked by the length of $\vec{x}$. Note
  % that if $t$ is rigid, by the previous sentence, the remi
  %
  % If $A^m$ is instantiated, 
  %
  % Whenever the two variables are assigned to the non-unifiable terms,
  % the chain of \linketa is entirely waken up refining little by little
  % the intermediate variables between $A^m$ and $C^n$ leading eventually to a failure. 
  % On the other
  % hand, if the two variables still remain unifiable after their assignement, 
  % then the activation of 
  % the corresponding \linketa just refines the chain, by reducing eventually
  % its length.
\end{proof}

% \todo{this proof is still unreadable}
% \todo{Not sure I fullfy fixed the proof, but the old oen was not ok}

% Note that shortening the chain of links allows unification to fail if
% $X_1$ and $X_2$ are unified with different terms.
% \todo{what are these X1 and X2?}

\paragraph{Example of \mapdeduplication}
If we take back the example given at the beginning of this section,
\mapdeduplication produces the following result:
% 
% make test ONLY=7007 TEX=tex
\printAlll
  {{{\lambda x.(X\appsep x),f},
    {X,\lambda x.a}}}
  {{{A,f},
    {C,\lambda x.a}}}
  {{{X,B,1}}}
  {{{\eta,,C,\lambda x.B_{x}},
    {\eta,,A,\lambda x.B_{x}}}}

\noindent
Note that $\mapping{X}{C}{0}$ disappears from \mapStore
in favour of the auxiliary \linketa \linketaM{}{C}{\lambda x.B_{x}}.
The resolution of $\hoUnifPb_1$ assignes $a$ to $A$. This wakes up
$\linkStore_2$ by \progressetaleft, assigning <<$f \appsep x$>> to $B_x$.
The resolution of $\hoUnifPb_2$ instantiates $C$ to $\lambda x.a$, which, in
turn triggers $\linkStore_1$ by \progressetaleft. In turn, this performs
$\lambda x.a \Ue \lambda x.(f \appsep x)$ that fails as expected.

\begin{comment}
% make test ONLY=7001 TEX=tex
\printAlll
  {{{\lambda x.\lambda y.(X\appsep y\appsep x),\lambda x.\lambda y.x},
    {\lambda x.(f\appsep (X\appsep x)\appsep x),Y}}}
  {{{A,\lambda x.\lambda y.x},
    {D,F}}}
  {{{Y,F,0},
    {X,C,2}}}
  {{{\eta,x,E_{x},\lambda y.C_{x y}},
    {\eta,,D,\lambda x.(f\appsep E_{x}\appsep x)},
    {\eta,,A,\lambda x.B_{x}},
    {\eta,x,B_{x},\lambda y.C_{y x}}}}

\noindent
The resolution of $\hoUnifPb_1$ assigns $\lambda x.\lambda y.x$ to $A$
that in turn triggers $\linkStore_3$ and then $\linkStore_4$ by \progressetaleft,
one per $\lambda$ in the solution of $A$.
The unification variable $C_{yx}$ is therefore unified with $x$
(the second variable of its scope).
As a result $\linkStore_1$ becomes \linketaM{x}{E_x}{\lambda y.y},
and the \rhs is no more in \maybeeta so \progressetaright fires and
$E_x$ is unified with $\lambda y.y$; then $\linkStore_2$ makes progress thanks to the same 
rule giving
$\sigma = \{~ A \mapsto \lambda x.\lambda y.x ~;~ B_x \mapsto \lambda y.x ~;~ C_{yx} \mapsto x ~;~ D \mapsto f\appsep (\lambda y.y) ~;~ E_x \mapsto \lambda y.y ~\}$.\\
The remaining step $\hoUnifPb_2$ identifies $D$ with $F$,
hence the resulting \Fo{} substitution is
$\rho = \{X \mapsto \lambda x.\lambda y.y ~;~ Y \mapsto f\appsep (\lambda y.y)\}$.
\end{comment}

\section{Handling of \notllambda}\label{sec:beta}

As observed in~\cite{Nadathur2001} it is worth 
handling terms in \notllambda since, in practice, these terms often
re-enter \llambda at runtime.
%for example,
%or to accommodate for the introduction of heuristics.

In the following example problem $\foUnifPb_2$, namely $X \appsep a \Uo a$,
admits two different solutions: $\rho_1 = \{~X \mapsto \lambda x.x~\}$
and $\rho_2 = \{~X \mapsto \lambda x.a~\}$.
The unification algorithm alone
cannot chose the right one, since no solution is more general than the other,
but for the first problem the choice is obvious, and in turn
this choice makes $\foUnifPb_2 \subseteq \llambda$:
%
\printAlll
  {{{X,\lambda x.a},
    {(X\appsep a),a}}}
  {{{A,\lambda x.a},
    {(A\appsep a),a}}}
  {{{X,A,0}}}
  {{}}

% In this section we suppose the unification of the object language between two
% terms $t_1$ and $t_2$ to fail each time at least one of the between $t_1$ or
% $t_2$ is outside $\llambda$. This means for instance that $X \nUo Y \appsep Z$ and 
% $X\appsep Y \nUo X\appsep Y$.

% On
% the other hand, we also point out that the \maybeeta detection spot out
% potential $\eta$-redex for terms that are not in \llambda. For example,
% $\lambda x.F \appsep G_x$ is considered as \maybeeta, since we have the
% application of term whose argument can reduce to $x$.

% \noteE[inline]{IIRC, second order unification, where variables can stand for functions, is semi-decidable, indeed Huet's algorith enumerates all solutions. Third order, functions that take functions, is undecidable, proved by Dowek. IMO here the problem is that there is no unique solution, (semi)decidability is a secondary point. This paragraph needs fixing.}
% \noteD[inline]{I've rewritten it, it is clearer?}
% In general, unification between \notllambda terms admits more then one
% solution and committing one of them in the substitution does not guarantee
% \cref{prop:complete-ml}. For instance, $X \appsep a \Uo a$ 
% admits two different substitutions: $\rho_1 = \{X \mapsto \lambda x.x\}$
% and $\rho_2 = \{X \mapsto \lambda \_.a\}$. Prefer one over the other may break
% future unifications.

% Given a list of unification problems, $\foUnifPb_1\dots
% \foUnifPb_n$ with $\foUnifPb_n$ in \notllambda, it is often the case that the resolution of
% $\bigwedge_{i=0}^{n-1}\foUnifPb_i$ gives a partial substitution $\rho$, such
% that $\rho\foUnifPb_n$ falls again in \llambda.
% %
% % make test ONLY=7003 TEX=TEX + deactivate beta tex
% \printAlll
%   {{{X,\lambda x.a},
%     {(X\appsep a),a}}}
%   {{{A,\lambda x.a},
%     {(A\appsep a),a}}}
%   {{{X,A,0}}}
%   {{}}

\noindent
In order to support this scenario we have to improve a little our compiler
and \progress routines.
In particular $\hoUnifPb_1$ generates $\sigma = \{~ A
\mapsto \lambda x. a ~\}$ making $\sigma\hoUnifPb_2$ equal to
$(\lambda x. a) \appsep a \Ue a$ that \Ue{} cannot solve since it
lacks rules \ref{clause:beta1} and \ref{clause:beta2}.

To address this problem the compiler must recognize
\notllambda terms
and replace them
with fresh variables and generate a new kind of links
that we call \linkbeta.

In addition to \fullref{inv:linklhs}, the term on the \rhs of a \linkbeta
has the following property:

\newcommand{\rhsBetaHead}{\ensuremath{X_{s_1\dots s_n}}}
\newcommand{\rhsBeta}{\ensuremath{\rhsBetaHead\appsep t_1\dots t_m}\xspace}
% \noteE[inline]{Io direi $X_{s_1\ldots s_n}\appsep l_1\ldots l_m$
% ... where S = [s1 .. sn] e L = [l1 .. lm]}

\begin{invariant}[\linkbeta \rhs]
  The \rhs of any \linkbeta has the shape $X_{s_1\ldots s_n}\appsep t_1\dots t_m$
  where $X$ is a unification variable in \llambda (with scope $s_1\dots s_n$)
  and $t_1\dots t_m$ is a list of terms such that $m>0$ and
  $t_1$ is either a variable occurring in $s_1\dots s_n$ or
  a term other than a variable.
  \label{inv:beta-rhs}
\end{invariant}

\noindent Note that the shape of such as \rhs is
  <<\elpiIn{app [uva X S | L]}>>, where
  \elpiIn{S} is \elpiIn{[s~$_1$~, ~$\ldots$~, s~$_n$~]}
  and \elpiIn{L} is  \elpiIn{[t~$_1$~, ~$\ldots$~, t~$_m$~]}.
  
% \linkbeta are put in \linkStore and activated when \rhs falls in \llambda. 
% \noteE[inline]{un po' ridondante con 8.2, non so se serva}

% This property can be relaxed, to accept approximations: a dedicated
% section is given in a future section

% From \cref{lemma:keep-fidelity}, we can deduce the following corollary

% \begin{corollary}
%   Given a \linkbeta \linkbetaM{\Gamma}{X}{T}, if it exists a second link
%   $l\in\linkStore$, then unification fails.   
% \end{corollary}

\subsection{Compilation and decompilation}

We insert this rule
just before rule~\ref{rule:compapp}:

\begin{elpicode}
  comp (o-app [o-uva A|Ag]) (uva B Sc) M1 M3 L1 L3 S1 S4 :- !,
    pattern-fragment-prefix Ag Pf Extra, alloc S1 B S2,
    len Pf Ar, m-alloc (ov A) (mv C (arity Ar)) M1 M2 S2 S3,
    fold6 comp Pf    Pf1    M2 M2 L1 L1 S3 S3,
    fold6 comp Extra Extra1 M2 M3 L1 L2 S3 S4,
    Beta = app [uva C Pf1 | Extra1],
    get-scope Beta Sc, 
    L3 = [val (link-llam (uva B Sc) Beta) | L2].
\end{elpicode}


\noindent
The list \elpiIn{Ag} is split into two parts: 
\elpiIn{Pf} is in \llambda, and can be empty; 
\elpiIn{Extra} cannot be empty and is such that
<<\elpiIn{append Pf Extra Ag}>>.
The \rhs of the \linkbeta is
the application of a fresh variable \elpiIn{C} having
in scope all names in \elpiIn{Pf1} (the compilation of \elpiIn{Pf}).
The variable \elpiIn{B}, returned as the compiled term, is a fresh variable
having in scope all the free variables occurring in \elpiIn{Pf1} and
\elpiIn{Extra1} (the compilation of \elpiIn{Extra}).
This construction enforces \fullref{inv:beta-rhs}.

% In the following, we pose $X_{s_1\dots s_n}$ to represent a variable $X$ with
% the list of distinct names $s_1\dots s_n$ as scope. Moreover, $X_{s_1\dots
% s_n}\appsep t_1\dots t_m$ is the application of that variable to the list of
% terms $t_1\dots t_m$. Note that \rhsBeta is equivalent to \elpiIn{app[uva N S |
% T]}.

% \noteD{Si puo togliere se serve spazio}
% \begin{corollary}
%   Let \rhsBeta be the \rhs of a \linkbeta, then $m > 0$.
% \end{corollary}

% \begin{comment}
% \begin{proof}[Proof sketch]
%   Assume we have a \linkbeta, by contradiction, if $m = 0$, then the original
%   \Fo{} term has the shape \elpiIn{o-app[o-uva M | Ag]} where \elpiIn{Ag} is a
%   list of distinct names (i.e. the list \elpiIn{Extra} is empty). This case is
%   however captured by rule~\ref{rule:complam} (from \cref{sec:compilation}) and
%   no \linkbeta is produced which contradicts our initial assumption.
% \end{proof}
% \end{comment}

% \noteD{Si puo togliere se serve spazio}
% \begin{corollary}
%   Let \rhsBeta be the \rhs of a \linkbeta, then $t_1$ either appears in $s_1\dots s_n$ or it is 
%   not a name.
% \end{corollary}

% \begin{comment}
% \begin{proof}[Proof sketch]
%   By construction, the lists $s_1\dots s_n$ and $t_1\dots t_m$ are built by splitting
%   the list \elpiIn{Ag} from the original term \elpiIn{o-app [o-uva A|Ag]}.
%   $s_1\dots s_n$ is the longest prefix of the compiled terms in \elpiIn{Ag} which is
%   in \llambda. Therefore, by definition of \llambda, $t_1$ must appear  
%   in $s_1\dots s_n$, otherwise $s_1\dots s_n$ is not the longest prefix in
%   \llambda, or it is a term with a constructor of \elpiIn{tm} as functor.  
% \end{proof}
% \end{comment}


% \begin{definition}[$\wellb_{\beta \eta \llambda}$]
%   $
%   \wellb_{\beta \eta}(X) \Leftrightarrow \forall t \in \subterm{X}, t \not\in \notllambda{}
%   $
% \end{definition}

\begin{lemma}[$\wellb$-enforcement]\label{lem:w-enforcement-maybellam}
  $\forall s, \C{s}{t}{m}{l}$, then $\:\wellb(t)$.
\end{lemma}

\begin{proof}[Proof sketch]
  By \cref{lem:w-enforcement-maybebeta,lem:w-enforcement-maybeeta} and the 
  rule above.
\end{proof}


\paragraph{Decompilation}
All \linkbeta should be solved before decompilation.
If any \linkbeta remains in \linkStore, decompilation fails.

\subsection{Progress}

\newcommand{\progBetaLL}{\emph{\llambda-progress-refine}\xspace}
\newcommand{\progBetaRH}{\emph{\llambda-progress-\rhs}\xspace}
% \newcommand{\progBetaDedup}{\emph{progress-\llambda-dedup}\xspace}
\newcommand{\progBetaFail}{\emph{\llambda-progress-fail}\xspace}

%We add the followins rules to \elpiIn{progress1}.
Let $l$ be a \linkbeta of the form \linkbetaM{\Gamma}{T}{\rhsBeta}

\begin{definition}[\progBetaLL]
  Let $\sigma$ be a substitution such that $\sigma t_1$ is a name $s$
  not occurring in $s_1\dots s_n$. If $m = 1$, then $l$ is removed and
  the \lhs is
  unified with $X_{s_1\ldots s_n~s}$.
  If $m > 1$, then $l$ is replaced by the link
  $\linkbetaM{\Gamma}{T}{Y_{s_1\dots s_n~s}\appsep t_2\dots t_m}$
  (where $Y$ is a fresh variable of arity $n+1$) and link
  \linketaM{\Gamma}{X_{s_1\dots s_n}}{\lambda x.Y_{s_1\dots s_n~x}}
  is added to \linkStore.   
  %
  % A link \linkbetaM{\Gamma}{T}{\rhsBeta} is removed from \linkStore, if given
  % the substitution $\sigma$, $\sigma t_1$ is a name, say $t$, such that, $t
  % \notin s_1\dots s_n$. In this case, let $Y$ a fresh variable, then
  % \linketaM{\Gamma}{X_{s_1\dots s_n}}{\lambda x.Y_{s_1\dots s_n,x}} is added to
  % \linkStore, and 1) if $m = 1$ then $Y_{s_1\dots s_n, t}$ is unified with $T$,
  % 2) otherwise, the refined \linkbeta, \linkbetaM{\Gamma}{T}{M_{s_1\dots
  % s_n,t}\appsep t_2\dots t_m}, is added to \linkStore.
  \label{def:progBetaLL}
  % % \noteE[inline]{non è chiaro. la sostituzione immagino sia la corrente, non se ne esiste una. Poi non è chiaro se otherwise si riferische a in this case. Usa un enumerate}
  % %\noteE[inline]{Is it clearer?}
  % \noteE[inline]{l1 or t1? Forse è più chiaro dire che il ilnk beta viene raffinato con link beta cone meno argomenti, Se poi il numero di argomenti extra è 0 il tutto è in llambda e quindi viene buttato via. In questo modo la terminazione è ancora più chiara perchè si vede già che prima decresce la lista di argomenti e poi il numero di beta}
  % \noteD{L'ho riformulato}
\end{definition}

\begin{definition}[\progBetaRH]
  Link $l$ is removed from
  \linkStore if \rhsBetaHead is instantiated to a term $t$ and
  $t \appsep t_1 \ldots t_m$ $\beta$-reduces to a $t' \in \llambda$.
  \label{def:progBetaRH}
\end{definition}
% Example of this `X a = Y a'
% \begin{definition}[\progBetaDedup]
%   Given a \linkbeta $l_1$ and second link $l_2 \in\linkStore$, such that they
%   share the same \lhs. The two \rhs are unified and $l_2$ is
%   removed from \linkStore.
%   \label{def:progBetaDedup}
% \end{definition}

\begin{definition}[\progBetaFail]
  \progress fails when either
  \begin{itemize}
  \item
  there exists another link $l' \in \linkStore$ with the same \lhs as $l$; or
  \item the \lhs of $l$
  become rigid.
  \end{itemize}
  \label{def:progBetaFail}
\end{definition}
\noindent
We relax this condition and accommodate for
% the implementation of 
heuristics in~\ref{sec:heuristics}.

Finally
%in order to correctly correlate a variable appearing in the \lhs of a
%\linketa and in the \rhs of a \linkbeta, 
we introduce the following
\linketa progress rule.

\newcommand{\progressetaright}{\emph{$\eta$-progress-\rhs}\xspace}
\begin{definition}[\progressetaright]\label{def:progressetaright} A link
\linketaM{\Gamma}{X}{T} is removed from \linkStore when either 
\begin{itemize}
  \item $\testmaybeeta~T$ does not hold (anymore), in this case $X$ is unified
        with $T$; or
  \item $T$ $\eta$-contracts to a term $T'$ not starting with the \elpiIn{lam}
        constructor. In this case $X$ is unified with $T'$
\end{itemize}
\end{definition}

The idea behind this rule is to instantiate the \lhs variable of a \linketa
whenever its \rhs is for sure a \wellb term.

\begin{lemma}
  \elpiIn{progress} terminates
\end{lemma}

\begin{proof}[Proof sketch]
  Let $l = \linkbetaM{\Gamma}{T}{\rhsBeta}$ a \linkbeta in the store \linkStore.
  The progression made by \progBetaRH{} terminates: $l$ is removed from
  \linkStore after having performed terminating instructions. If $l$ is not
  activated by \progBetaRH, then $X$ is a variables. The activation
  of \progBetaLL replaces $l$ with the new \linkbeta 
  $\linkbetaM{\Gamma}{T}{Y_{s_1\dots s_n~s}\appsep t_2\dots t_m}$.
  This progression can be triggered at most $m$ times, since at each time
  the number of terms applied to $X$ decreases. In the end $l_m$
  will be removed from \linkStore after the unification between \lhs with \rhs.
  We also note that each \progBetaLL generates a \linketa, yet, according to
  \cref{lemma:prog-eta-terminates}, this does not affect termination. 
  \progBetaFail terminates since it stop progression with failure.
  Finally, \progressetaright performs 
  terminating operations and, if its permises succeed, the considered
  \linketa is removed from \linkStore, ensuring termination.
  %
  % \Cref{def:progBetaRH} makes \elpiIn{progress} terminates, since it makes a
  % \linkbeta disappear from \linkStore. On the other hand, \cref{def:progBetaLL},
  % creates each time a new \linketa and replace the old \linkbeta with a new one.
  % This progression can however triggered at most $m$ times, since each new
  % \linkbeta as a smaller list of applied variables. At the $m^{th}$ progression,
  % the \linkbeta is removed from \linkStore which is now filled by $m$ new
  % \linketa. By \cref{lemma:prog-eta-terminates}, we know that \elpiIn{progress}
  % terminates if \linkStore is made by only \linketa and therefore
  % \elpiIn{progress} still terminates.
  %\noteE[inline]{is it ok?}
  % \noteE[inline]{funziona. per essere più precisi io parlerei di ordine lessicografico (tipico ordine ben fondato usato per dimostrare terminazione). Nl nostro caso è la tripla (argomenti extra dei beta, numero di beta, numero di eta).}
\end{proof}

% \begin{corollary}
%   Given a \linkbeta, the variables occurring in its \rhs are in \llambda.
%   % \noteE[inline]{vuoi dire: le variabili che occorrono nel rhs sono in...?}
%   \noteD[inline]{is it clearer?}
% \end{corollary}

% \begin{proof}[Proof sketch]
%   By construction, the \rhs of \linkbeta has the shape \rhsBeta, $s_1\dots s_n$
%   is in \llambda and all the terms $t_1\dots t_n$ are in \llambda, too. If a
%   \linkbeta is triggered by \progBetaRH, then, by \cref{def:progBetaRH}, that
%   link is removed by \linkStore, and the property is satisfied. If the \linketa
%   is activated by \progBetaLL, then, by \cref{def:progBetaLL}, the new \linkbeta
%   as a variable as a scope which is still in \llambda. 
% \end{proof}

\begin{theorem}[Fidelity in \Fo]
  The introduction of \linkbeta guarantees \fullref{prop:fidelity-recovery}
  if $\foUnifPb \not\subseteq \llambda$.
\end{theorem}

\begin{proof}[Proof sketch]
  Let $\foUnifPb_i$ be the first problem in \foUnifPb{} such that it is not in \llambda and let
  $\sigma$ be the substitution obtained solving
  $\hoUnifPb_1\ldots\hoUnifPb_{i-1}$. After the execution of \hstep
  for the $i-1$ time,
  each variable $X$ corresponding to a \maybeeta subterm in the original \Fo 
  unification problem, is instantiated iff it is known for sure
  that $X$ is no more in \maybeeta. If $\sigma\hoUnifPb_i$ is in \llambda, then 
  % each variable that the
  % compiler used to replace a ``problematic'' term is assigned and hence
  by \cref{def:progBetaLL,def:progBetaRH} the associated \linkbeta
  is solved and removed. In this case, all
  calls to \Ue{} are between terms in \llambda and by \cref{th:fidelity-beta}
  fidelity is guaranteed. If $\sigma\hoUnifPb$ is still in \notllambda,
  then by \cref{def:progBetaFail} unification fails as the
  corresponding unification in \Fo{} would (it is called outside its domain).
\end{proof}

\begin{comment}
\paragraph{Example of \progBetaLL}
Consider the \linkbeta below:
%
% make test ONLY=7004 TEX=tex
\printAlll
  {{{X,\lambda x.x},
    {\lambda x.(Y\appsep (X\appsep x)),f}}}
  {{{A,\lambda x.x},
    {B,f}}}
  {{{Y,D,0},
    {X,A,0}}}
  {{{\eta,,A,\lambda x.E_{x}},
    {\eta,,B,\lambda x.C_{x}},
    {\llambda,x,C_{x},(D\appsep E_{x})}}}

\noindent
Initially the \linkbeta \rhs is a variable $D$ applied to the $E_x$.
The first unification problem results in $\sigma =
\{A \mapsto \lambda x.x\}$. In turn this instantiation
triggers $\linkStore_1$ by \progressetaleft 
and $E_x$ is assigned to $x$.
Under this substitution the \linkbeta becomes
\linkbetaM{x}{C_x}{(D \appsep x)}, and by \progBetaLL
it is replaced with the link:
\linketaM{}{E}{\lambda x.D_x}, while $C_x$ is unified with $D_x$. The second unification
problem assigns $f$ to $B$, that in turn activates
the second \linketa ($f$ is assigned to $C$), and then all the remaining links
are solved. The final \Ho substitution is $\sigma = \{~A _{} \mapsto \lambda x.x ~;~
B _{} \mapsto f ~;~ C _{x} \mapsto (f\appsep x) ~;~
D _{} \mapsto f ~;~ E _{x} \mapsto x ~;~
F _{x} \mapsto C_{x} ~\}$ and is decompiled into $\rho = \{~X \mapsto \lambda x.x ~;~
Y \mapsto f~\}$.

\paragraph{Example of \progBetaRH}
We give a second look at the opening example in \cref{sec:beta}.
% \todo{this is a little more complex that the opening problem}
The problem is compiled into:
%
% make test ONLY=7003 TEX=tex
\printAlll
  {{{X,\lambda x.a},
    {(X\appsep a),a}}}
  {{{A,\lambda x.a},
    {B,a}}}
  {{{X,A,0}}}
  {{{\llambda,,B,(A\appsep a)}}}

% \todo{Rinominare i print dei problemi}

The first unification problem is solved with the substitution
$\sigma = \{~A \mapsto \lambda x.a~\}$. The \linkbeta becomes
\linkbetaM{}{B}{((\lambda x.a) \appsep a)} whose \rhs can be $\beta$-reduced to
$a$. $a$ is in \llambda and is unified with $B$. The resolution of the second
unification problem
confirm that $B$ equal $a$. The final substitution $\sigma = \{~A \mapsto \lambda x.a ~;~ B \mapsto a ~\}$ is decompiled into $\rho = \{~ X \mapsto \lambda x.a ~;~ Y \mapsto
a ~\}$.
\end{comment} 
\subsection{\texorpdfstring{Relaxing \fullref{def:progBetaFail}}{Relaxing progress-llam-fail}}
\label{sec:heuristics}
\newcommand{\progBetaNoLLWait}{\emph{progress-beta-\notllambda}}

% from https://www.cs.mcgill.ca/~bpientka/papers/unif_miller60.pdf
Working with terms in \llambda is sometime too restrictive~\cite{Abel2018ExtensionsTM}
and we could find in literature a few strategies to go beyond \llambda without 
implementing Huet's algorithm~\cite{Huet75}.
Some implementations of $\lambda$Prolog~\cite{lamProlog} such as
Teyjus~\cite{Nadathur2001} delay the resolution of \notllambda unification
problems until the substitution makes them reenter \llambda.
Other systems apply heuristics like preferring projection over mimic,
and commit to that solution. This is the case for
the unification algorithm Coq uses in its type-class
solver~\cite{sozeau08}, 

In this section we show how we can implement these strategies
by simply adding (or removing) rules to the
\elpiIn{progress} predicate.
In the example below $\foUnifPb_1$ is in \notllambda.
%
% make test ONLY=7005 TEX=tex
\printAlll
  {{{(X\appsep a),a},
    {X,\lambda x.Y}}}
  {{{A,a},
    {B,\lambda x.C}}}
  {{{Y,C,0},
    {X,B,0}}}
  {{{\llambda,,A,(B\appsep a)}}}

\noindent
If we want the object-language unification to delay the first unification problem
(waiting for $X$ to be be instantiated),
we can relax \cref{def:progBetaFail}. 
Instead of failing when the \lhs of $\linkStore_1$ becomes rigid (equal to $a$),
we keep it in \linkStore until the head of its \rhs also become rigid.
In this case, since both the \lhs and \rhs have rigid heads, they can be unified.
While this relaxed rule does not break \fullref{prop:fidelity} per se,
the \elpiIn{occur-check-links} procedure becomes incomplete since
\fullref{inv:linklhs} is broken. Also note that delaying unification outside \llambda can leave \linkbeta for the
decompilation phase. Therefore \elpiIn{commit-links} should
be modified accordingly.

If instead we want \Uo{} to follow the second strategy and pick an
arbitrary solution we can modify \progress by applying the desired heuristic
instead of failing.
For instance, in $X \appsep a \appsep b = Y \appsep b$, the last
argument of the two terms is the same and unification can succeed by assigning
$X a$ to $Y$. This heuristic is used by~\cite{sozeau08}.

%\begin{elpicode}
  type app-align-heuristic tm -> tm -> subst -> subst -> o.
  app-align-heuristic (app [HD|L1]) (app [UV|L2]) S1 S2 :-
    append Prefix L2 L1,
    ho.beta HD Prefix T,
    (UV ~$\Ue$~ T) S1 S2.
\end{elpicode}

%\noindent
%Note that \elpiIn{UV} is in \llambda as well as \elpiIn{T}. 
% \vspace{-1em}
\section{Actual implementation in Elpi}\label{sec:implementation}

In this paper we did study a compiler and a simulation loop on a minimal language.
The actual implementation uses the Coq-Elpi meta language to both compile
the sequence of problems (the rules) and execute them, that is Elpi
plays the role of \Ho{} as well.

The main difference is that we cannot implement \hrun since it is
the runtime of the programming language. In particular the runtime iterates
\Ue, but \hstep also needs to check links for \progress. Luckily Elpi extends
$\lambda$Prolog with constraints (suspended goals) and
Constraint Handling Rules (CHR) to operate on them~\cite{TASSI_2019,fruehwirth2017constraint}.
Similarly to \cite{pfenning1993} a constraint is a goal % unification problems
suspended on a list of variables that is resumed \emph{as soon as one of these variable
is assigned}, before any other existing goal is considered. 
In turn link activation grants~\fullref{prop:fidelity}.
We point out that \cite{pfenning1993} delays \notllambda unification problems
while we also delay problems that are in \maybeeta.
For brevity we only provide two pseaudo-code snippets. The first one depicts
how \linketa are suspended or make \progress.

\begin{elpicode}
link-eta L R :- not (var L), !, eta-progress-lhs L R.
link-eta L R :- not (maybe-eta R), !, eta-progress-rhs L R.
link-eta L R :- declare_constraint (link-eta L R) [L,R].
\end{elpicode}

\noindent
The second snippet illustrates the deduplication of \linketa. The syntax
<<\elpiIn{N ~$\triangleright$~ G ?- P}>> denotes a $\lambda$Prolog sequent, that is a 
goal \elpiIn{P} under a set \elpiIn{G} of hypothetical rules (introduced
by \elpiIn{=>}) and where the program context binds (via the \elpiIn{pi}
operator) eigenvariables in \elpiIn{N}. Sequents are presented to
CHR with their higher order unification variables replaced by
``frozen constants'', that is $A_{xyz}$ becomes <<\elpiIn{uvar ~$f_A$~ [x,y,z]}>>
for a fresh constant $f_A$. Frozen constants are ``defrost'' when
they are part of a new goal (see \cite[section 4.3]{TASSI_2019}).

\begin{elpicode}
  rule (N1 ~$\triangleright$~ G1 ?- link-eta (uvar X LX1) T1)   % match
    /  (N2 ~$\triangleright$~ G2 ?- link-eta (uvar X LX2) T2)   % remove
    |  (relocate LX1 LX2 T2 T2')               % condition
   <=> (N1 ~$\triangleright$~ G1 ?- T1 = T2').                  % new goal
\end{elpicode}

\noindent
The first directive matches a constraint
whose \lhs is a variable \elpiIn{X}; the second matches and removes
a constraint on the same variable; the third
relocates \elpiIn{T2} and the latter crafts the new goal that unifies
\elpiIn{T1} with \elpiIn{T2'} under \elpiIn{G1} and the set of
heigenvariables \elpiIn{N1}.

% Remark that \fullref{inv:uvaarity} makes it possible to relocate \elpiIn{T2}
% using the name-to-name mapping obtained by zipping
% the lists (of equal length) \elpiIn{LX1} and \elpiIn{LX2}.

\noindent
\section{Related work and conclusion}

Object-language terms can be unified using
different strategies.

The first approach that comes to mind consist in implementing \Uo as a
regular routine, i.e. write rules as follows

\begin{elpicode}
  decision X :- unify X (all A x\ app [P, x]), finite A,
    pi x\ decision (app [P, x]).
\end{elpicode}

\noindent
where \elpiIn{unify} is a regular predicate.
This method fails to take advantage of the logic
programming engine provided by the meta language since
it removes all the data from the rule's heads and makes
indexing degrade. Additionally,
implementing a unification procedure in the meta language is likely to be significantly
slower compared to the built-in one on the common domain of first order terms.

Another possibility is to avoid having application and abstraction nodes
in the syntax tree, and use the meta language ones, as in:

\begin{elpicode}
decision (all A x\ P x) :- finite A, pi x\ decision (P x).
\end{elpicode}

\noindent
However, this encoding has two big limitations. First it is
not always feasible to adopt it for Coq due to the fact that the type system
of the meta language is too limited to accommodate the one of the object language,
e.g. Coq can typecheck variadic functions~\cite{cpdt}. \\
Second, the encoding of Coq terms provided by Elpi is primarily used for meta
programming, i.e. to extend the Coq system. Consequently, it must be able to
manipulate terms that are not known in advance
without relying on introspection primitives such as Prolog's
\texttt{functor/3} and \texttt{arg/3}. To avoid that constants 
cannot be symbols of the meta language, but must rather
live in
an open world, akin to the \elpiIn{string} data type used in the \elpiIn{con}
constructor.

In the literature we could find a related encoding of the Calculus of
Constructions (CC)~\cite{felty93lics}. The goal of that work is to exhibit
a logic program performing proof checking for CC and hence relate the
proof system of intuitionistic higher-order logic (that animates $\lambda$Prolog
programs) with the one of CC. The encoding is hence tailored
toward a different goal, for example it utilizes three relations to represent the
equational theory of CC, and that choice alone makes things harder for us.
Section 6 contains a discussion about the use of the
unification procedure of the meta language in presence of non ground goals, but
the authors are not interested in exploiting a decidable fragment of
higher-order unification but are rather leaning towards an interactive
system where the user is given
control over the search procedure.

Another work that is, somewhat surprisingly, only superficially related to ours is the
type-class engine built in Isabelle's meta language. In~\cite{wenzel97}
classes identify (simple) types, they are not higher order predicates
as in Coq, hence the solver does not require a higher-order unification
procedure.\\
~\\
\emph{The approach presented in this paper}
%provides a third option that 
addresses all the concerns
mentioned earlier. It takes advantage of the
unification capabilities of the meta language
at the price of handling problematic sub terms on
the side.
As a result %of this choice 
our encoding takes
advantage of indexing data structures and
mode analysis for clause filtering.
It is worth mentioning that we replace terms with variables only
when it is strictly needed, leaving the rest of the term structure intact
and hence indexable. % by the meta language logic programming engine.
Some preliminary benchmarks on the Stdpp
library~\cite{JUNG_KREBBERS_JOURDAN_BIZJAK_BIRKEDAL_DREYER_2018} show
encouraging results: our type-class solver has some
overhead on small goals but is consistently faster than
the Coq one starting from goals made of $32$ nodes.
Moreover our approach is flexible enough to accommodate different strategies
and heuristics to handle terms outside the pattern fragment.
Finally, our encoding is sufficiently shallow to: 1)
allow to lift forms of static analysis for
the meta language, such as determinacy, also to the object
language; 2) take advantage of future improvements to the
logic-programming engine of Elpi, such as tabled search.

We considered mechanizing our results with Abella~\cite{gacek2008abella},
especially in the early phase of this work, but unfortunately we could not.
The main reason is that the logic of Abella does not let one quantify over
predicates nor use the cut operator. Eliminating cut or
higher order
combinators like \elpiIn{map} or \elpiIn{forall2} is surely
possible, but 
has a high cost in verbosity given how pervasively they are used.
Moreover, and more importantly, it creates a distance
between the verified and the actual code, which we found to
hinder our exploration. However, we did establish a
test suite with about a hundred cases.
It is available at \githubUrl.

% ---

% Benefits: less work, reuse efficient ho unif (3x faster), indexing,

% Future: tabling and static analysis (reuse for ML again).

% Very little is Coq specific. Applies to all OL that are not a subsystem of 
% HOL, or for ML that are used for meta programming.

\printbibliography

% \clearpage

% \section*{Appendix}

Note that \elpiIn{(a infix b) c d} de-sugars to \elpiIn{(infix) a b c d}.

Explain builtin name (can be implemented by loading name after each pi)

\section{The memory}

\begin{elpicode}
  kind addr type.
  type addr nat -> addr.
  typeabbrev (mem A) (list (option A)).
  type set? addr -> mem A -> A -> o.
  set? (addr A) Mem Val :- get A Mem Val.

  type unset? addr -> mem A -> o.
  unset? Addr Mem :- not (set? Addr Mem _).

  type assign-aux nat -> mem A -> A -> mem A -> o.
  assign-aux z (none :: L) Y (some Y :: L).
  assign-aux (s N) (X :: L) Y (X :: L1) :- assign-aux N L Y L1.

  type assign addr -> mem A -> A -> mem A -> o.
  assign (addr A) Mem1 Val Mem2 :- assign-aux A Mem1 Val Mem2.

  type get nat -> mem A -> A -> o.
  get z (some Y :: _) Y.
  get (s N) (_ :: L) X :- get N L X.

  type alloc-aux nat -> mem A -> mem A -> o.
  alloc-aux z [] [none] :- !.
  alloc-aux z L L.
  alloc-aux (s N) [] [none | M] :- alloc-aux N [] M.
  alloc-aux (s N) [X | L] [X | M] :- alloc-aux N L M.

  type alloc addr -> mem A -> mem A -> o.
  alloc (addr A as Ad) Mem1 Mem2 :- unset? Ad Mem1, 
    alloc-aux A Mem1 Mem2.

  type new-aux mem A -> nat -> mem A -> o.
  new-aux [] z [none].
  new-aux [A | As] (s N) [A | Bs] :- new-aux As N Bs.

  type new mem A -> addr -> mem A -> o.
  new Mem1 (addr Ad) Mem2 :- new-aux Mem1 Ad Mem2.


\end{elpicode}


\section{The object language}

\input{code/object\_language}

\section{The meta language}

\input{code/meta\_language}

\section{The compiler}

\begin{elpicode}
  kind arity type.
  type arity nat -> arity.
  kind ovariable type.
  type ov addr -> ovariable.
  kind mvariable type.
  type mv addr -> arity -> mvariable.
  kind mapping type.
  ~\PYG{k+kd}{type} \PYG{n+nf}{(<->)} \PYG{k+kt}{ovariable -> mvariable -> mapping}~.
  typeabbrev mmap (list mapping).

  typeabbrev scope (list tm).
  typeabbrev inctx ho.inctx.
  kind baselink type.
  type link-eta  tm -> tm -> baselink.
  type link-llam tm -> tm -> baselink.
  typeabbrev links (list (inctx baselink)).
  typeabbrev link (inctx baselink).

  macro @val-link-eta T1 T2 :- ho.val (link-eta T1 T2).
  macro @val-link-llam T1 T2 :- ho.val (link-llam T1 T2).


  type get-lhs link -> tm -> o.
  get-lhs (val (link-llam A _)) A.
  get-lhs (val (link-eta A _)) A.

  type get-rhs link -> tm -> o.
  get-rhs (val (link-llam _ A)) A.
  get-rhs (val (link-eta _ A)) A.


  type occurs-rigidly to -> to -> o.
  occurs-rigidly N N.
  occurs-rigidly _ (o-app [o-uva _|_]) :- !, fail.
  occurs-rigidly N (o-app L) :- exists (occurs-rigidly N) L.
  occurs-rigidly N (o-lam B) :- pi x\ occurs-rigidly N (B x).

  type reducible-to list to -> to -> to -> o.
  reducible-to _ N N :- !.
  reducible-to L N (o-app[o-uva _|Args]) :- !, 
    forall1 (x\ exists (reducible-to [] x) Args) [N|L]. 
  reducible-to L N (o-lam B) :- !, 
    pi x\ reducible-to [x | L] N (B x).
  reducible-to L N (o-app [N|Args]) :-
    last-n {len L} Args R,
    forall2 (reducible-to []) R {rev L}.

  type maybe-eta to -> list to -> o.                  ~\customlabel{rule:maybeeta}{(\maybeeta)}~
  maybe-eta (o-app[o-uva _|Args]) L :- !,
    forall1 (x\ exists (reducible-to [] x) Args) L, !. 
  maybe-eta (o-lam B) L :- !, pi x\ maybe-eta (B x) [x | L].
  maybe-eta (o-app [T|Args]) L :- (name T; T = o-con _),
    split-last-n {len L} Args First Last,
    none (x\ exists (y\ occurs-rigidly x y) First) L,
    forall2 (reducible-to []) {rev L} Last.


  type locally-bound tm -> o.
  type get-scope-aux tm -> list tm -> o.
  get-scope-aux (con _) [].
  get-scope-aux (uva _ L) L1 :- 
    forall2 get-scope-aux L R,
    flatten R L1.
  get-scope-aux (lam B) L1 :- 
    pi x\ locally-bound x => get-scope-aux (B x) L1.
  get-scope-aux (app L) L1 :- 
    forall2 get-scope-aux L R,
    flatten R L1.
  get-scope-aux X [X] :- name X, not (locally-bound X).
  get-scope-aux X [] :- name X, (locally-bound X).

  type names1 list tm -> o.
  names1 L :-
    names L1, 
    new_int N,
    if (1 is N mod 2) (L1 = L) (rev L1 L).

  type get-scope tm -> list tm -> o.
  get-scope T Scope :-
    get-scope-aux T ScopeDuplicata,
    undup ScopeDuplicata Scope.
  type rigid to -> o.
  rigid X :- not (X = o-uva _).

  type comp-lam (to -> to) -> (tm -> tm) -> 
    mmap -> mmap -> links -> links -> subst -> subst -> o.
  comp-lam F G M1 M2 L1 L3 S1 S2 :-
    pi x y\ (pi M L S\ comp x y M M L L S S) =>         ~\customlabel{rule:complamh}{(\ensuremath{H_\lambda})}~
      comp (F x) (G y) M1 M2 L1 (L2 y) S1 S2,
    close-links L2 L3.

  type close-links (tm -> links) -> links -> o.
  close-links (v\[X v|L v]) [abs X|R] :- close-links L R.
  close-links (_\[]) [].
  type comp to -> tm -> mmap -> mmap -> links -> links ->
    subst -> subst -> o.
  comp (o-con C) (con C) M M L L S S.
  comp (o-lam F) (uva A Scope) M1 M2 L1 L3 S1 S3 :-
    maybe-eta (o-lam F) [], !,
      alloc S1 A S2,
      comp-lam F F1 M1 M2 L1 L2 S2 S3,
      get-scope (lam F1) Scope, 
      L3 = [val (link-eta (uva A Scope) (lam F1)) | L2].
  comp (o-lam F) (lam F1) M1 M2 L1 L2 S1 S2 :-            ~\customlabel{rule:complam}{(\ensuremath{c_\lambda})}~
    comp-lam F F1 M1 M2 L1 L2 S1 S2.
  comp (o-uva A) (uva B []) M1 M2 L L S1 S2 :- 
    m-alloc (ov A) (mv B (arity z)) M1 M2 S1 S2.
  comp (o-app [o-uva A|Ag]) (uva B Ag1) M1 M2 L L S1 S2 :-
    pattern-fragment Ag, !,
      fold6 comp Ag Ag1 M1 M1 L L S1 S1,
      len Ag Arity, 
      m-alloc (ov A) (mv B (arity Arity)) M1 M2 S1 S2.
  comp (o-app [o-uva A|Ag]) (uva B Sc) M1 M3 L1 L3 S1 S4 :- !,
    pattern-fragment-prefix Ag Pf Extra, alloc S1 B S2,
    len Pf Ar, m-alloc (ov A) (mv C (arity Ar)) M1 M2 S2 S3,
    fold6 comp Pf    Pf1    M2 M2 L1 L1 S3 S3,
    fold6 comp Extra Extra1 M2 M3 L1 L2 S3 S4,
    Beta = app [uva C Pf1 | Extra1],
    get-scope Beta Sc, 
    L3 = [val (link-llam (uva B Sc) Beta) | L2].
  comp (o-app A) (app A1) M1 M2 L1 L2 S1 S2 :-            ~\customlabel{rule:compapp}{(\ensuremath{c_@})}~
    fold6 comp A A1 M1 M2 L1 L2 S1 S2.

  type alloc mem A -> addr -> mem A -> o.
  alloc S N S1 :- mem.new S N S1.

  type compile-terms-diagnostic 
    triple diagnostic to to -> 
    triple diagnostic tm tm -> 
    mmap -> mmap -> 
    links -> links -> 
    subst -> subst -> o.
  compile-terms-diagnostic (triple D FO1 FO2) (triple D HO1 HO2) M1 M3 L1 L3 S1 S3 :-
    beta-normal FO1 FO1',
    beta-normal FO2 FO2',
    comp FO1' HO1 M1 M2 L1 L2 S1 S2,
    comp FO2' HO2 M2 M3 L2 L3 S2 S3.

  type compile-terms 
    list (triple diagnostic to to) -> 
    list (triple diagnostic tm tm) -> 
    mmap -> links -> subst -> o.
  compile-terms T H M L S :- 
    fold6 compile-terms-diagnostic T H [] M_ [] L_ [] S_,
    print-compil-result T H L_ M_, 
    deduplicate-map M_ M S_ S L_ L.

  type make-eta-link-aux nat -> addr -> addr -> 
    list tm -> links -> subst -> subst -> o.
  make-eta-link-aux z Ad1 Ad2 Scope1 L H1 H1  :-
    rev Scope1 Scope, eta-expand (uva Ad2 Scope) T1,
    L = [val (link-eta (uva Ad1 Scope) T1)].
  make-eta-link-aux (s N) Ad1 Ad2 Scope1 L H1 H3 :-
    rev Scope1 Scope, alloc H1 Ad H2, 
    eta-expand (uva Ad Scope) T2,
    (pi x\ make-eta-link-aux N Ad Ad2 [x|Scope1] (L1 x) H2 H3),
    close-links L1 L2,
    L = [val (link-eta (uva Ad1 Scope) T2) | L2].

  type make-eta-link nat -> nat -> addr -> addr -> 
          list tm -> links -> subst -> subst -> o.
  make-eta-link (s N) z Ad1 Ad2 Vars L H H1 :- 
    make-eta-link-aux N Ad2 Ad1 Vars L H H1.
  make-eta-link z (s N) Ad1 Ad2 Vars L H H1 :- 
    make-eta-link-aux N Ad1 Ad2 Vars L H H1.
  make-eta-link (s N) (s M) Ad1 Ad2 Vars Links H H1 :-
    (pi x\ make-eta-link N M Ad1 Ad2 [x|Vars] (L x) H H1),
    close-links L Links.

  type deduplicate-map mmap -> mmap -> 
      subst -> subst -> links -> links -> o.
  deduplicate-map [] [] H H L L.
  deduplicate-map [((ov O <-> mv M (arity LenM)) as X1) | Map1] Map2 H1 H3 L1 L3 :-
    take-list Map1 ((ov O <-> mv M' (arity LenM'))) _, !,
    std.assert! (not (LenM = LenM')) "Deduplicate map, there is a bug",
    print "arity-fix links:" {ppmapping X1} "~!~" {ppmapping ((ov O <-> mv M' (arity LenM')))}, 
    make-eta-link LenM LenM' M M' [] New H1 H2,
    print "new eta link" {pplinks New},
    append New L1 L2,
    deduplicate-map Map1 Map2 H2 H3 L2 L3.
  deduplicate-map [A|As] [A|Bs] H1 H2 L1 L2 :- 
    deduplicate-map As Bs H1 H2 L1 L2, !.
  deduplicate-map [A|_] _ H _ _ _ :- 
    halt "deduplicating mapping error" {ppmapping A} {ho.ppsubst H}.
\end{elpicode}


\section{The progress function}

\begin{elpicode}
  
  macro @one :- s z.

  type contract-rigid list ho.tm -> ho.tm -> ho.tm -> o.
  contract-rigid L (ho.lam F) T :- 
    pi x\ contract-rigid [x|L] (F x) T. % also checks H Prefix does not see x
  contract-rigid L (ho.app [H|Args]) T :- 
    rev L LRev, append Prefix LRev Args,
    if (Prefix = []) (T = H) (T = ho.app [H|Prefix]).

  type progress-eta-link ho.tm -> ho.tm -> ho.subst -> ho.subst -> links -> o.
  progress-eta-link (ho.app _ as T) (ho.lam x\ _ as T1) H H1 [] :- !, 
    ({eta-expand T @one} ==l T1) H H1.
  progress-eta-link (ho.con _ as T) (ho.lam x\ _ as T1) H H1 [] :- !, 
    ({eta-expand T @one} ==l T1) H H1.
  progress-eta-link (ho.lam _ as T) T1 H H1 [] :- !, 
    (T ==l T1) H H1.
  progress-eta-link (ho.uva _ _ as X) T H H1 [] :- 
    contract-rigid [] T T1, !, (X ==l T1) H H1.
  progress-eta-link (ho.uva Ad _ as T1) T2 H H [@val-link-eta T1 T2] :- !, 
    if (ho.not_occ Ad H T2) true fail.

  type is-in-pf ho.tm -> o.
  is-in-pf (ho.app [ho.uva _ _ | _]) :- !, fail.
  is-in-pf (ho.lam B) :- !, pi x\ is-in-pf (B x).
  is-in-pf (ho.con _).
  is-in-pf (ho.app L) :- forall1 is-in-pf L.
  is-in-pf N :- name N.
  is-in-pf (ho.uva _ L) :- pattern-fragment L.

  type arity ho.tm -> nat -> o.
  arity (ho.con _) z.
  arity (ho.app L) A :- len L A.

  type occur-check-err ho.tm -> ho.tm -> ho.subst -> o.
  occur-check-err (ho.con _) _ _ :- !.
  occur-check-err (ho.app _) _ _ :- !.
  occur-check-err (ho.lam _) _ _ :- !.
  occur-check-err (ho.uva Ad _) T S :-
    not (ho.not_occ Ad S T).

  type progress-beta-link-aux ho.tm -> ho.tm -> 
          ho.subst -> ho.subst -> links -> o.
  progress-beta-link-aux T1 T2 S1 S2 [] :-  is-in-pf T2, !,
    (T1 ==l T2) S1 S2.
  progress-beta-link-aux T1 T2 S S [@val-link-beta T1 T2] :- !.

  type progress-beta-link ho.tm -> ho.tm -> ho.subst -> 
        ho.subst -> links -> o.
  progress-beta-link T (ho.app[ho.uva V Scope | L] as T2) S S2 [@val-link-beta T T2] :- 
    arity T Arity, len L ArgsNb, ArgsNb >n Arity, !,
    minus ArgsNb Arity Diff, mem.new S V1 S1,
    eta-expand (ho.uva V1 Scope) Diff T1,
    ((ho.uva V Scope) ==l T1) S1 S2.

  progress-beta-link (ho.uva _ _ as T) (ho.app[ho.uva Ad1 Scope1 | L1] as T1) S1 S3 NewLinks :-
    append Scope1 L1 Scope1L,
    pattern-fragment-prefix Scope1L Scope2 L2,
    not (Scope1 = Scope2), !,
    mem.new S1 Ad2 S2,
    len Scope1 Scope1Len,
    len Scope2 Scope2Len,
    make-eta-link Scope1Len Scope2Len Ad1 Ad2 [] LinkEta S2 S3,
    if (L2 = []) (NewLinks = LinkEta, T2 = ho.uva Ad2 Scope2) 
      (T2 = ho.app [ho.uva Ad2 Scope2 | L2], 
      NewLinks = [@val-link-beta T T2 | LinkEta]).

  progress-beta-link T1 (ho.app[ho.uva _ _ | _] as T2) _ _ _ :- 
    not (T1 = ho.uva _ _), !, fail.

  progress-beta-link (ho.uva _ _ as T) (ho.app[ho.uva _ _ | _] as T2) S1 _ _ :- 
    occur-check-err T T2 S1, !, fail.

  progress-beta-link T1 (ho.app[ho.uva _ _ | _] as T2) H H [@val-link-beta T1 T2] :- !.

  progress-beta-link T1 (ho.app [Hd | Tl]) S1 S2 B :-
    ho.beta Hd Tl T3, 
    progress-beta-link-aux T1 T3 S1 S2 B.

  type solve-link-abs link -> links -> ho.subst -> ho.subst -> o.
  solve-link-abs (ho.abs X) R H H1 :- 
    pi x\ ho.copy x x => (pi S\ ho.deref S x x) => 
      solve-link-abs (X x) (R' x) H H1,
    close-links R' R.

  solve-link-abs (@val-link-eta A B) NewLinks S S1 :- !,
    progress-eta-link A B S S1 NewLinks.

  solve-link-abs (@val-link-beta A B) NewLinks S S1 :- !,
    progress-beta-link A B S S1 NewLinks.

  type take-link link -> links -> link -> links -> o.
  take-link A [B|XS] B XS :- link-abs-same-lhs A B, !.
  take-link A [L|XS] B [L|YS] :- take-link A XS B YS.

  type link-abs-same-lhs link -> link -> o.
  link-abs-same-lhs (ho.abs F) B :- 
    pi x\ link-abs-same-lhs (F x) B.
  link-abs-same-lhs A (ho.abs G) :- 
    pi x\ link-abs-same-lhs A (G x).
  link-abs-same-lhs (@val-link-eta (ho.uva N _) _) (@val-link-eta (ho.uva N _) _).

  type same-link-eta link -> link -> ho.subst -> ho.subst -> o.
  same-link-eta (ho.abs F) B H H1 :- !, pi x\ same-link-eta (F x) B H H1.
  same-link-eta A (ho.abs G) H H1 :- !, pi x\ same-link-eta A (G x) H H1.
  same-link-eta (@val-link-eta (ho.uva N S1) A)
                (@val-link-eta (ho.uva N S2) B) H H1 :-
    std.map2 S1 S2 (x\y\r\ r = ho.copy x y) Perm,
    Perm => ho.copy A A',
    (A' ==l B) H H1.

  type solve-links links -> links -> ho.subst -> ho.subst -> o.
  solve-links [] [] X X.
  solve-links [A|L1] [A|L3] S S2 :- take-link A L1 B L2, !,
    same-link-eta A B S S1, 
    solve-links L2 L3 S1 S2.
  solve-links [L0|L1] L3 S S2 :- deref-link S L0 L,
    solve-link-abs L R S S1, !,
    solve-links L1 L2 S1 S2, append R L2 L3.
\end{elpicode}

\section{The decompiler}

\begin{elpicode}
  
  type abs->lam ho.assignment -> ho.tm -> o.
  abs->lam (ho.abs T) (ho.lam R)  :- !, pi x\ abs->lam (T x) (R x).
  abs->lam (ho.val A) A.

  type commit-links-aux link -> ho.subst -> ho.subst -> o.
  commit-links-aux (@val-link-eta T1 T2) H1 H2 :- 
    ho.deref H1 T1 T1', ho.deref H1 T2 T2',
    (T1' ==l T2') H1 H2.
  commit-links-aux (@val-link-beta T1 T2) H1 H2 :- 
    ho.deref H1 T1 T1', ho.deref H1 T2 T2',
    (T1' ==l T2') H1 H2.
  commit-links-aux (ho.abs B) H H1 :- 
    pi x\ commit-links-aux (B x) H H1.

  type commit-links links -> links -> ho.subst -> ho.subst -> o.
  commit-links [] [] H H.
  commit-links [Abs | Links] L H H2 :- 
    commit-links-aux Abs H H1, !, commit-links Links L H1 H2.

  type decompl-subst map -> map -> ho.subst -> 
    fo.subst -> fo.subst -> o.
  decompl-subst _ [A|_] _ _ _ :- fail.
  decompl-subst _ [] _ F F.
  decompl-subst Map [mapping (fv VO) (hv VM _)|Tl] H F F2 :- 
    mem.set? VM H T, !, 
    ho.deref-assmt H T TTT,
    abs->lam TTT T', tm->fm Map T' T1, 
    fo.eta-contract T1 T2, mem.assign VO F T2 F1,
    decompl-subst Map Tl H F1 F2. 
  decompl-subst Map [mapping _ (hv VM _)|Tl] H F F2 :- 
    mem.unset? VM H, decompl-subst Map Tl H F F2.

  type tm->fm map -> ho.tm -> fo.fm -> o.
  tm->fm _ (ho.con C)  (fo.fcon C).
  tm->fm L (ho.lam B1) (fo.flam B2) :- 
    pi x y\ tm->fm _ x y => tm->fm L (B1 x) (B2 y).
  tm->fm L (ho.app L1) T :- forall2 (tm->fm L) L1 [Hd|Tl], 
    fo.mk-app Hd Tl T.
  tm->fm L (ho.uva VM TL) T :- mem L (mapping (fv VO) (hv VM _)), 
    forall2 (tm->fm L) TL T1, fo.mk-app (fo.fuva VO) T1 T.

  type add-new-map-aux ho.subst -> list ho.tm -> map -> 
        map ->  fo.subst -> fo.subst -> o.
  add-new-map-aux _ [] _ [] S S.
  add-new-map-aux H [T|Ts] L L2 S S2 :- 
    add-new-map H T L L1 S S1, 
    add-new-map-aux H Ts L1 L2 S1 S2.

  type add-new-map ho.subst -> ho.tm -> map -> 
      map ->  fo.subst -> fo.subst -> o.
  add-new-map _ (ho.uva N _) Map [] F1 F1 :- 
    mem Map (mapping _ (hv N _)), !.
  add-new-map H (ho.uva N L) Map [Map1 | MapL] F1 F3 :-
    mem.new F1 M F2,
    len L Arity, Map1 = mapping (fv M) (hv N (arity Arity)),
    add-new-map H (ho.app L) [Map1 | Map] MapL F2 F3.
  add-new-map H (ho.lam B) Map NewMap F1 F2 :- 
    pi x\ add-new-map H (B x) Map NewMap F1 F2.
  add-new-map H (ho.app L) Map NewMap F1 F3 :- 
    add-new-map-aux H L Map NewMap F1 F3.
  add-new-map _ (ho.con _) _ [] F F :- !.
  add-new-map _ N _ [] F F :- name N.

  type complete-mapping-under-ass ho.subst -> ho.assignment -> 
    map -> map ->  fo.subst -> fo.subst -> o.
  complete-mapping-under-ass H (ho.val Val) Map1 Map2 F1 F2 :- 
    add-new-map H Val Map1 Map2 F1 F2.
  complete-mapping-under-ass H (ho.abs Abs) Map1 Map2 F1 F2 :- 
    pi x\ complete-mapping-under-ass H (Abs x) Map1 Map2 F1 F2.

  type complete-mapping ho.subst -> ho.subst ->   
    map -> map -> fo.subst -> fo.subst -> o.
  complete-mapping _ [] L L F F.
  complete-mapping H [none | Tl] L1 L2 F1 F2 :-   
    complete-mapping H Tl L1 L2 F1 F2.
  complete-mapping H [some T0 | Tl] L1 L3 F1 F3 :-
    ho.deref-assmt H T0 T,
    complete-mapping-under-ass H T L1 L2 F1 F2, 
    append L1 L2 LAll,
    complete-mapping H Tl LAll L3 F2 F3.

  type decompile map -> links -> ho.subst -> 
    fo.subst -> fo.subst -> o.
  decompile Map1 L HO FO FO2 :- 
    commit-links L L1_ HO HO1, !,
    complete-mapping HO1 HO1 Map1 Map2 FO FO1,
    decompl-subst Map2 Map2 HO1 FO1 FO2.
\end{elpicode}

\section{Auxiliary functions}

\begin{elpicode}
  type fold4 (A -> A1 -> B -> B -> C -> C -> o) -> list A -> 
    list A1 -> B -> B -> C -> C -> o.
  fold4 _ [] [] A A B B.
  fold4 F [X|XS] [Y|YS] A A1 B B1 :- F X Y A A0 B B0, 
    fold4 F XS YS A0 A1 B0 B1.
  
  type len list A -> nat -> o.
  len [] z.
  len [_|L] (s X) :- len L X.
  
  \end{elpicode}
  

\end{document}