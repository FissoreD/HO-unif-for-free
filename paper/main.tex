\documentclass[sigconf,natbib=false]{acmart}
\usepackage[]{biblatex}

\AtBeginDocument{%
  \providecommand\BibTeX{{%
    Bib\TeX}}}
\addbibresource{bib.bib}

\usepackage{myTools}
\usepackage{macros}

% TODO: set this fields
%\setcopyright{cc}
%\setcctype{by}
\copyrightyear{2024}
\acmYear{XXXX 2024}
\acmBooktitle{YYY}
\acmDOI{ZZZZZZZZZZZZ}    


% \xspaceaddexceptions{]\}}

\def\elpi{\proglang{elpi}}
\def\coqelpi{\proglang{coq-elpi}}
\def\lambdaprolog{\proglang{$\lambda$-prolog}}
\def\coq{\proglang{coq}}

\newcommand{\library}[1]{\textit{#1}\xspace}
\def\stdpp{\library{stdpp}}
\def\iris{\library{iris}}

\newcommand*{\acronym}[1]{\texttt{#1}\xspace}

\def\ol{\acronym{ol}} % object language
\def\ml{\acronym{ml}} % meta language
\def\lf{\acronym{lf}} % logical framework
\def\ho{\acronym{ho}} % higher order
\def\Forall{$\forall$}

\newcommand{\U}{\ensuremath{=_o}\xspace}
\newcommand{\Ue}{\ensuremath{=_\lambda}\xspace}
\newcommand{\Fo}{\ensuremath{\mathcal{F}_{\!o}\xspace}}
\newcommand{\Ho}{\ensuremath{\mathcal{H}_o}\xspace}

\newcommand*{\eqtau}{\ensuremath{\mathrel{\overset{\mathrm{\tau}}{=}}}}

\begin{document}

\title{HO unification from object language to meta language} 

\author{Enrico Tassi}
\email{enrico.tassi@inria.fr}
\affiliation{%
  \institution{Université Côte d'Azur, Inria}
  % \city{Nice}
  \country{France}}

\author{Davide Fissore}
\email{davide.fissore@inria.fr}
\affiliation{%
  \institution{Université Côte d'Azur, Inria}
  % \city{Nice}
  \country{France}}

\begin{abstract}
  Specifying and implementing a logic from scratch requires significant effort.
  Logical Frameworks and Higher Order Logic Programming Languages provide
  dedicated, high-level Meta Languages (ML) to facilitate this task in two
  key ways: 1) variable binding and substitution are simplified when ML binders
  represent object logic ones; 2) proof construction, and even proof search, is
  greatly simplified by leveraging the unification procedure provided by the ML.
  Notable examples of ML are Elf~\cite{elf}, Twelf~\cite{twelf},
  $\lambda$Prolog~\cite{miller_nadathur_2012} and
  Isabelle~\cite{10.1007/978-3-540-71067-7_7}
  which have been utilized to implement various formal systems such as
  First Order Logic~\cite{felty88cade},
  Set Theory~\cite{10.1007/BF00881873},
  Higher Order Logic~\cite{books/sp/NipkowPW02}, and even the Calculus of
  Constuctions~\cite{felty93lics}.
  
  The object logic we are interested in is Coq's~\cite{Coq-refman}
  Dependent Type Theory (DTT),
  for which we aim to implement a unification procedure \U using the ML
  Elpi~\cite{dunchev15lpar}, a dialect of $\lambda$Prolog.
  Elpi comes equipped with the equational theory \Ue, comprising
  $\eta\beta$ equivalence
  and higher order unification restricted to the pattern
  fragment~\cite{miller92jsc}.
  We want \U to feature the same equational theory as \Ue but on the
  object logic DTT. Elpi also comes with an encoding for DTT that works well
  for meta-programming~\cite{tassi:hal-01637063,tassi:hal-01897468,gregoire:hal-03800154,newtc}.
  Unfortunately this encoding, which we refer to as \Fo,
  ``underuses'' \Ue by restricting it to first-order unification problems only. 
  To address this issue, we propose a better-behaved encoding, \Ho,
  demonstrate how to map unification problems in \Fo
  to related problems in \Ho, and illustrate
  how to map back the unifiers found by \Ue, effectively implementing
   \U on top of \Ue for the encoding \Fo.
   
  We apply this technique to the implementation of a type-class~\cite{wadler89}
  solver for Coq~\cite{Coq-refman}.
  Type-class solvers are proof search procedures based on
  unification that back-chain designated lemmas, providing essential
  automation to widely used 
  Coq libraries such as Stdpp/Iris~\cite{JUNG_KREBBERS_JOURDAN_BIZJAK_BIRKEDAL_DREYER_2018}
  and TLC~\cite{10.1007/978-3-642-14052-5_15}. These two libraries
  constitute our test bed.
\end{abstract} 

\keywords{Logic Programming, Meta-Programming, Higher-Order Unification, Proof Automation}

\maketitle

\section{Introduction}
\label{sec:intro}

Specifying and implementing a logic from scratch requires significant effort.
Logical Frameworks and Higher Order Logic Programming Languages provide
dedicated, high-level Meta Languages (ML) to facilitate this task in two
key ways: 1) variable binding and substitution are simplified when ML binders
represent object logic ones; 2) proof construction, and even proof search, is
greatly simplified by leveraging the unification procedure provided by the ML.
Notable examples of ML are Elf~\cite{elf}, Twelf~\cite{twelf},
$\lambda$Prolog~\cite{miller_nadathur_2012} and
Isabelle~\cite{10.1007/978-3-540-71067-7_7}
which have been utilized to implement various formal systems such as
First Order Logic~\cite{felty88cade},
Set Theory~\cite{10.1007/BF00881873},
Higher Order Logic~\cite{books/sp/NipkowPW02}, and even the Calculus of
Constuctions~\cite{felty93lics}.

The object logic we are interested in is Coq's~\cite{Coq-refman}
Dependent Type Theory (DTT), and we want to code a type-class~\cite{wadler89}
solver for Coq~\cite{Coq-refman} using the Coq-Elpi~\cite{tassi:hal-01637063}
meta programming framework.
Type-class solvers are unification based proof search procedures
that combine a set of designated lemmas in order to providing essential
automation to widely used Coq libraries.

As the running example we take the \coqIn{Decide} type class,
from the Stdpp~\cite{JUNG_KREBBERS_JOURDAN_BIZJAK_BIRKEDAL_DREYER_2018}
library. The class identifies predicates equipped with a decision procedure.
The following three designated lemmas (called \coqIn{Instances} in the
type-class jargon) state that: 1) the type \coqIn{fin n}, of natural numbers
smaller than \coqIn{n} is finite; 2) the predicate \coqIn{nfact n nf},
linking a natural number \coqIn{n} to its prime factors \coqIn{nf}, is decidable;
3) the universal closure of a predicate has a decision procedure if the
predicate has and if its domain is finite.

\begin{coqcode}
Instance fin_fin n : Finite (fin n).
Instance nfact_dec n nf : Decision (nfact n nf).
Instance forall_dec A P : Finite A ~$\to$~ 
  ~$\forall$~x:A, Decision (P x) ~$\to$~ Decision (~$\forall$~x, P x).
\end{coqcode}

\noindent Under this context the type-class solver is able to prove
the the following statement automatically by back-chaining
the three instances.

\begin{coqcode}
  Check _ : Decision (forall y: fin 7, nfact y 3).
\end{coqcode}

\noindent
The encoding of DTT provided by Elpi, that we will discuss at length later in
\cref{sec:encodings,sec:lang-spec}, features the following term constructors:

\begin{elpicode}
kind tm type.
type lam tm -> (tm -> tm) -> tm. % lambda abstraction
type app list tm -> tm.          % n-ary application
type all tm -> (tm -> tm) -> tm. % forall quantifier
type c string -> tm.             % constants
\end{elpicode}

\noindent
\marginpar{TODO: explain HOAS}
Following this term encoding the three instances are represented by the
following rules:

\begin{elpicode}
finite (app[c"fin", N]).
decision (app [c"nfact", N, NF]).
decision (all A x\ app[P, x]) :- finite A,
  pi x\ decision (app[P, x]).
\end{elpicode}

\noindent
\marginpar{TODO: explain pi, cons}
Unfortunately this direct translation of the instances considers the
predicate \coqIn{P} as a first order term. If we try to backchain the
third rule on the encoding of the goal above:
  
\begin{elpicode}
decision (all (app[c"fin", c"7"]) y\
  app[c"nfact", y, c"3"]).
\end{elpicode}

\noindent
we fail because of this ``higher order'' unification problem (in DTT)
is phrased as a first order unification problem in the meta language.

\begin{elpicode}
app[c"nfact", y, c"3"] = app[P, y]
\end{elpicode}

\noindent
In this paper we study a more sophisticated encoding of Coq terms allowing
us to rephrase the problematic rule as follows:

\begin{elpicode}
decision (all A x\ Pm x) :- link Pm A P, finite A,
  pi x\ decision (app[P, x]).
\end{elpicode}

\noindent
This time \elpiIn{Pm} is an higher order unification variable (of type
\elpiIn{tm -> tm}). The resulting unification problem is now:

\begin{elpicode}
app[c"nfact", y, c"3"] = Pm y
\end{elpicode}

\noindent
That admits one solution:

\begin{elpicode}
Pm = y\ app[c"nfact", y, c"3"]
A = app[c"fin",c"7"]
\end{elpicode}
  
\noindent
Elpi succeeds in the application of the new rule and then runs
the premise \elpiIn{link Pm A P} that is in charge of bringing the
assignment back to the domain of Coq terms (the type \elpiIn{tm}):

\begin{elpicode}
P = lam A a\ app[c"nfact", a, c"3"]
\end{elpicode}

\noindent
This simple example is sufficient to show that the encoding we seek
is not trivial. Indeed the solution for \elpiIn{P} generates a
(Coq) $\beta$-redex in the second premise (under the \elpiIn{pi x}):

\begin{elpicode}
decision (app[lam A (a\ app[c"nfact", a, c"3"]), x])
\end{elpicode}

\noindent
In turn the redex prevents the second rule to backchain properly since
the following unification problem has no solution:

\begin{elpicode}
app[lam A (a\ app[c"nfact", a, c"3"]), x] =
app[c"nfact", N, NF]
\end{elpicode}

\noindent
This time the root cause is that the unification procedure of \Ue of the
meta language is not aware of the equational theory of the object logic \U,
even if both theories include $\eta\beta$-conversion and admit most general
unifiers for problems in the pattern fragment~\cite{miller92jsc}.

In this paper we discuss alternative encodings of Coq in
Elpi~\ref{sec:encodings}, then we identify a minimal language \Fo
in which the problems sketched here can be fully described.
We then detail an encoding \elpiIn{comp} from \Fo to \Ho (the language of
the meta language) and a decoding \elpiIn{decomp} to relate the unifiers
bla bla..

\section{Alternative encodings} %%%%%%%%%%%%%%%%%%%%%%
\label{sec:encodings}

Our choice of encoding of DTT may look weird to the reader familiar with
LF, since used a shallow encoding of classes and binders, but not of the
``lambda calculus'' part of DTT. Here a more lightweight encoding
that unfortunately does not fit our use case

\begin{elpicode}
finite (fin N).
decision (nfact N NF).
decision (all A x\ P x) :- 
  (pi x\ decision (P x)), finite A.
\end{elpicode}

but in DTT this is not always possible and not handy in our use case,
since the arity of constants is not fixed.

\begin{coqcode}
Fixpoint narr T n := 
  if n is S m then T -> narr T m else T.
Definition nsum n : narr nat (n+1).
Check nsum 2   8 9 : nat.
Check nsum 3 7 8 9 : nat.
\end{coqcode}
  
moreover we use the same encoding for meta programming, or even just to provide
hand written rules. We want to access the syntax of OL, so our embedding cannot
be that shallow. We want to keep it shallow for the binders, but we need
the c, app and lam nodes. % all ?

Another alternative

\begin{elpicode}
decision X :- unif X (all A x\ app[P, x]), 
  (pi x\ decision (app[P, x])), finite A.
\end{elpicode}

gives up all half of what the ML gives us. Moreover even if unif here embodies
the eq theory of DTT which is much stronger than the one of the ML, we don't need 
it. According to our experience eta beta suffice, but HO is needed.



Note that this~\cite{felty93lics} is related and make the
discrepancy between the types of ML and DTT visible. In this case
one needs 4 application nodes. Moreover the objective is an encoding
of terms, proofs, not proof search. Also note the conv predicate,
akin to the unif we rule out.

This other paper~\cite{10.1007/978-3-031-38499-8_25} should also be cited.

\section{Languages description}
\label{sec:lang-spec}

\setlength{\abovecaptionskip}{0pt plus 3pt minus 2pt}
% \begin{figure*}
%   \begin{minipage}{.40\textwidth}

%     \begin{elpicode}
%       # Common code
%       kind tm type.
%       type app list tm -> tm.
%       type lam (tm -> tm) -> tm.
%       type c string -> tm.
%     \end{elpicode}
    
%   \end{minipage}

%   \vspace{3pt}

%   \begin{tabular}{lr}
%     \begin{minipage}{.40\textwidth}

%       \begin{elpicode}
%         # OL code
%         type fo_uv  nat -> tm.
%         typeabbrev fo_subst list (option tm).
%       \end{elpicode}
%     \end{minipage}
%     &
%     \begin{minipage}{.40\textwidth}
      
%       \begin{elpicode}
%         # ML code
%         type uv  nat -> list tm -> tm.
%         typeabbrev subst list (option assmt).
%       \end{elpicode}
%     \end{minipage}
%   \end{tabular}
  
%   \caption{Language description}
%   \label{code:lang-descr}
%   \Description[lang-spec]{Language description}
% \end{figure*}

% The two equals: equal of OL and ML
% In ML we don't have eta beta -> only unification is allowed on variables


\begin{figure*}
  \begin{tabular}{c}
    \begin{minipage}{.80\textwidth}
      \begin{elpicode}
        type fo_equal subst -> tm -> tm -> o.
        % deref
        fo_equal S (uv N) T1 :- assigned? N S T, fo_equal S T T1.
        fo_equal S T1 (uv N) :- assigned? N S T, fo_equal S T1 T.
        % congruence
        fo_equal S (app L1) (app L2) :- forall2 (fo_equal S) L1 L2.
        fo_equal S (lam F1) (lam F2) :- pi x\ fo_equal S x x => fo_equal S (F1 x) (F2 x).
        fo_equal _ (c X) (c X).
        fo_equal _ (uv N) (uv N).
      \end{elpicode}
    \end{minipage}
  \end{tabular}
  \caption{Term equality}
  \label{code:term-equal}
  \Description[term-equal]{Term equality}
\end{figure*}

% \begin{figure*}
%   \begin{tabular}{c}
%     \begin{minipage}{.80\textwidth}
%       \begin{elpicode}
%         % beta
%         equal S (app [uv N|A]) T1 :- assigned? N S F, beta F A T, equal S T T1.
%         equal S T1 (app [uv N|A]) :- assigned? N S F, beta F A T, equal S T1 T.
%         equal S (app [lam X | TL]) T :- beta (lam X) TL T', equal S T' T.
%         equal S T (app [lam X | TL]) :- beta (lam X) TL T', equal S T T'.
%         % eta
%         equal S (lam F) T :- not (T = lam _),
%           pi x\ beta T [x] (T' x), equal S (lam F) (lam T').
%         equal S T (lam F) :- not (T = lam _),
%           pi x\ beta T [x] (T' x), equal S (lam T') (lam F).
%       \end{elpicode}
%     \end{minipage}
%   \end{tabular}
%   \caption{$\eta\beta$ in the OL}
%   \label{code:eta-beta}
%   \Description[Eta-beta-OL]{Eta-beta in OL}
% \end{figure*}

% Description of languages (OL and ML)

\def\eqfo{eq\_fo\xspace}

In order to reason about unification of the terms of an objet language within a
meta language, we start by formally describing the two languages. Employing
meta-programming for this purpose, \cref{code:lang-descr} presents, on the left,
the typing structure of object language terms along with the signature of the
\eqfo function.

In this encoding, the fo\_tm type stands for the first order representation of
the object language terms. We model unification variables as integers
corresponding to memory addresses. The memory is represented as list of optional
terms (the subst\_fo type abbreviation). If the cell $i$ is none, then the
variable $i$ is not instantiated, otherwise it has already been assigned to some
value. 

% Why the fo prefix for fo_tm && Uvar in OL have no scope
Because of our translation of the terms of the object language into the meta
language outlined in \cref{sec:intro}, the variables of the object language have 
no scope.
An illustration of this observation can be found by analyzing deeply the subterm \coqIn{P x} from
the instance forall\_dec. Here, \coqIn{P} is an higher order variables
with type \coqIn{A ~$\to$~ Prop} and
\coqIn{x} is a name bound to \coqIn{P}.
However, at the meta level, the translation of \coqIn{P x} becomes \elpiIn{app[uv
1, x]}, that is, \elpiIn{uv 1} cannot reference \elpiIn{x}.

As our objective is to handle higher-order variables, a viable solution involves
addressing this issue through the compilation of the term received as input into
a more expressive version. This can be achieved by employing a second type
structure for the terms of the OL, as outlined in the second column of
\cref{code:lang-descr}. The primary difference lies in the fact that a
unification variable (always identified by an integer) now takes a list of terms
representing its scope. The abstractions of the ML are
symbolized using the constructor \elpiIn{abs} of the type \elpiIn{assmt}. The
constructor \elpiIn{val} is utilized to contain terms of the object language,
and in this context, there is no need for an application node in the ML.

\def\eqfo{equal\_fo\xspace}
\def\eqho{equal\_ho\xspace}
\def\etabeta{$\eta\beta$\xspace}

% We have two different equal functions
The equality relation over terms under a given substitution mapping is possible 
thanks to the equal predicate. There exist
two version of this predicate: \eqfo and \eqho. Their implementation is given in
\cref{code:term-equal}. 

The first tests if two terms are equal in the object language, that is, given a
substitution $\theta$, is it true that the two terms are equal? Note that no
variable instantiation is done: unification is not performed by the
\elpiIn{equal} predicate. Therefore, a unification variable $i$ equals to
another term $t$ if $t$ is a variable with same index $i$, or if $i$ is assigned
to $t'$ in $\theta$ and \elpiIn{equal ~$t'$~ ~$t$~}. Moreover, since \eqfo 
represents the equality of the OL, and since the OL accept term equality up to
\etabeta, then \eqfo also quotient over these reductions.

The \eqho predicate tests if two terms are equal in the world of the meta
languages. This equality function is not capable to understand the \etabeta
reductions over terms of the object language. The ML sees a term of the type
\elpiIn{tm} as atom on which only structural equality can be performed. It is
only possible to dereference a variable if assigned.
\marginpar{TODO: say HO is same as ML}

\subsection[Compilation: fo\_tm to tm]{Compilation: \elpiIn{fo_tm} $\to$ \elpiIn{tm}}

- how we transform an fo\_tm in tm
- the role of links

\subsection{Unification in ML}

- we accept HO unif with PF

\subsection{Term de-compilation}
% Compile FO to HO

% Assignement in HO should be decompiled




-------------------------------------------------------------------

\noindent


\subsection{implementation}

l'HO encoding e' esattamente lambda Prolog/elpi, il compilatore
in pratica potrebbe essere scritto in un meta language, qui lo si presenta in
elpi stesso. HO e FO in questo paper sono deep embedded in elpi per parlarne,
ma in pratica il nostro solever, prendere XX, scrivere la clausola compilata.


\section{recovering eta}

\begin{elpicode}
q (all x\ F x) = q (all x\ app[f,x]) /\ p f = p F
F = fun a => app [f,a] ----> F = f
\end{elpicode}

l'utene da p su f, mentre l'istanza pe q forza F a fun .. 

\section{recovering beta}

\begin{elpicode}
  q (all x\ F x) = q (all x\ app[f,x,x]) /\ p1 (app[f,a,a]) = p1 (app[F,a])
  F = fun y => app [f,y,y] ----> (app[F,a]) ~> app[f, a, a].
\end{elpicode}
  
qui la sintesi di F puo generare un beta redex, quindi ci mettiamo
p1 F1, e decomp beta F [a] F1.

\section{recovering eta-beta within unification (non linear variables)}

se i problemi di cui sopra avvengono nello stesso termine

\begin{elpicode}
  q2 (all x\ F x) (app[F,a]) = q2 (all x\ app[f,x,x]) (app[f,a,a])
\end{elpicode}

bisogna slegare le due F e poi unificare le soluzioni tra di loro

\section{heuristic / binary app}

fo approx / sub pattern fragment

\begin{elpicode}
  p (all x\app[F,x,a]) (app[F,b]) = p (all x\app[f,x,x,a]) (app[f,b,b])
  p (all x\G x) F' = 
  G = x\ f x x a
  F = lam x\f x x
  F' = (app[f,b,b])
  link (F a) F'
  link G F
\end{elpicode}


% questo esempio recupera l'app binaria perche

\begin{elpicode}
  (app (app F x) a) = (app (f x x) a)
\end{elpicode}

\printbibliography

\end{document}